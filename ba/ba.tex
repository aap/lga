\documentclass[12pt,a4paper,normalheadings]{scrartcl}
\usepackage[a4paper,top=3cm, bottom=3cm, left=4cm, right=3cm]{geometry}

\usepackage{polyglossia}
\defaultfontfeatures{Mapping=tex-text}
\setmainlanguage{german}
\setotherlanguage[variant=ancient]{greek}
\input{$HOME/griechisch/greektokens.tex}
\setmainfont{Junicode}
\setmonofont[Scale=MatchLowercase]{DejaVu Sans Mono}
%\setmonofont{Courier New}
\newfontfamily\greekfont{GFS Neohellenic}

\usepackage{setspace}
\usepackage{calc}

\renewcommand{\theparagraph}{\arabic{paragraph}}
\setcounter{secnumdepth}{5}

\newcounter{para}
\newcommand{\mypara}[1]{\vspace{1em}\par\noindent\refstepcounter{para}%
\textbf{\textsf{\thepara}}\enspace\textsf{#1}\quad}

\usepackage{enumitem}
\setlist{parsep=0pt,listparindent=\parindent}

\usepackage[backend=bibtex,style=authoryear,maxcitenames=1,language=german]{biblatex}
\bibliography{ba}

%\usepackage[texcoord,gridunit=cm]{showframe}

\usepackage{xparse}

\NewDocumentCommand{\lex}{mmg}{%
  \medskip
  \noindent
  \IfNoValueTF{#3}
    {%
      #1 > #2%
    }
    {%
      #1 > #2 \textbackslash\ #3%
    }%
}

\newcommand{\up}[1]{\textsuperscript{#1}}
\newcommand{\ul}[1]{\underline{#1}}
\newcommand{\ur}[1]{\up{\ref{#1}}}
\def\rek#1{*\textit{#1}}
\def\bel#1{\textit{#1}}
\def\tt#1{\texttt{#1}}

\begin{document}
\onehalfspacing

\section{Einleitung}

Ziel dieser Arbeit ist es, ein Program zu entwickeln,
das Lautgesetze auf Wörter anwendet,
und die Lautgesetze vom Urindogermanischen in die altgriechischen Dialekte
zu sammeln und zu sortieren.

Das junggrammatische Postulat der Ausnahmslosigkeit der Lautgesetze,
das fordert, dass Lautgesetze wie mechanisch ablaufen,
wirkt wie eine Einladung,
tatsächlich einen Mechanismus zu entwickeln,
der die Anwendung von Lautgesetzen simuliert.
Im 19. Jh. war dies natürlich unmöglich,
aber trotz der Computerrevolution des 20. Jh. hat sich offenbar
noch niemand daran gemacht,
diese eigentlich sehr naheliegenden Idee zu verwirklichen.
(TODO: doch, Amir Zeldes ansatzweise)

Eine solche Lautwandelsimulation erfordert,
dass die Lautgesetze präzise und in korrekter Chronologie formuliert werden,
und erlaubt sowohl,
selbige Aufstellung ohne großen Aufwand an beliebig vielen Testwörtern
zu überprüfen,
als auch belegte Wörter auf ihre Lautgesetzlichkeit hin zu überprüfen.

Das hier vorgestellte Programm \tt{lga} ist
der Nachfolger eines Prototyps
(komplett in \tt{sed} geschrieben),
der primitiver war und einige Mängel aufzeigte.
Ebenso ist aber \tt{lga} nur als Prototyp
für ein elaborierteres Programm zu verstehen.
Im Laufe der Arbeit wird sich herausstellen,
welche Probleme, Mängel und weiteren Anforderungen es gibt,
die man in einer nächsten Version angehen müsste.

Im ersten Teil dieser Arbeit werde ich \tt{lga}
in seinen Designideen und in seiner Funktionsweise vorstellen.
Im zweiten Teil werde ich mit Hilfe von \tt{lga} eine relative Chronologie der
Lautgesetze, die vom Urindogermanischen in die altgriechischen Dialekte führen,
aufstellen.
Hierbei ist \tt{lga} prinzipiell natürlich nicht notwendig,
aber es ist eine große Hilfe dabei Fehler und Ungenauigkeiten zu finden,
die sonst nicht aufgefallen wären.

\section{Zum Lautwandel allgemein}
TODO. Junggrammatiker, Katicic, ...

\section{lga}

\subsection{Vorüberlegungen}

Wenn ein Lautgesetzt wirkt,
wird ein Laut oder eine Lautgruppe durch eine andere oder nichts ersetzt.
Da man Laute phonetisch notieren kann,
kann man auch ein Lautgesetz durch Ersetzung
von Zeichen durch andere Zeichen beschreiben.

Da Lautwandel meist durch die Lautumgebung bedingt ist
und sich oft nicht nur ein Laut sondern Lautklassen verändern
(bspw. Stimmloswerdung von stimmhaften Lauten),
ist es sinnvoll,
eine Sprache zu entwicklen,
die Lautgesetze kompakt beschreiben
und von einem Computerprogramm verstanden werden kann.
Eine naheliegende Wahl ist daher eine Form von \emph{regulären Ausdrücken},
die in vielen Programmen implementiert und weithin bekannt sind.
Reguläre Ausdrücke sind zwar für diesen Zweck nicht perfekt geeignet,
aber für einen Prototyp ausreichend.
Wenn sich gezeigt hat,
welche Defizite reguläre Ausdrücke für diesen Anwendungsfall haben,
wird man eine besser geeignete Sprache entwickeln und implementieren können.

Als Programmiersprache habe ich die Scheme-Implementation CHICKEN
\footnote{\tt{http://www.call-cc.org/}}
und die \tt{irregex}-Engine für reguläre Ausdrücke
\footnote{\tt{http://wiki.call-cc.org/man/4/Unit\%20irregex}}
gewählt.

\subsection{Reguläre Ausdrücke}

Hier eine (informelle)
Beschreibung der hier verwendeten Untermenge regulärer Ausdrücke.
Die vollständige Dokumentation für die \tt{irregex}-Engine für
CHICKEN findet sich unter dem oben genannten Link.

Die grundlegende Funktion von regulären Ausdrücken ist das
Finden (engl. \bel{match}) von Zeichenketten (String) anhand eines Musters.
Der gematchte Text kann dann durch einen anderen ersetzt werden,
der Teile des Matches wieder aufnehmen kann.

Ein regulärer Ausdruck matcht einen String aus einem Alphabet.
Dies ist hier die Menge der Unicode \emph{Codepoints}, nicht die
der Unicode \emph{Grapheme},
welche nämlichen aus mehreren Codepoints bestehen können.

Ein Zeichen des Alphabets, das kein Metazeichen ist (dazu unten),
matcht sich selbst.
Metazeichen verlieren ihre spezielle Funktion,
wenn ihnen ein Backslash \tt{\textbackslash} vorangeht.
Der Ausdruck \tt{foo} matcht also den String \tt{foo}
(und auch das \tt{foo} in \tt{foobar}).

Der Punkt \tt{.} matcht ein beliebiges Zeichen.
Der Ausdruck \tt{...} matcht also alle Strings mit drei Zeichen,
\tt{\textbackslash.\textbackslash.\textbackslash.} matcht drei Punkte.

Eine in eckigen Klammern stehende Menge an Zeichen matcht eins dieser Zeichen,
oder alle außer dieser Zeichen,
wenn das erste Zeichen ein Zirkumflex \tt{\textasciicircum} ist.
Dabei können auch Zeichenbereiche mit Bindestrich angegeben werden.
\tt{[abc]} matcht also \tt{a}, \tt{b} oder \tt{c},
\tt{[\textasciicircum{}abc]} alle Zeichen außer \tt{a}, \tt{b} und \tt{c}.
Alternativ hätte man auch \tt{[a-c]} schreiben können.

Der Zirkumflex \tt{\textasciicircum} und das Dollarzeichen \tt{\$}
matchen den Anfang bzw. das Ende einer Zeile.
\tt{\textasciicircum{}foo} matcht also \tt{foo} am Anfang der Zeile,
\tt{bar\$} matcht \tt{bar} am Ende der Zeile
und \tt{\textasciicircum{}quux\$} die Zeile, die nur \tt{quux} enthält.
Da in \tt{lga} jedes Wort in einer eigenen Zeile steht,
kann man die Zeichen für den Anfang bzw. das Ende des Wortes benutzen.

\tt{?}, \tt{*} und \tt{+} sind Quantoren
und matchen, was ihnen vorangeht, null oder einmal, null bis unendlich mal
und ein bis unendlich mal.
Der Ausdruck \tt{.?[a-d]*x+} z.B.
matcht optional ein beliebiges Zeichen, dann eins der Zeichen \tt{a-d}
null bis unendlich mal und schließlich mindestens ein \tt{x}.

Mit Klammern \tt{()} werden analog zur mathematischen Notation
Matches enger gruppiert.
So würde \tt{[ab]c+} \tt{a} oder \tt{b} und dann mindestens ein
\tt{c} matchen (also z.B. \tt{accc} oder \tt{bc}).
\tt{([ab]c)+} dagegen matcht \tt{[ab]c} mindestens einmal
(also z.B. \tt{acbcbc}).

Neben ihrer Gruppierungsfunktion werden die Submatches innerhalb
der Klammern für die Ersetzung gespeichert.
\tt{([0-9]+)foo([0-9]+)} würde also zwei Zahlen und \tt{foo} dazwischen
matchen.
Die Zahlen wären dann in den Submatches \tt{1} und \tt{2} gespeichert
und können im Ersetzungstext wieder aufgegriffen werden.
Dabei gilt, dass verschachtelte Klammerausdrücke vor Klammern derselben
Ebene numeriert werden. Also wären die Submatches in z.B. \tt{(a(b)(c))(x)}
der Reihe nach \tt{abc}, \tt{b}, \tt{c}, \tt{x}.

Mit dem senkrechten Strich \tt{e0|e1} wird \tt{e0}
oder \tt{e1} gematcht (der Senkrechtstrich hat die niedrigste Präzedenz).
\tt{foo|bar+} matcht also \tt{foo} oder mindestens ein \tt{bar}.
\tt{(foo|bar)+} dagegen matcht mindestens ein \tt{foo} oder \tt{bar}.

\subsection{Funktionsweise}

Im Kern arbeitet \tt{lga} mit einer Liste von Wörtern
und einer Liste von Lautgesetzen
(in der Form von Textersetzung durch reguläre Ausdrücke)
und wendet der Reihe nach alle Lautgesetze auf die Liste der Wörter an.
Das Resultat sind die Wörter nach Anwendung aller Lautgesetze.

Die Liste der Lautgesetze aber hängt davon ab,
von welcher in welche Sprache die Wörter transformiert werden sollen
und wird aus einem \emph{Baum} von Lautgesetzen und einem
zugehörigen Sprachstammbaum generiert.

Die Liste der Wörter wird vorher durch eine weitere Liste von Regeln,
die von der Ausgangssprache abhängt,
von einer graphematisch/morphonologischen
Form in eine interne phonetische gebracht
(was natürlich eine eindeutige Graphie voraussetzt)
und am Ende durch noch eine Liste von Regeln,
die von der Zielsprache abhängig ist,
wieder in eine graphematische Darstellung gebracht.

Die Funktion \tt{(run-list from to words)}
generiert die drei Listen von Regeln anhand der Sprachen \tt{from} und \tt{to},
macht aus der Liste von Wörtern \tt{words} aus Effizienzgründen einen
einzigen String,
in dem jedes Element von \tt{words} in einer eigenen Zeile steht,
wendet darauf die drei Regellisten an,
und gibt das Ergebnis zurück.

Zunächst wird mit der Funktion \tt{(make-path tree start end)} anhand
eines Sprachstammbaums \tt{tree}
eine Liste aller Sprachen von der Ausgangs- zur Zielsprache generiert.
Der Aufruf \tt{(make-path lang-tree 'uridg 'lesb)} würde bspw.
die Liste \tt{(uridg urgr nwgr aiol lesb)} erzeugen.

Der Sprachstammbaum (gespeichert in der globalen Variable \tt{lang-tree})
definiert das Verwandtschaftsverhältnis von Sprachen.
Wenn eine Sprache keine Untersprachen hat,
ist sie (in Scheme-Terminologie) ein Symbol
(also hat z.B. \tt{myk} keine Abkömmlinge).
Hat eine Sprache Untersprachen,
ist sie eine Liste, dessen erstes Element ein Symbol für die Sprache selber ist;
die weiteren Elemente sind Untersprachen (wiederum Symbole oder Listen).
Ein vereinfachter Stammbaum könnte also bspw. folgendermaßen aussehen:
\begin{verbatim}
(urgr (nwgr (dor nwdor sardor sdor)
            (aiol boiot (thess wthess othess) lesb))
      (sogr (ach myk ark kypr pamph)
            (ion-att ion att euboi)))
\end{verbatim}
Es sei angemerkt,
dass die so beschriebenen Verhältnisse nicht streng als genetische
Verwandtschaft interpretiert oder zumindest verwendet werden sollten.
%
%TODO: besser schreiben
%
Att. und ion. mögen zwar auf ein Urion. zurückgeführt werden,
aber auch nach ihrer Aufspaltung teilen sie noch gewisse Entwicklungen.
Ebenso gibt es gemeingriechische Entwicklung die nach-urgriechisch sind.
Eine präzisere Trennung von Verwandtschaft und ``Sprachgruppen''
würde aber das Modell möglicherweise zu sehr verkomplizieren.
Die beste Möglichkeit ist es womöglich,
auf die Vorsilbe \textit{ur-} zu verzichten und bspw. \tt{gr}
gleichermaßen für urgriechisch und gemeingriechisch zu verwenden
(ebenso wäre statt \tt{urion} besser \tt{ion-att} zu verwenden).
%
%

Der Lautgesetzbaum ist eine Liste von vier möglichen Elementen, nämlich
Funktionen, die einen String als Argument nehmen und einen String zurückgeben
(also Funktionen, die Lautgesetze anwenden),
Symbole, die eine erreichte Sprachstufe markieren,
und zwei Arten von Listen des Typs \tt{(br|sub lang-lst rule-tree)},
deren zweites Element eine Liste von Sprachen ist,
für die sie gelten (bzw. nicht gelten, wenn das erste Element \tt{not} ist),
und deren drittes Element ein weiterer Lautgesetzbaum ist.
\tt{br} (für \emph{branch}) leitet einen neuen Sprachzweig ein,
der durch \tt{rule-tree} definiert ist
(d.h. dass alle Regeln nach einem erfolgreichen \tt{br} ignoriert werden).
\tt{sub} (für \emph{subrules}) funktioniert ähnlich,
macht aber nach Abarbeitung von \tt{rule-tree}
nach der Regel weiter.
Dies impliziert,
dass es sich hier bei \tt{rule-tree} nicht um einen echten Baum handelt.
\tt{sub} ist nur dazu gedacht eine Liste von Lautgesetzen bedingt anzuwenden.
Im Griechischen ist dies besonders nützlich,
da zwar alle Dialekte ähnliche Entwicklungen machen,
diese sich aber im einzelnen unterscheiden.
Mit \tt{sub} kann man einfach Regeln für bestimme Dialekte oder
Dialektgruppen definieren ohne für jeden Dialekt einen eigenen Sprachzweig
haben zu müssen.
% TODO: problem von oben wieder aufgreifen

Die Funktion \tt{(make-rules tree path)}
gibt anhand des Lautgesetz\emph{baums} \tt{tree} eine \emph{Liste}
von Lautgesetzen zurück,
die die Sprachentwicklung in \tt{path} (von \tt{make-path} erzeugt, s.o.)
reflektiert.
Von \tt{(apply-rules rules words)} wird dann diese Liste \tt{rules}
auf die Wörter \tt{words} angewandt.

Die Lautgesetzfunktionen werden von der Funktion \tt{(s pattern . subst)}
erzeugt,
die aus einem String \tt{pattern} einen regulären Ausdruck macht
und eine Funktion zurückgibt,
die \tt{irregex-replace/all} auf diesen regulären Ausdruck
und die Ersetzungen in der Liste \tt{subst} anwendet.
Matcht der reguläre Ausdruck einen Teil der Eingabe,
wird diese durch \tt{subst} ersetzt,
dessen Elemente Strings, Zahlen als Indizes der Submatches, sowie
Funktionen, die anhand eines Matches Strings zurückgeben, sind.
\tt{(irregex-replace/all "(foo)(bar)" "foobar" 2 (lambda (m)
    (string-reverse (irregex-match-substring m 1))) "quux")}
z.B. würde den String \tt{"foobar"} durch \tt{"baroofquux"} ersetzen.

Die oben beschriebenen regulären Ausdrücke
sind für die Modellierung von Laugesetzen nicht ausreichend.
Neben anderen Unzulänglichkeiten,
die sich im Laufe der Arbeit herausstellen werden,
gibt es eindeutig Probleme, mit Lautklassen umzugehen,
und noch allgemeiner, Laute durch Zeichen zu kodieren.

Wenn jeder Laut durch genau ein Zeichen bzw. einen Unicode Codepoint kodiert wird,
gibt es der Lautkodierung keine Probleme.
Da es allerdings sinnvoll ist,
einige Laute mit mehreren Codepoints zu kodieren,
kann es zu Problemen kommen.
Will man bspw. ein Lautgesetz \tt{(s "b" "p")} formulieren,
so hätte dies die Nebenwirkung,
dass auch \tt{bʰ} zu \tt{pʰ} würde,
was im allgemeinen Fall unerwünscht ist.
Wenn man Laute wie \bel{bʰ} oder \bel{gʷ} mit einem einzigen Codepoint
repräsentieren würde,
hätte man dieses Problem natürlich zwar nicht,
jedoch bietet Unicode für solche Vorhaben keine definierten Codepoints.
Dafür müsste man die Private Use Area benutzen,
wodurch die Kodierung jedoch von der Schriftart abhängig wird,
eine unschöne Lösung.
Eine Möglichkeit, dieses Problem zu umgehen, ist,
Polygraphen sowohl in den zu transformierenden Wörtern als auch in den
die Lautgesetze beschreibenden regulären Ausdrücken durch
beliebige unbenutzte Codepoints zu ersetzen
und am Ende die umgekehrte Ersetzung durchzuführen,
so dass die nur intern verwendeten Codepoints nirgendwo graphisch auftauchen.
In der hier entwickelten Version von \tt{lga} wird das Problem
aus Effizienzgründen jedoch noch nicht gelöst.
Stattdessen müssen die Lautgesetze vorläufig
besonders vorsichtig formuliert werden,
der obere Fall also als \tt{(s "b([\textasciicircum{}ʰ])" "p" 1)}.

Bei der Formulierung von Lautgesetzen ist es ferner sinnvoll,
Laute in Lautklassen zusammenzufassen um eine konzise
und leicht abstrahierende Darstellung zu haben.
Der POSIX Standard definiert zwar Zeichenklassen wie \tt{[:digit:]}
für Ziffern oder \tt{[:lower:]} für Kleinbuchstaben,
aber die Bedürfnisse, die man als Linguist hat,
sind damit nicht abgedeckt,
da sich zur Laufzeit Zeichenklassen weder neu definieren noch verändern lassen.
Dieses Problem lässt sich vorläufig lösen,
indem die regulären Ausdrücke,
bevor sie von \tt{string->irregex} in eine interne Form kompiliert werden,
noch verändert werden,
so dass die tatsächlich im Quelltext vorkommenden regulären Ausdrücke
erst zu echten regulären Ausdrücken verarbeitet werden,
die von der \tt{irregex}-Engine verstanden werden.
Konkret funktioniert dies so,
dass in einer assoziativen Liste Lautklassennamen (als Symbole)
mit Strings assoziiert werden
und bei der Vorbearbeitung der regulären Ausdrücke
Strings der Form \tt{<lautklassenname>} durch den mit dem Symbol
\tt{lautklassenname} assoziierten String ersetzt werden.
Die Funktion \tt{(set-class key value)} assoziiert den Namen \tt{key}
mit dem String \tt{value} und \tt{(l key)} (kurz für \emph{lookup})
findet den zu \tt{key} passenden String in der assoziativen Liste.

So kann man bspw. mit \tt{(set-class 'kurz-vok "a|e|i|o|u")} und
\tt{(set-class 'lang-vok "ā|ē|ī|ō|ū")} Kurz- und Langvokale definieren,
mit \tt{(set-class 'vok (s+ (l 'kurz-vok) "|" (l 'lang-vok)))} dann die Menge
aller Vokale
(\tt{s+} ist eine Abkürzung für \tt{string-append} zum Verketten von Strings)
und diese in einem Lautgesetz wie \tt{(s "<vok>" "a")} verwenden.

Die Lautklassen sind jedoch zur Laufzeit nicht vernünftig veränderbar,
da sie zur Zeit der Auswertung von \tt{s} in den regulären Ausdruck
eingefügt werden,
man die Definition aber wohl am ehesten aus dem Lautgesetzbaum heraus
verändern wollen würde
(Lautklassen sollten also dynamisch statt lexikalisch gebunden sein).
Hier wird nur eine neuentwickelte Lautgesetzengine wirklich
Abhilfe schaffen können.

Es gibt jedoch noch ein weiteres Problem.
So kann man nicht ohne weiteres z.B. einen Langvokal durch seinen
entsprechenden Kurzvokal oder einen silbischen durch sein unsilbisches
Pendant ersetzen.
Beispiele für solche Vorgänge gibt es genug und es ist eine Methode notwendig,
mit der man diese Fälle unkompliziert ausdrücken kann.
In einer späteren Version wäre vielleicht eine Ersetzung wie
\tt{(s "<lang-vok>" "<kurz-vok>")} wünschenswert,
in der aktuellen Version können jedoch auch Funktionen,
die auf Submatches angewandt werden, denselben Effekt erzielen.
Als Ersetzungsargumente kann man an \tt{irregex-replace/all}
neben Strings und Zahlen, die das entsprechende Submatch bezeichnen,
auch Funktionen übergeben,
die auf das aktuelle Match angewandt werden und einen String zurückgeben.
Um also z.B. einen Langvokal in ein Kurzvokal zu verwandeln,
braucht es nur eine Funktion,
die auf ein Submatch eine Reihe von regulären Ausdrücken anwendet
und das Ergebnis zurückgibt.
Eine Funktion, die reguläre Ausdrücke auf einen String anwendet,
gibt es ja schon: \tt{apply-rules}.
Diese wird von der Funktion \tt{(match-rulelist rules)} benutzt,
die eine Liste von Ersetzungsfunktionen \tt{rules}
auf ein Submatch anwendet.
Dazu gibt sie eine Funktion zurück,
die einen Submatchindex \tt{i} bindet und eine Funktion zurückgibt,
auf die \tt{irregex-replace/all} angewandt werden kann.
Die Vokalkürzung könnte somit folgendermaßen definiert werden:
\begin{verbatim}
(define kuerzung
  (match-rulelist
    (list (s "ā" "a")
          (s "ē" "e")
          (s "ō" "o")
          (s "ī" "i")
          (s "ū" "u"))))
\end{verbatim}
Ein Lautgesetz, das alle Langvokale kürzt,
könnte dann als \tt{(s "(<lang-vok>)" (kuerzung 1))} formuliert werden.

Die eingangs erwähnten Regellisten zur Umwandlung von einer
graphematisch/morphonologischen in eine phonetische und von
einer phonetischen in eine graphematisch/phonologische Darstellung
sind unter ihrem Sprachnamen ebenfalls in
assoziativen Listen abgelegt.

Hiermit ist die Funktionalität von \tt{lga} weitestgehend erläutert.
Eine exakte Beschreibung kann natürlich nur der Quelltext selber liefern.

\section{Modellierung der Lautentwicklung des Altgriechischen}
\subsection{Vorbemerkungen}
Vor der Beschreibung der Lautentwicklung noch einige Anmerkungen,
die nicht speziell das Griechische betreffen.
Aufgrund von Fehlern in der Implementation von \tt{irregex}
können in regulären Ausdrücken keine höheren Unicodezeichen
in \tt{[]}-Sets benutzt werden.
Da diese aber notwendig sind, um gewisse Zeichen auszuschließen
-- insbesondere \tt{ʰ} zur Unterscheidung von Aspiraten --
müssen diese aus dem ASCII-Vorrat stammen.
Aus dem Grund wird für die Aspiration \tt{!} statt \tt{ʰ} verwendet.
Eigentlich wäre wie schon oben beschrieben für Aspiraten oder sonstige Laute,
die mit mehreren Codepoints ausgedrückt werden,
ein einziger Codepoint
oder eine Möglichkeit, mehrere Codepoints als ein Zeichen zu behandeln,
praktischer und angemessen.
Für die Labiovelare wurden im Gegensatz zu den Velaren Großbuchstaben verwendet,
um dieses Problem zu vermeiden und auch
die silbischen Resonanten \bel{r̥l̥m̥n̥} werden mit den Großbuchstaben \tt{RLMN} ausgedrückt.
Die Halbvokale \bel{i̯u̯} werden mit \tt{yw} und
die Laryngale \bel{h[₁₂₃]} mit \tt{H[123]} bezeichnet.
Akzente werden in der graphematischen Schreibweise
als combining diacritics geschrieben und intern mit \tt{'} und \tt{\textasciitilde}
bezeichnet.

Die Umwandlung dieser graphematischen Schreibungen in die interne
(mehr oder weniger phonetische) Darstellung geschieht in der Liste der
phonologischen Regeln für \tt{uridg} (in \tt{rules.scm}).

Als weitere phonologische Regeln für das Uridg.
sind Laryngalumfärbung (\rek{e} neben \rek{h₃} jedoch noch von \rek{o}
verschieden, was allerdings für das Gr. belanglos ist)
und Stimmhaftwerdung von \rek{s} neben stimmhaften Konsonanten beschrieben.
Weitere phonologische Regeln müssen ggf. ergänzt werden.
Lautgesetze wie \textsc{Stang}s Gesetz,
die vor dem Erreichen des Sprachzustandes, der mit \tt{uridg} bezeichnet wird,
durchlaufen wurden,
wurden nicht berücksichtigt und die Rekonstrukte bzw. Transponate,
die als Eingabe dienen, sollten einen entsprechenden Lautstand vorweisen
(um bei \textsc{Stang}s Gesetz zu beleiben wäre also \rek{di̯ēm} nicht \rek{di̯eu̯m} \tt{uridg}).
Insbesondere enthält \tt{uridg} den \textit{þ}-Laut,
dessen Herkunft und Phonetik hier ebenfalls nicht weiter hinterfragt wird.

Im Folgenden werden die Lautgesetze vom Urindogermanischen ins Griechische
in drei Etappen aufgestellt.
Zunächst wird die Lautentwicklung bis ins Urgriechische modelliert,
dann bis und in mykenischer Zeit und schließlich in die Dialekte
der alphabetischen Zeit.
Die dazu verwendete Literatur ist im wesentlichen \cite{rix},
\cite{bart_frueh},
\cite{bart_hand},
\cite{sihler},
\cite{buck}.

Eine (relative) Chronologie aller Lautgesetze lässt sich nicht immer
mit Sicherheit aufstellen,
da viele Lautgesetze kaum mit anderen Lautgesetzen interagieren
oder mehrere Entwicklungen denkbar sind,
so dass man gerade in der frühesten Zeit,
in der das Griechische noch nicht belegt ist,
in diesen Fällen wenig mehr als nur raten kann.
Hier werden dann die Lautgesetze entweder mit ähnlichen
anderen Gesetzen oder nach Gefühl einsortiert.

Auch ist die Aufstellung hier keinesfalls als endgültig anzusehen.
Besonders Entwicklungen, die in die spätere dialektale Zeit fallen
wie Digammaschwund und Kontraktionen
die zum Teil nach Einsetzen der Überlieferung stattfinden,
sind zu unübersichtlich und unklar,
um sie hier angemessen zu modellieren ohne den Rahmen der Arbeit zu sprengen
(Schwierigkeiten werden im Text genannt).
Aber auch über sicherere Entwicklungen lässt sich im Detail fast immer streiten.
Sowohl das Programm \tt{lga} als auch die Formulierung
der Lautgesetze sind somit als vorläufiges Ergebnis oder Gerüst zu sehen,
auf dem weitere Arbeiten aufbauen sollen.

\subsection{Das Lautsystem des Urindogermanischen}

%TODO: ausführlicher sein

Das hier angenommene Lautsystem des Uridg. besitzt folgende Laute.
Umstrittene Laute stehen in Klammern.
Teilweise ist nur der Phonemstatus umstritten,
aber Phoneme sind für Lautgesetze ohnehin uninteressant.
Andererseits sind die hier aufgeführten Laute natürlich immer noch
eine Abstraktion über die tatsächlichen Laute,
was aber in der Natur der Rekonstruktion liegt
und auch aus pragmatischen Gründen kaum anders gehandhabt werden könnte.

\bigskip
\noindent
Vokale: \rek{a}, \rek{ā}, \rek{e}, \rek{ē}, \rek{o}, \rek{ō},
\rek{i}, (\rek{ī}), \rek{u}, (\rek{ū})

\noindent
Silbische Resonanten: \rek{r̥}, \rek{l̥}, \rek{m̥}, \rek{n̥}

\noindent
Resonanten und Halbvokale: \rek{r}, \rek{l}, \rek{m}, \rek{n}; \rek{i̯}, \rek{u̯}

\noindent
Verschlusslaute:\\
\begin{tabular}{l l l l l}
Labiovelare & \rek{kʷ} & \rek{gʷ} & \rek{gʷʰ} & (\rek{kʷʰ})\\
(Rein)velare & \rek{k} & \rek{g} & \rek{gʰ} & (\rek{kʰ})\\
Palatale & \rek{k̑} & \rek{g̑} & \rek{g̑ʰ} & (\rek{k̑ʰ})\\
Dentale & \rek{t} & \rek{d} & \rek{dʰ} & (\rek{tʰ})\\
Labiale & \rek{p} & \rek{b} & \rek{bʰ} & (\rek{pʰ})\\
\end{tabular}

\noindent
Frikative und sonstige: \rek{s}, \rek{z}, \rek{h₁}, \rek{h₂}, \rek{h₃}, (\rek{þ})

\bigskip
\rek{z} ist Allophon von \rek{s} vor stimmhaften Konsonanten.
Beide werden aber der Einfachheit halber in \tt{lga} als \rek{s} repräsentiert.

Die genaue Artikulation der Laryngale \rek{h₁}, \rek{h₂}, \rek{h₃}
ist nicht klar. Meist werden Frikative vermutet,
neben denen ggf. Sprossvokale eingefügt werden.

Der Laut \rek{þ} ist umstritten,
sowohl seine Existenz als auch seine Artikulation
(vielleicht dentale Affrikata?).

\subsection{Urindogermanisch bis Urgriechisch}

Die Lautveränderung,
die ins Urgr. führen, sind grob gesprochen
die Kentumvertretung der Tektale,
die Stimmloswerdung der Mediae aspiratae,
die Entwicklung von \rek{þ},
jegliche Laryngalentwicklungen,
die Entwicklung der silbischen Resonanten (teilweise aber auch erst dialektal),
einige Okklusivassimilationen,
die Entwicklung von \rek{s}
und die Anfänge diverser Palatalisierungen.

\mypara{Kentum}
Das Griechische ist eine Kentumsprache,
es fallen also die uridg. Palatale mit den Velaren zusammen (Rix §92-94).
Da das Griechische keinerlei Anhaltspunkte dafür bietet,
dass es je einen Unterschied zwischen diesen Lauten gab,
wird der Zusammenfall früh datiert.

\lex{\rek{k̑}, \rek{g̑}, \rek{g̑ʰ}, (\rek{k̑ʰ})}{\rek{k}, \rek{g}, \rek{gʰ}, (\rek{kʰ})}

uridg. \rek{k̑}: \rek{k̑m̥tóm}: gr. ἑκατόν vs. ai. \bel{śatáṃ} `hundert'

uridg. \rek{k}: \rek{leu̯kós}: gr. λευκός vs. ai. \bel{rocás} `leuchtend'

uridg. \rek{g̑}: \rek{h₂ég̑onti}: att. ἄγουσι vs. av. \bel{azənti} `sie führen'

uridg. \rek{g}: \rek{gerh₂nós}: gr. γερανός vs. lit. \bel{gervė} `Kranich'

uridg. \rek{g̑ʰ}: \rek{u̯eg̑ʰ-/u̯ogʰ-}: att. ὄχος `Wagen' vs. av. \bel{vazāmi} `ich fahre'

uridg. \rek{gʰ}: \rek{h₃migʰ-/h₃moi̯gʰ-}: gr. ὀμίχλη `Nebel' vs. ai. \bel{meghás} `Wolke'

\mypara{Thorn}
\rek{þ} (genauer Lautwert unklar) tritt nach Velaren und Labiovelaren auf.
Nach Tenues entsprich \rek{t}, nach Media aspirata \rek{dʰ} (Rix §81).

\lex{\rek{þ}}{\rek{t}}{\{Tenuis\}\_}

uridg. \rek{k̑þ}: \rek{tek̑þō(n)}: gr. τέκτων vs. ai. \bel{tákṣā} `Zimmermann'

\lex{\rek{þ}}{\rek{dʰ}}{\{Media asp.\}\_}

uridg. \rek{gʷʰþ}: \rek{ń̥gʷʰþitom}: gr. ἄφθιτον vs. ai. \bel{ákṣitam} `unvergänglich'

\mypara{*MA > TA}
Mediae aspiratae werden zu Tenues aspiratae (Rix §94).
Aufgrund der Nähe zum Makedonischen und Phrygischen,
in denen der Reflex vermutlich Media ist,
ist dieser Wandel vielleicht später anzusetzen.
Spätestens im Myk. ist die Stimmlosigkeit belegt.

\lex{\rek{bʰ}, \rek{dʰ}, \rek{gʰ}, \rek{gʷʰ}}{\rek{pʰ}, \rek{tʰ}, \rek{kʰ}, \rek{kʰʷ}}

uridg. \rek{bʰ}: \rek{bʰéronti}: dor. φέροντι vs. ai. \bel{bháranti} `sie tragen'

uridg. \rek{dʰ}: zu \rek{dʰeh₁-}: att. τίθημι vs. ai. \bel{dadhā́mi} `ich setze/stelle/lege'

uridg. \rek{g̑ʰ}: zu \rek{stei̯g̑ʰ-}: gr. στείχω vs. got. \bel{steigan} `steigen'

uridg. \rek{gʷʰ}: zu \rek{gʷʰen-}: gr. φόνος `Mord' vs. ai. \bel{gʰnánti} `sie erschlagen' (zu \rek{kʷʰ} > \rek{pʰ} s. TODO)

\mypara{\rek{(H)i̯-}}
Da es im Gr. scheinbar zwei Reflexe von \rek{i̯-} gibt,
versucht man mit der Laryngaltheorie die beiden Reflexe als \rek{Hi̯-}
und \rek{i̯-} zu unterscheiden.
Welcher Anlaut zu urgr. \rek{dz-} verschärft wurde,
ist unklar -- nach Rix §68,80e ist es \rek{Hi̯-} --
und nach dem Datenmaterial kann für beides argumentiert werden,
wenn man aber ὑγιής `gesund' aus \rek{h₂i̯u-gʷih₃-ḗs} `langes Leben habend'
herleiten will (\rek{h₁su-} würde εὐ- ergeben),
darf man \rek{Hi̯-} > \rek{i̯-} vermuten.
Auch lautlich plausibel ist eine
frühe Verschärfung von uridg. \rek{i̯-} zu urgr. \rek{dz-}
(vielleicht über Zusammenfall mit ererbtem \rek{di̯-},
das ebenfalls zu \rek{dz-} wird),
danach Wegfall aller Laryngale (also keine Vokalisierung wie vor \rek{u̯}),
wodurch \rek{Hi̯-} als \rek{i̯-} stehenbleibt,
wie es im Frühmyk. belegt ist.

\lex{\rek{i̯-}}{\rek{di̯-}}

uridg. \rek{i̯-}: \rek{i̯ugóm}: gr. ζυγόν vs. ai. \bel{yugám} `Joch'

\lex{\rek{Hi̯-}}{\rek{i̯-}}

uridg. \rek{h₂i̯-}: s.o. als \rek{h₂i̯u-} `langes Leben (→ lange Zeit → Zeit)'
wegen Vollstufe \rek{h₃ei̯u-} in z.B.
hom. αἰεί `immer' < \rek{h₂ei̯u̯ési} `zu allen Zeiten' zu rekonstruieren.

\mypara{Laryngale}
\label{laryngale}
Die Entwicklung der Laryngale folgt Rix §79-85.
Teilweise sind die Lautgesetze weniger explizit als bei Rix formuliert,
um eine lautlich plausiblere Entwicklung zu modellieren.
So wird z.B. intervokalischer Laryngal nach \rek{i} oder \rek{u}
nicht zum Gleitlaut sondern dieser entsteht erst als Hiattilger
nach dem Laryngalschwund.
Der Schwund der Laryngale dürfte auch eher früh passiert sein.
Der Wandel, der hier als \rek{H} > \rek{V} geschrieben ist,
ist als \rek{h₁}, \rek{h₂}, \rek{h₃} > \rek{e}, \rek{a}, \rek{o} zu verstehen.

\lex{\rek{H}}{\rek{V}}{C\_\#}

\lex{\rek{(i, u)H}}{\rek{(i, u)V}}{CC\_\#}

\lex{\rek{(i, u)H}}{\rek{(i̯, u̯)V}}{C\_\#}

\lex{\rek{VH}}{\rek{V̄}}{\_(C, R̥, \#)}

\lex{\rek{H}}{\rek{V}}{(\#, C)\_C}

\lex{\rek{n̥h₂}}{\rek{na}}{\_\#}

\lex{\rek{HR̥}}{\rek{VR}}{C\_}

\lex{\rek{Ŕ̥H}}{RV̄́}{\_C}

\lex{\rek{R̥H}}{V́RV}{\_C}

\lex{\rek{H(u, R̥)}}{\rek{V(u̯, R)}}{\#\_}

\lex{\rek{H}}{\rek{Ø}}

\mypara{\rek{u̯eu̯} > \bel{u̯ei̯}}
\rek{u̯eu̯} wird zu \rek{u̯ei̯} dissimiliert,
wie εἶπον < \rek{e-u̯e-u̯kʷ-om} (vgl. ai. \bel{avocam}) lehrt.
Dies muss ferner vor \ref{labvel-vel} passieren,
wie εἶπον ebenfalls zeigt.

\lex{\rek{u̯}}{\rek{i̯}}{u̯e\_}

\mypara{Labiovelar > Velar}
\label{labvel-vel}
Neben \rek{ū̆}, \rek{u̯} und vor \rek{i̯}
fallen die Labiovelare mit den Velaren zusammen (Rix §97).

\lex{\rek{kʷ, gʷ, kʷʰ}}{\rek{k, g, kʰ}}{(u, ū, u̯)\_; \_(u, ū, u̯, i̯)}

\mypara{Silbische Resonanten}
Die Entwicklung der silbischen Resonanten folgt Rix §75 und §76.
Für das Urgr. wird ein Sprossvokal \rek{ə} angesetzt,
welcher sich dialektal zu \bel{a} oder \bel{o} entwickelt.

Genauere Datierung unklar.

\lex{\rek{N̥}}{\rek{aN}}{(V, V̯)}

\lex{\rek{N̥}}{\rek{ə}}

\lex{\rek{R̥}}{\rek{əR}}{(\#, V, V̯)}

\lex{\rek{R̥}}{\rek{Rə}}

\mypara{Nasal vor Okklusiv}
Vor Okklusiven wird der homorgane Nasal realisiert (Rix §78).
Gegen Rix jedoch nicht \rek{ms} > \rek{ns} wegen
ἔνειμα < \rek{enemsa}.
Hier kann \bel{m} zwar analog wieder eingeführt worden sein,
aber ohne ein Beispiel, das bei Rix fehlt, ist die Regel zunächst unnötig.
Die Nasalassimilation passiert auch später noch und
dürfte tatsächlich wohl über lange Zeit eine synchrone phonologische Regel sein.

\lex{\rek{N}}{\rek{m}}{\{Labial\}}

\lex{\rek{N}}{\rek{n}}{\{Dental, Velar, Labiovelar\}}

\mypara{\rek{mi̯} > \rek{ni̯}}
Rix §77.
Datierung unklar.
Vielleicht nach \rek{m̥} > \rek{am}, aber \rek{m̥} > \rek{n̥} \textbackslash\ \rek{i̯}
ist ebenfalls als zusätzliche Regel denkbar.

\lex{\rek{m}}{\rek{n}}{i̯}

\mypara{Benachbarte Vokale nach Laryngalschwund (1. Kontraktion)}
Kontraktion von \rek{e}, \rek{a}, \rek{o},
zwischen denen Laryngal geschwunden ist (Rix §81).
Hier entsteht zuerst der Intonationsunterschied zwischen Akut und Zirkumflex.
Trägt der erste Vokal den Akzent, ist das Kontraktionsprodukt zirkumflektiert.
Trägt ihn der zweite, ist es akutiert.
Datierung unklar.

\lex{\rek{VV}}{\rek{V̄}} (\rek{V} = \rek{a}, \rek{e}, \rek{o})

Nach Vokal werden \rek{i}, \rek{u} zu Halbvokalen (es entstehen also Diphthonge).
Im umgekehrten Fall entsteht ein Gleitlaut zwischen den beiden Lauten.

\lex{\rek{i, u}}{\rek{i̯, u̯}}{V\_}

\lex{\rek{i, u}}{\rek{ii̯, uu̯}}{\_V}

\mypara{Okklusivassimilation}
Okklusive werden an folgende Okklusive in
Aspiration und Stimmhafigkeit assimiliert (Rix §106a).

Hier dürfte es sich wohl eigentlich um eine phonologische Regel handeln.

\lex{\rek{\{Okklu.\}}}{\rek{\{Media\}}}{\{Media\}}

\lex{\rek{\{Okklu.\}}}{\rek{\{Tenuis\}}}{\{Tenuis\}}

\lex{\rek{\{Okklu.\}}}{\rek{\{Aspirata\}}}{\{Aspirata\}}

\mypara{\rek{Ku̯} > \rek{(Kʷ)Kʷ}}
\label{kw}
Velar + \rek{u̯} wird zum Labiovelar, zwischen Vokalen als Geminate (Rix §104),
um das Silbengewicht zu erhalten (vielleicht nur nach Kurzvokal?).

\lex{\rek{Ku̯}}{\rek{KʷKʷ}}{V\_V}

\lex{\rek{Ku̯}}{\rek{Kʷ}}

\mypara{Beseitigung von Geminaten}
\label{gem}
Das Gr. muss eine Zeit lang Geminaten vermieden haben.
Entweder generell wie das Uridg.
-- dann müsste diese Regel vor \ref{kw} kommen
und dort die Gemination nur nach Kurzvokal passieren --
oder nur im Silbenanlaut (also nur nicht nach Kurzvokal)
wie αἰπόλος < \rek{ai̯kʷolos} < \rek{ai̯g-kʷolos} zeigt.
Weitere Beispiele könnten die Verhältnisse klarer machen.

Datierung unklar.
Vermutlich nach \ref{kw} und sicher vor \ref{pal}.

\lex{\rek{C:}}{\rek{C}}{.\_}

\mypara{Assimilation über \bel{s}}
\label{s-assim}
Eine Tenuis hinter \rek{s} wird an eine vor \rek{s} stehende
Media oder Tenuis aspirata in ihrer Artikulationsart angeglichen (Rix §105).

Auf jeden Fall vor \ref{okklu-s} wegen πάσχω < \rek{pʰatʰskō},
λέσχη < \rek{lekʰskā}, μίσγω < \rek{migskō}.

\lex{\rek{\{Tenuis\}}}{\rek{\{Media\}}}{\{Media\}s\_}

\lex{\rek{\{Tenuis\}}}{\rek{\{Aspirata\}}}{\{Aspirata\}s\_}

\mypara{CsC > sC}
Steht \rek{s} zwischen zwei gleichen Konsonanten, schwindet der erste (Rix §104).
Welche Konsonanten das im einzelnen sind, scheint nicht ganz klar zu sein.

\lex{\rek{C₁}}{\rek{Ø}}{\_C₂} (C₁ = C₂)

\mypara{Cs}
\label{okklu-s}
Vor \bel{s} werden Okklusive zu Tenues (Rix §105).

Auf jeden Fall nach \ref{s-assim} wegen μίσγω.

\lex{\rek{\{Okklu.\}}}{\rek{\{Tenuis\}}}{s}

\subsection*{urgr1}

\mypara{\rek{s} > \rek{h}}
\label{sh}
\rek{s} wird nach Vokal oder im Anlaut und vor Vokal, Resonant oder \rek{u̯},
nach unbetonter Silbe zwischen Resonant oder \rek{u̯} und Vokal
und zwischen Nasal und Vokal zu \rek{h} verhaucht (Rix §86,89).

Der Wandel ist vor \ref{osthoff} zu datieren.

\lex{\rek{s}}{\rek{h}}{(\#, V)\_(V, R, u̯)}

\lex{\rek{s}}{\rek{h}}{(R, u̯)\_V} (nach unbetonter Silbe)

\lex{\rek{s}}{\rek{h}}{N\_V}

In einigen Fällen ist \rek{s} auf unerklärte Weise erhalten,
teilweise beim selben Wort:
σῦς neben ὗς `Schwein', σμικρός neben μικρός `klein'.

\mypara{\rek{s} neben \rek{i̯}}
Nach Rix §89g ist \rek{s} neben \rek{i̯} über \rek{h} assimilert worden.
Die Beispiele belegen jedoch nur eine Assimilation bei \rek{si̯}
und der Weg über \rek{h} scheint nur aus systematischen Gründen gewählt zu sein.

\rek{si̯} wird intervokalisch zu \rek{i̯i̯}
und anlautend möglicherweise zu \rek{i̯}
(nur ein Beispiel, das auch noch uneindeutig ist).

\rek{i̯s} kommt intervokalisch im I.Pl. \rek{-oi̯si} vor,
der myk. <-o-i> geschrieben wird.
Die Lautung ist unklar -- vielleicht \rek{-oi̯i̯i} oder \rek{-oi̯hi} --
da später das \rek{s} analogisch restituiert wurde.

\lex{\rek{si̯}}{\rek{i̯i̯}}{V\_V}

\lex{\rek{si̯}}{\rek{i̯}}{\#\_V}

\lex{\rek{i̯s}}{\rek{i̯i̯}}{V\_V} (?)

\mypara{\textsc{Osthoff}}
\label{osthoff}
Die klassische Formulierung von \textsc{Osthoff}s Gesetz ist
Kürzung von Langvokalen vor Resonant oder Halbvokal und Konsonant
(Rix §58, 64).
\textsc{Osthoff}s Gesetz ist jedoch umstritten und es gibt mindestens
ein sicheres Gegenbeispiel,
das vermutlich auf eine andere Syllabifizierung deutet.
Dann wäre die Vormulierung aber einfach die,
dass Langvokal vor tautosyllabischem Resonanten oder Halbvokal gekürzt wird
(im Auslaut ist die antevokalische Sandhivariante verallgemeinert,
so dass Langvokale erhalten bleiben).

In att. G.Sg. μηνός < \rek{mēnsós} ist der Langvokal geblieben
(Ersatzdehnung hätte geschlossenes \bel{ē} <ει> ergeben).
Die beste Lösung ist die s-Verhauchung (\ref{sh})
vor \textsc{Osthoff} zu datieren
und eine Syllabifizierung \rek{mē.nhós} anzunehmen,
bei der das Gesetz nicht gilt.
In welchen Fällen genau eine andere Syllabifizierung angenommen werden sollte,
werde ich hier nicht detailliert behandeln.
Simkin nimmt neben \rek{nh} auch noch \rek{nm} und \rek{u̯i̯} an,
%TODO: literatur
aber ich werde mich vorerst auf \rek{nh} beschränken.

Das Lautgesetz muss vor \ref{konsausl} datiert werden
wie z.B. hom. ἔσταν < \rek{estānt} < \rek{e-steh₂-n̥t} zeigt.

\lex{\rek{V̄}}{\rek{V}}{\_(R, V̯).} (außer im Auslaut)

\mypara{Konsonanten im Auslaut}
\label{konsausl}
Wortauslautende Okklusive gehen verloren (Rix §100).

Dies muss nach der Vokalisierung der silbischen Liquiden passiert sein,
da ὑπόδρα < \rek{upodrak} < \rek{upodr̥k̑} den auslautenden Konsonanten
voraussetzt.
Außerdem nach \textsc{Osthoff} (\ref{osthoff}) wie dort gezeigt.

\lex{\rek{\{Okklu.\}+}}{\rek{Ø}}{\_\#}

Auslautendes \rek{m} erscheint als \bel{n} (Rix §77).

Wann dies passierte, ist vollkommen unklar.
Es wird hier nur zusammen mit dem obigen Gesetz gruppiert,
da beide den Auslaut betreffen.

\lex{\rek{m}}{\rek{n}}{\_\#}

\mypara{Palatalisierung}
\label{pal}
Die Anfänge der Palatalisierungen liegen wohl in urgr. Zeit
und die genaue Entwicklung in den Dialekten ist kompliziert.
Zu diesem Thema siehe Allen 1958, Rix §102,103 und Bartoněk S. 140ff.

Linear B hat zwei Reihen für die Schreibung der späteren Zischlaute,
deren Konsonanten mit <z> und <s> transliteriert werden.
Welche Laute mit diesen Zeichen bezeichnet wurden, ist unklar.
Oft wird <s> als einfacher Sibilant und <z> als Affrikate verstanden,
jedoch ist es sehr gut möglich,
dass es sich tatsächlich um einen Unterschied der Palatalität handelt
und beide Zeichenreihen sowohl Sibilanten als auch Affrikaten bezeichnen können.
die Z-Reihe würde dann einen palataleren Laut als die S-Reihe bezeichnen.
Mit dieser Annahme lassen sich die mykenischen Schreibungen
und die Lautentwicklung fast zufriedenstellend erklären.
Noch besser ist vielleicht die Annahme,
dass es sich bei der Z-Reihe um eine Art präpalatalen Plosiv
und bei der S-Reihe um einen Sibilanten oder eine Affrikate handelt.

Dentale Tenuis (aspirata) + \rek{i̯} im Anlaut ergibt in allen Dialekten \bel{s},
das wohl als urgr. gelten darf.
Im Inlaut wird hier \rek{ts} angesetzt, das mit altem \rek{ts} zusammenfällt
und sich dann dialektal unterschiedlich entwickelt.
In Linear B wird für diesen Laut die S-Reihe benutzt,
weswegen eine palatale Qualität sehr unwahrscheinlich ist.

\lex{\rek{(t, tʰ)i̯}}{\rek{s}}{\#\_}

\lex{\rek{(t, tʰ)i̯}}{\rek{ts}}{\_}

%TODO: klären
Dieselbe Lautkombination an morphologisch transparenter Stelle
ergibt einen anderen Laut,
der in alphabetischer Zeit mit dem Resultat der Palatalisierung der
Velare zusammenfällt,
in Linear B aber mit der S-Reihe geschrieben wird.
Vielleicht wurde an dieser Stelle das \rek{i̯} restituiert
und die Gruppe \rek{tsi̯} nachmyk. weiterpalatalisiert,
so dass sie mit Resultat der palatalisierten Velare zusammenfiel.
Möglicherweise handelt es sich hier aber auch um denselben Laut,
den ich unten für die Substratwörter ansetze,
da Schreibung und Entwicklung mit diesem völlig übereinstimmt.

Stimmhafte Dentale und Velare fallen vor \rek{i̯} in einem Laut zusammen,
der in Linear B mit der Z-Reihe und im Alphabetgriechischen mit ζ
geschrieben wird.
Der Laut dürfte entweder eine stimmhafte Affrikate oder ein
stimmhafter präpalataler Plosiv gewesen sein
(dafür schreibe ich hier vereinfachend <ǰ>).
Dass hier Dentale und Velare zusammenfallen ist eigenartig.

\lex{\rek{(d, g)i̯}}{\rek{ǰ}}

Velare Tenuis (aspirata) + \rek{i̯} scheint eine spätere Palatalisierung
als die der Dentale zu sein.
Das Ergebnis wird in Linear B ebenfalls mit der Z-Reihe geschrieben,
was eine stimmlose Version des obigen Lautes nahelegt,
den ich mit <č> bezeichne.

\lex{\rek{(k, kʰ)i̯}}{\rek{č}}

Ein anderer Laut, der in Substratwörtern vorkommt,
fällt im Alphabetgriechischen völlig mit diesem \rek{č} zusammen,
wird aber im Linear B mit der S-Reihe geschrieben.
Da zwischen \rek{ts} und dem zuerst vorgeschlagenen \rek{tš} als Ergebnis
der Velarpalatalisation jedoch wenig Spielraum für eine weitere Affrikate bleibt,
die zwar später mit \rek{tš} zusammenfiele, aber dental genug wäre,
um in Linear B mit der S-Reihe geschrieben zu werden,
ist die Bestimmung der Laute der Z-Reihe als irgendwie palatale Plosive
vielleicht die beste Lösung.
Dann könnte man den Substratlaut als \rek{tš} verstehen
und eine spätere Entwicklung \rek{ǰ}, \rek{č} > \rek{dž}, \rek{tš} annehmen.
Heteromorphemisches \rek{ti̯} könnte sich ebenfalls zu \rek{tš} entwickelt
und somit die diese Möglichkeit der Einordnung des Substratlautes in das
Lautsystem überhaupt erst ermöglicht haben.

Labial + \rek{i̯} ergibt \bel{pt} in allen Dialekten,
muss jedoch eine Vorstufe urgr. \rek{pč}
(ohne eine genaue phonetische Interpretation nahelegen zu wollen) gehabt haben,
da das Myk. ein eigenes Zeichen für diesen Laut kennt,
der erst später auch für ererbtes \bel{pt} verwendet wird.
%TODO: artikel dazu

\lex{\rek{\{Labial\}i̯}}{\rek{pč}}

Die Resonanten \rek{l}, \rek{r}, \rek{n}
(\rek{m} in dieser Position schon mit \rek{n} zusammengefallen)
werden vor \rek{i̯} zu palatalisierten Geminaten.
Ob hier Anlaut oder Silbengewicht eine Rolle spielen, ist noch zu überprüfen.

\lex{\rek{(l, r, n)i̯}}{\rek{(ĺĺ, ŕŕ, ńń)}}

\mypara{\rek{dl} > \rek{ll}}
Datierung unklar.

\lex{\rek{d}}{\rek{l}}{\_l}

\mypara{Metathese von \rek{t} neben Okklusiven}
Nach Rix (§106) werden \rek{tk} und \rek{tp} metathetiert.
Für \rek{tp} > \rek{pt} gibt er als Beispiel \rek{kʷid-pe} > τίπτε,
das aber wohl besser als \rek{kʷid-kʷe} > \rek{kʷikʷte} > τίπτε zu erklären ist.
Will man den rätselhaften Anlaut \bel{pt-} als Variante von \bel{p-}
als Sandhi aus \rek{-t\#\#p-} erklären,
so wäre auch eine Metathese neben Labialen anzunehmen.
So kann man die allgemeine Regel aufstellen, dass \bel{t} + Okklusiv
metathetiert wird.

\lex{\rek{t(k, kʷ, p)}}{\rek{(k, kʷ, p)t}}

Datierung unklar.

\subsection*{urgr}

\subsection{Urgriechisch bis mykenische Zeit}

In diesem Abschnitt wird die Lautentwicklung bis zur Zeit,
in der die mykenische Überlieferung endet, modelliert.
Am Anfang dieser Zeit steht die erste dialektale Aufspaltung.
Da außer dem recht einheitlichen myk. Dialekt keine weiteren aus dieser
Zeit überliefert sind,
wird der Lautstand des Myk.
-- sofern es keinen Grund gibt, etwas anderes anzunehmen --
als repräsentativ für das gesamte gr. Sprachgebiet gesehen.

Die dialektale Gliederung ist im wesentlichen aus \cite{bart_frueh}
Tab. B genommen und sieht in Scheme wie folgt aus:
\begin{verbatim}
(urgr1 (urgr (nwgr (dor (nwdor phok lokr aitol)
                        (sardor kor meg oarg)
                        (sdor el lak inseldor warg kret))
                   (aiol boiot (thess wthess othess) lesb))
             (sogr (ach myk ark kypr pamph)
                   (ion-att (ostion hom) inselion euboi att))))
\end{verbatim}

\mypara{\rek{ti} > \bel{si}}
Dieser Lautwandel wird als grundlegende Unterscheidung zwischen Südost-
und Nordwestgriechisch gesehen,
ist aber weniger Lautgesetz als einem lieb wäre (Rix §101).
Auch westgr. Dialekte haben \bel{si} in \emph{ti}-Abstrakta
und das Myk. geht in \bel{ko-ri-si-o} weiter als andere Dialekte,
die alle Κορίνθιοι zeigen (\cite[144]{bart_hand}).
In dieser Darstellung wird der Lautwandel wegen z.B. ἔστι, κτίσις
nach \rek{s} und \rek{k} ausgesetzt.
Eine genauere Lautumgebung -- falls man sie überhaupt angeben kann --
ist unsicher.
Vermutlich ist der Wandel schlicht nicht lautgesetzlich
sondern als eine Verallgemeinerung von Sandhivarianten zu sehen
(und somit Analogie).

\lex{\rek{t}}{\rek{s}}{\_i} außer nach \rek{s}, \rek{k}

\mypara{\rek{ə} > \rek{a}/\rek{o}}

Der Sprossvokal, der neben/aus Liquida bzw. Nasalis sonans entstanden ist
wird dialektal zu \rek{a} bzw. \rek{o}.
Da die genaue Verteilung schwierig ist,
nehme ich hier als ungefähre Annäherung nur \rek{a} an,
wissend, dass dies eigentlich nicht stimmt.

\lex{\rek{ə}}{\rek{a}}

\mypara{\rek{pč}}{pt}
Die Linear B Schreibung für den späteren Laut \bel{pt} schwankt.
Vermutlich fällt die Entwicklung von \rek{pč} in diese Zeit.

\lex{\rek{pč}}{\rek{pt}}

\mypara{\rek{i̯} > \rek{h}}

Anlautendes und intervokalisches \rek{i̯} wird im Myk. teilweise mit der J-Reihe,
teilweise (wenn möglich) mit einem H-Zeichen,
und teilweise gar nicht geschrieben.
Daraus ist zu schließen,
dass die Verhauchung im Laufe der Mykenischen Überlieferung passierte.

\lex{\rek{i̯}}{\rek{h}}{(\#, V)\_V}

\subsection*{myk}

\subsection{Nachmykenische/alphabetische Zeit}
In dieser Zeit bilden sich die meisten Unterschiede der Dialekte heraus.
Durch die Fülle an Dialekten und der dazu im Vergleich eher
spärlichen Belegsituation sind die genauen Verhältnisse
in vielen Fällen unklar und die hier aufgestellten Regeln
entsprechend mit mehr oder weniger großer Unsicherheit behaftet.
In fast jedem Fall müsste man sich eingehend mit den Dialektinschriften
auseinandersetzen, um das Phänomen so gut es geht beschreiben zu können.
In vielen Fällen sind die Lautwandel zwar recht klar,
aber wie sie dialektal verstreut sind und wann sie zu datieren sind,
ist oft sehr unsicher.

Ein Hauptproblem bei der Überlieferung ist,
dass viele Dialekte erst in späterer Zeit überliefert sind,
in denen man schon von Beeinflussung durch die ion-att. Koiné rechnen muss.
Daher weiß man oft nicht,
ob ein Merkmal in einem Dialekt ererbt oder entlehnt ist.

Letztendlich ist es nicht mein Ziel,
die Lautung jedes belegten Dialektes so gut es geht aus dem Urgr. herzuleiten,
da diese Aufgabe wegen des zeitlich und räumlich gesäten Variantenreichtums
in dieser Arbeit nicht zu schaffen ist.
Stattdessen werde ich versuchen den Lautstand der etwas größeren Dialektgebiete
zur Zeit des Einsetzens ihrer Überlieferung zu generieren
(auch wenn sich dieser Zeitpunkt von Dialekt zu Dialekt recht stark unterscheiden kann).

Die eingeschobenen ungefähren Jahreszahlen habe ich anhand einiger Termini
post quos für die Dialektaufspaltung,
die ich aus den von den Lautwandeln betroffenen Dialekten geschlossen habe,
aus dem Dialektstammbaum von \textcite{bart_frueh} eingefügt.
Einige Zeiträume scheinen merkwürdig ereignislos, andere wiederum sehr bewegt.
Auch hier ist wohl also noch keine Wahrheit gefunden und
an Datierung sowohl der dialektalen Aufspaltung als auch der Lautgesetze
sollte noch gearbeitet werden.

\mypara{\rek{ǰ} > \rek{ds}}
\rek{ǰ} wird zur dentalen Affrikate, die,
da wir bei \bel{s} in den Regeln stimmhaftigkeit ignorieren,
mit \rek{ds} bezeichnet wird.
Phonetisch ist von [\bel{dz}] auszugehen.
Die genaue spätere Lautung ist umstritten.
Geschrieben wird der Laut mit ζ,
das aber auf jeden Fall auch [\bel{zd}] bezeichnen konnte,
was Wörter zeigen, bei denen \rek{sd} ursprünglich ist
(z.B. ὄζος = got. \bel{asts} `Ast').
Eine Metathese \rek{zd} > \rek{dz} ist unwahrscheinlich
und einer Erhaltung von \rek{dz} < \rek{(d, g)i̯} widerspricht
ἔρδω `machen' < \rek{u̯erzdō} < \rek{u̯erdzō} < \rek{u̯erǰō} < \rek{u̯erg̑i̯oh₂}.
\rek{dz} wird also nur als Zwischenstufe zwischen \rek{ǰ} und \rek{zd}
angenommen.

\lex{\rek{ǰ}}{\rek{ds}}
\label{ds}

\mypara{el. \rek{ē} > \rek{ā}}

Im Elischen ist \rek{ē} als \bel{ā} wiedergegeben.
\textcite{bart_langvok} setzt diesen Wandel um etwa ??? an.

\lex{\rek{ē}}{\bel{ā}} el.

\mypara{Labiovelare}

Die genaue Entwicklung der Labiovelare ist in einigen Dialekten etwas unklar.
Gemeingriechisch ist der Wandel \{Labiovelar\} > \{Labial\}.
Darüber hinaus haben die meisten Dialekte einen Wandel
\{Labiovelar\} > \{Dental\} in bestimmten Umgebungen.

Dies hier ist eine vereinfachte Darstellung.
Das Äolische hat z.B. entgegen der Regel -τε `und' < \rek{-kʷe}.
(Rix §96-99)

%TODO: literatur, bessere formulierung

\lex{\rek{kʷ}}{\bel{t}}{\_i} (nicht ach, aiol)

\lex{\rek{\{Labiovelar\}}}{\bel{\{Dental\}}}{\_e} (nicht ach, aiol)

\lex{\rek{\{Labiovelar\}}}{\bel{\{Labial\}}}

Nach dem Wandel assimilieren sich die Nasale wieder an den folgenden Laut.
Dies ist wohl auch als phonologische Regel anzusetzen.

\mypara{\rek{b}, \rek{g} > m, ŋ}

\rek{b} und \rek{g} assimilieren sich an ein folgendes \bel{n}
und werden zu Nasalen (Rix §105).

Dies muss, wenn man keinen labiovelaren Nasal annehmen will,
nach der Beseitigung der Labiovelare passiert sein.
(σεμνός < \rek{sebnós} < \rek{ti̯egʷnós})

\lex{\rek{b}, \rek{g}}{\bel{m}, \bel{ŋ}}{\_n}

\mypara{\rek{Tu̯}}
\rek{tu̯} verhält sich im Anlaut wie \rek{ti̯}, sonst wie \rek{ki̯} (Rix §104).
Eine Hesychglosse (TODO: quelle) erweist unverändertes \rek{tu̯e} fürs Kretische.
Dass \rek{tu̯} nicht komplett mit einer der beiden Lautgruppen zusammenfällt,
ist etwas überraschend.

Nach Rix bleibt \rek{tʰu̯} zunächst erhalten.
Da Beispiele fehlen,
könnte man aber auch (zumindest im Inlaut) als Ergebnis \bel{stʰ} annehmen,
wodurch man die Endung der 2.Pl. -σθε mit ai. \bel{-dhvam} verbinden kann.

\lex{\rek{tu̯}}{\bel{s}}{\#\_} (nicht kret.)

\lex{\rek{tu̯}}{\rek{tš}}

\lex{\rek{tʰu̯}}{\bel{sth}}

\mypara{\rek{ds} > \bel{sd}}
\label{sd}
Wie schon bei \ref{ds} erklärt, wird \rek{ds} > \rek{sd}.

Dies muss wegen ἔρδω vor \ref{CsC} passieren.

\lex{\rek{ds}}{\bel{sd}}

\mypara{\rek{(N, T)sC}}
\label{NTsC}
Nasal und Dental schwinden vor \rek{sC}.
Dies muss vor \ref{CsC} passieren,
wie att. ἴσος < \rek{u̯itsu̯os} und δεσπότης < \rek{dems-pot-} zeigen.

\lex{\rek{(N, T)}}{Ø}{\_sC}

\mypara{\rek{CsC}}
\label{CsC}
Zwischen Konsonanten wird \rek{s} in der Regel zu \rek{h} verhaucht und
schwindet dann (Rix §87c, Lejeune §132-133).
Die genaue Entwicklung ist nicht völlig klar,
aber zwischen Resonant oder Halbvokal und stimmlosem Okklusiv
scheint \rek{s} erhalten zu sein.
Bestimmte Lautgruppen sind dann jedoch dialektal teilweise vereinfacht worden.
Eine Tenuis vor dem \rek{s} wurde aspiriert,
woran sich eine folgende Tenuis assimilierte.
Eine folgende Media hatte Assimilation der Aspirata an die Media als Folge.

Dies muß -- wie schon gesagt -- nach \ref{sd} und \ref{NTsC} passieren.
Im Myk. ist die Lautverbindung noch intakt.

\lex{\rek{s}}{\rek{h}}{C\_C} (außer in (R, V̯)\_(T, Tʰ))

\lex{\rek{Th}}{\bel{Tʰ}}

\lex{\rek{T}}{\bel{Tʰ}}{Tʰ\_}

\lex{\rek{Tʰ}}{\bel{D}}{\_D}

\mypara{\rek{u̯i̯} > \rek{ẅẅ}}
\rek{u̯i̯} wird zu einer palatalen Geminate.
Damit gliedert sie sich in die Reihe der palatalen \rek{ĺĺ}, \rek{ŕŕ}, \rek{ńń}
ein, mit denen sie sich weitgehend parallel entwickelt.
(Rix §73).

\lex{\rek{u̯i̯}}{\rek{ẅẅ}}

\mypara{Palatale Geminaten}
\label{palgem}
Nach \bel{a} oder \bel{o} spalten sich \rek{ŕŕ}, \rek{ńń}, \rek{ẅẅ}
in \bel{i̯} und ihre ungeminierte unpalatale Entsprechung.
Im Kypr. ist das auch bei \rek{ĺĺ} der Fall (zumindest nach \bel{a}).
Ansonsten wird \rek{ĺĺ} entpalatalisiert und bleibt Geminate.
Im Thess. und Lesb. werden \rek{ŕŕ}, \rek{ńń} in allen anderen Fällen entpalatalisiert.

\lex{\rek{ŕŕ, ńń, ẅẅ}}{\bel{i̯r, i̯n, i̯u̯}}{(a, o)\_}

\lex{\rek{ĺĺ}}{\bel{i̯l}}{(a, o)\_} (kypr.)

\lex{\rek{ĺĺ}}{\bel{ll}}

\lex{\rek{ŕŕ, ńń}}{\bel{rr, nn}} (thess., lesb.)

\mypara{Erste Ersatzdehnung}
\label{ED1}
Die erste Ersatzdehnung dehnt grob gesprochen Vokale vor palatalen Geminaten,
\rek{h} + Resonant und \rek{ln} außer im Thess. und Lesb.,
wo sie später geminiert werden (s. \ref{gem}).
Gedehnte \rek{e}, \rek{o} fallen dabei nur dialektal mit alten \rek{ē}, \rek{ō}
zusammen.
Im Ion-att., Nordwestdor. und Saronisch-dor. (TODO: quelle bartonek)
entstehen dabei neue geschlossene Langvokale \rek{ẹ̄}, \rek{ọ̄}.

Datiert werden muss der Wandel vor der Hebung von \rek{ā} (\ref{a>e})
im Ion-att.

\lex{\rek{(e, i, u)ŔŔ}}{\bel{(ẹ̄, ī, ū)R}} (mit Ŕ = ŕ, ń, ẅ)

\lex{\rek{VRh}, \rek{VhR}}{\rek{V̄R}}{\_V} (mit R = Resonant oder u̯; nicht thess., lesb.)

\lex{\rek{Vln}}{\bel{V̄l}}{\_n} (nicht thess., lesb.)

\lex{\rek{Vln}}{\bel{Vll}}

\mypara{\rek{u̯s} > \rek{u̯h}}
\rek{s} nach \rek{u̯}, das noch nicht zu \rek{h} verhaucht ist (s. \ref{sh}),
wird es jetzt,
da es nicht der Ersatzdehnung, aber der etwas späteren Gemination (\ref{gem})
unterworfen ist.
Eine reine Gemination wäre zwar auch mit \rek{s} denkbar,
aber weil damit der Hauchumsprung einherzugehen scheint,
ist eine Verhauchung wahrscheinlich.

\lex{\rek{s}}{\rek{h}}{u̯\_}

\mypara{Hauchumsprung}
Die genauen Bedingungen für den Hauchumsprung sind etwas unklar.
Grob gesagt springt ein \rek{h} direkt nach der ersten Silbe
in den Anlaut derselben.
Wenn das \rek{h} intervokalisch ist,
geschieht dies nur, wenn die zweite Silbe unbetont ist.
Zwischen Resonant oder \bel{u̯} und Vokal scheint \rek{h}
nicht nur in den Anlaut der ersten Silbe zu springen,
sondern auch am urpsprünglichen Ort zu bleiben.

Die Formulierung klingt  nicht besonders überzeugend.
Möglicherweise sind die Regeln dialektal auch unterschiedlich.

\lex{\rek{VhV}}{\rek{hVV}}

\lex{\rek{V(R, u̯)hV}}{\rek{hV(R, u̯)hV}}

\mypara{Gemination}
\label{gem}
Zwischen Vokalen assimiliert sich \rek{h} neben Resonanten oder \bel{u̯} an diesen Laut,
so dass eine Geminate entsteht.

\lex{\rek{(R, u̯)h}}{\bel{(RR, u̯u̯)}}{V\_V}

\mypara{\rek{h} > \bel{Ø}}
\rek{h} schwindet außer im Anlaut.

\lex{\rek{h}}{\bel{Ø}} (außer \#\_)

\mypara{\rek{ts}}
\rek{ts} wird nach Konsonant (also auch nach Diphthong) zu \bel{s} vereinfacht.
Vor Konsonant ist dies schon bei \ref{NTsC} passiert.

Nach Vokal wird es Im Böot. und Kret. zu \bel{tt},
sonst überall zunächst zu \bel{ss},
das dann in allen Dialekten nach Langvokal,
im Ion-att. und Ark. überall vereinfacht wird. (Rix §102, Sihler 196)

\lex{\rek{ts}}{\bel{s}}{C\_}

\lex{\rek{ts}}{\bel{tt}} (böot., kret.)

\lex{\rek{ts}}{\bel{ss}}

\lex{\rek{ss}}{\bel{s}}{V̄\_}

\lex{\rek{ss}}{\bel{s}}	(ion-att., ark.)

\mypara{\bel{sd} > \bel{d(d)}}
In einigen Dialekten
wird \bel{sd} zu \bel{dd}, im Anlaut zu \bel{d}.

\lex{\bel{sd}}{\bel{dd}} (böot., kret., lak., el.)

\lex{\bel{dd}}{\bel{d}}{\#\_}

\subsection*{ca. 900}

\mypara{\bel{ā} > ion-att. \bel{ǣ}}
\label{a>e}
Das offensichtlichste Merkmal der ion-att. Dialekte.
\bel{ā} wird gehoben, fällt jedoch noch nicht mit ererbtem \bel{ē} zusammen,
da die beiden Laute im Inselion. noch mit unterschiedlichen Zeichen geschrieben werden.

Der Lautwandel ist zwischen der ersten und zweiten Ersatzdehnung anzusetzen,
da durch die erste Ersatzdehnung entstandenes \bel{ā} den Lautwandel mitmacht
und bei der zweiten ein neues \bel{ā} entsteht.

\lex{\bel{ā}}{\bel{ǣ}} (ion-att.)

\subsection*{ca. 800}

\mypara{Zweite Ersatzdehnung}
\label{ED2}

Bei der zweiten Ersatzdehnung schwindet \bel{n}
in der Lautgruppe Vokal + \bel{ns},
das entweder im Auslaut steht und somit nicht
der ersten Ersatzdehnung (s. \ref{ED1}) unterlag,
oder im Inlaut mit neuentstandenem \bel{s} < \rek{ts}.

Die zweite Ersatzdehnung betrifft nicht alle Dialekte in gleicher Art.
In einigen wird der Vokal gedehnt,
in anderen wird er zum i-Diphthong,
und in einigen unterbleibt der n-Schwund und damit die Ersatzdehnung völlig.
Das genaue Ergebnis kann inlautend und auslautend unterschiedlich sein
und auch vom Vokal abhängen.
Auslautendes \bel{-Vns} konnte schon durch \ref{NTsC} als antekonsonantische
Sandhivariante zu \bel{-Vs} werden und verallgemeinert werden,
so dass einige Dialekte zur Zeit der zweiten Ersatzdehnung \bel{-Vns}
schon ganz aufgegeben haben.

Die Ergebnisse von gedehnten \bel{e}, \bel{o} sind
im Ion-att., Norwestdor. und Inseldor. die geschlossenen Langvokale
\bel{ẹ̄}, \bel{ọ̄}, ansonsten fallen sie mit den alten offenen zusammen.

Die genauen Regeln sind also nicht ganz einfach zu formulieren.
Im wesentlichen halte ich mich an die Karte,
die bei \textcite{bart_frueh}
abgedruckt ist
(ursprünglich von Hainsworth, \textit{Cambridge Ancient History} III 1, 863).

\lex{\bel{Vns}}{\bel{Vi̯s}}{\_V} (lesb.)

\lex{\bel{Vns}}{\bel{V̄s}}{\_V} (ion-att., böot., el., lak., inseldor., kypr.)

\lex{\bel{Vns}}{\bel{Vi̯s}}{\_\#} (lesb., el.)

\lex{\bel{Vns}}{\bel{V̄s}}{\_\#} (ion-att., böot., lak., inseldor.)

\mypara{Kontraktionen}
\label{kontr1}
Einige Kontraktionen passieren vor dem Digammaschwund,
sind aber teilweise unklar und schwer zu formulieren.
Da keine Ergebnisse besser als schlechte sind,
werden diese hier übersprungen.

\mypara{Digamma}
Bei \bel{u̯} ist die dialektale Entwicklung äußerst schwierig.
Teilweise geht es vor der Überlieferung verloren (z.B. ion-att.),
teilweise ist es bis heute im Tsakonischen bewahrt.
Eine genaue Untersuchung wäre gerade zu diesem Thema sehr wünschenswert.
Die Darstellung hier beruht hauptsächlich auf \textcite{buck} und \textcite{lejeune} und sollte als sehr vorläufig betrachtet werden.

Manchmal wird \bel{u̯} im Anlaut auch zu \bel{h} statt ganz zu schwinden,
ohne dass die Bedingungen genau bekannt wären.

\lex{\bel{u̯}}{\bel{Ø}}{V\_V} (ion-att., inseldor., lesb., kret., ark.,)

\lex{\bel{u̯}}{\bel{h}}{\#\_Vs} (ion-att., inseldor., lesb.)

\lex{\bel{u̯}}{\bel{Ø}}{\#\_V} (ion-att., inseldor., lesb.)

\lex{\bel{i̯u̯}}{\bel{i̯i̯}}{V\_V} (ion-att., inseldor., lesb.)

\lex{\bel{u̯}}{\bel{Ø}}{h\_V} (ion-att., inseldor., lesb.)

\lex{\bel{u̯}}{\bel{h}}{\#\_r} (ion-att., inseldor.)

\mypara{\bel{\#hR}}
Vor \bel{l}, \bel{m}, \bel{n} schwindet \bel{h} im Ion-att.
In welchen Dialekten und in welcher Zeit dies noch geschieht,
bleibt herauszufinden.

\lex{\bel{h}}{\bel{Ø}}{\#\_(l, m, n)}

\mypara{att. \bel{ǣ} > \bel{ā}}
Das wichtigste Merkmal, das das Attische vom Ionischen unterscheidet.
Nach \bel{e}, \bel{i}, \bel{r} wird gehobenes \bel{ǣ} zurück zu \bel{ā}
gesenkt.

\lex{\bel{ǣ}}{\bel{ā}}{(e, ē, ẹ̄, i̯, i, ī, r)\_}

%TODO: vielleicht etwas früher?
\subsection*{ca. 700}

\mypara{Dritte Ersatzdehnung und Kontraktion}
\label{ED3}
In der Lautgruppe
Resonant, \bel{d} oder \bel{s} + \bel{u̯} schwindet das \bel{u̯} in vielen Dialekten.
In einigen Dialekten geht das \bel{u̯} spurlos verloren,
in anderen wird vorangehender Vokal ersatzgedehnt
und in anderen bleibt die Lautkombination intakt.
Das Ergebnis von gedehnten \bel{e}, \bel{o} sind die geschlossenen
\bel{ẹ̄}, \bel{ọ̄} im Att-ion., Nordwestdor. und Saronisch-dor.

Dasselbe Ergebnis haben die Kontraktionen von \bel{e} + \bel{e}
und \bel{o} + \bel{o}.

\lex{\bel{(\#, V)(r, l, n, d, s)u̯}}{\bel{(\#, V̄)(r, l, n, d, s)}}
(inselion., ostion., inseldor., kret.)

\lex{\bel{u̯}}{\bel{Ø}}{\bel{(r, l, n, d, s)\_}} (ion-att., lesb., lak.)

\lex{\rek{ee}}{\bel{ẹ̄} bzw. \bel{ē}}

\lex{\rek{oo}}{\bel{ọ̄} bzw. \bel{ō}}

%TODO: č > tš &c.
\mypara{\rek{tš}}
\rek{tš} scheint in einigen ionischen Dialekten länger erhalten geblieben
zu sein, da ein extra Zeichen für diesen Laut benutzt wird,
auch wenn die genaue Lautung unsicher ist.
Daher setzte ich diesen Lautwandel eher spät an
(und ungenauerweise in allen Dialekten gleichzeitig).
In den meisten Dialekten wird \rek{tš} zu \bel{ss}, das im Anlaut vereinfacht wird.
Im Att., Euboi., Boiot. und Kret. wird es jedoch zu \bel{tt} bzw. \bel{t}
im Anlaut.

\lex{\rek{tš}}{\bel{t}}{\#\_} (att., euboi., boiot., kret.)

\lex{\rek{tš}}{\bel{tt}} (att., euboi., boiot., kret.)

\lex{\rek{tš}}{\bel{s}}{\#\_}

\lex{\rek{tš}}{\bel{ss}}

\mypara{Psilose}
In einigen Dialekten geht anlautendes \bel{h} verloren (die einzige Position,
in der es überhaupt vorkommt).
Die genaue Datierung (gerade bei später belegten Dialekten wie dem Lesb.)
ist unsicher.

\lex{\bel{h}}{\bel{Ø}} (el., lesb., ostion., inselion., kret.)

\mypara{\bel{ǣ}}
\label{ae>e}
Das aus \bel{ā} gehobene \bel{ǣ} fällt mit \bel{ē} zusammen,
ist jedoch im Inselion. noch länger erhalten, wo die beiden Laute graphisch
unterschieden sind.

\lex{\bel{ǣ}}{\bel{ē}} (inselion. später)

\mypara{Kontraktionen}
\label{kontr2}
Einige Kontraktionen passieren nach dem Digammaschwund
(und vermutlich nach \ref{ae>e}),
sind aber ebenso schwer genau zu formulieren
wie die bei \ref{kontr1} übersprungenen
und werden deswegen ebenfalls übergangen.

\printbibliography

\end{document}
