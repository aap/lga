\documentclass[12pt,a4paper,normalheadings]{scrartcl}
\usepackage[a4paper,top=3cm, bottom=3cm, left=4cm, right=3cm]{geometry}

\usepackage{polyglossia}
\defaultfontfeatures{Mapping=tex-text}
\setmainlanguage{german}
\setotherlanguage[variant=ancient]{greek}
\input{$HOME/griechisch/greektokens.tex}
\setmainfont{Junicode}
\newfontfamily\greekfont{GFS Neohellenic}

\usepackage{setspace}
\usepackage{calc}

\usepackage{enumitem}
\setlist{parsep=0pt,listparindent=\parindent}

\usepackage[backend=bibtex,style=authoryear,maxcitenames=1,language=german]{biblatex}
\bibliography{ba}

%\usepackage[texcoord,gridunit=cm]{showframe}

\newcommand{\up}[1]{\textsuperscript{#1}}
\newcommand{\ul}[1]{\underline{#1}}
\newcommand{\ur}[1]{\up{\ref{#1}}}
\def\rek#1{*\textit{#1}}
\def\bel#1{\textit{#1}}

\begin{document}

\section{Einleitung}

Ziel dieser Arbeit ist es, ein Program zu entwickeln,
das Lautgesetze auf Wörter anwendet,
und die Lautgesetze vom Urindogermanischen in die altgriechischen Dialekte
zu sammeln und zu sortieren.

Das junggrammatische Postulat der Ausnahmslosigkeit der Lautgesetze,
das fordert, dass Lautgesetze wie mechanisch ablaufen,
wirkt sehr einladend,
tatsächlich einen Mechanismus zu entwickeln,
der die Anwendung von Lautgesetzen simuliert.
Im 19. Jh. war dies natürlich unmöglich,
aber trotz der Computerrevolution des 20. Jh. hat sich offenbar
noch niemand daran gemacht,
diese eigentlich sehr naheliegenden Idee zu verwirklichen.

Das hier besprochene Programm \texttt{lga} ist
der Nachfolger eines Prototyps (komplett in \texttt{sed} geschrieben),
der primitiver war und einige Mängel aufzeigte.
Ebenso ist aber \texttt{lga} nur als Prototyp
für ein elaborierteres Programm zu verstehen.
Im Laufe der Arbeit wird sich herausstellen,
welche Probleme, Mängel und weiteren Anforderungen es gibt,
die man in einer nächsten Version angehen müsste.

Im ersten Teil dieser Arbeit wird \texttt{lga}
in seinen Designideen und in seiner Funktionsweise vorgestellt.
Im zweiten Teil wird \texttt{lga} benutzt,
um die Lautentwicklung vom Urindogermanischen ins Altgriechische zu modellieren.

\section{lga}

Wenn ein Lautgesetzt wirkt,
wird ein Laut (oder mehrere) durch einen anderen ersetzt.
Da man Laute phonetisch notieren kann,
kann man auch ein Lautgesetz durch Ersetzung
von einem Zeichen durch ein anderes beschreiben.

Da Lautwandel meist durch die Lautumgebung bedingt ist
und sich oft nicht nur ein Laut sondern Lautgruppen verändern
(bspw. Stimmloswerdung von stimmhaften Lauten),
ist es sinnvoll,
eine Sprache zu entwicklen,
die Lautgesetze kompakt beschreiben
und von einem Computerprogramm verstanden werden kann.
Eine naheliegende Wahl ist daher eine Form von \emph{regulären Ausdrücken},
die in vielen Programmen implementiert und weithin bekannt sind.

\section{Lautentwicklung des Altgriechischen}

\printbibliography

\end{document}
