\documentclass[12pt,a4paper,normalheadings,bibliography=totoc]{scrartcl}
\usepackage[a4paper,top=3cm, bottom=3cm, left=4cm, right=3cm]{geometry}

\usepackage{polyglossia}
\defaultfontfeatures{Mapping=tex-text}
\setmainlanguage{german}
\setotherlanguage[variant=ancient]{greek}
\input{$HOME/griechisch/greektokens.tex}
\setmainfont{Junicode}
\setmonofont[Scale=MatchLowercase]{DejaVu Sans Mono}
%\setmonofont{Courier New}
\newfontfamily\greekfont{GFS Neohellenic}

\usepackage{setspace}
\usepackage{calc}

\renewcommand{\theparagraph}{\arabic{paragraph}}
\setcounter{secnumdepth}{5}

\newcounter{para}
\newcommand{\mypara}[1]{\vspace{1em}\par\noindent\refstepcounter{para}%
%\textbf{\textsf{\thepara}}\enspace{\large#1}\quad}
\textbf{\textsf{\thepara}}\enspace\textbf{#1}\quad}
%\textbf{\textsf{\thepara}}\enspace\textsf{#1}\quad}

\usepackage{enumitem}
\setlist{parsep=0pt,listparindent=\parindent}

\usepackage[backend=bibtex,style=authoryear,maxcitenames=2,language=german]{biblatex}
\bibliography{ba}

%\usepackage[texcoord,gridunit=cm]{showframe}

\usepackage{xparse}

\NewDocumentCommand{\lex}{mmg}{%
  \medskip
  \noindent
  \IfNoValueTF{#3}
    {%
      #1 > #2%
    }
    {%
      #1 > #2 \textbackslash\ #3%
    }%
}

\newcommand{\up}[1]{\textsuperscript{#1}}
\newcommand{\ul}[1]{\underline{#1}}
\newcommand{\ur}[1]{\up{\ref{#1}}}
\def\rek#1{\mbox{*\textit{#1}}}
\def\bel#1{\mbox{\textit{#1}}}
\def\i#1{\mbox{\textit{#1}}}
\def\tt#1{\texttt{#1}}
\def\rx#1{\texttt{/#1/}}
\def\str#1{\texttt{"#1"}}

\begin{document}
\begin{titlepage}
{
\begin{center}
\huge Computersimulierter Lautwandel:\\
\Large ~\\Vom Urindogermanischen in die altgriechischen Dialekte
\vfill

\noindent
\large
Bachelorarbeit\\
zur Erlangung des akademischen Grades\\
Bachelor of Arts (B.A.)\\
im Fach Historische Linguistik
\end{center}
\vfill

\normalsize
\noindent
\begin{tabular}{l}
Humboldt-Universität zu Berlin\\
Philosophische Fakultät II\\
Institut für deutsche Sprache und Linguistik
\end{tabular}
\vspace{3em}

\noindent
\begin{tabular}{l l}
eingereicht von & Angelo Papenhoff\\
& geb. am 24.7.1992\\
& in Berlin
\end{tabular}
\vspace{3em}

\noindent
\begin{tabular}{l l}
1. Gutachter: & Herr Prof. Dr. Wolfgang Hock\\
2. Gutachterin: & Frau Prof. Dr. Anke Lüdeling
\end{tabular}
\vspace{3em}

\noindent
\begin{tabular}{l l}
Berlin, den 10.8.2015
\end{tabular}
}
\end{titlepage}

\setcounter{page}{1}
\tableofcontents
\newpage

\onehalfspacing


\section{Abkürzungen und Konventionen}

Reguläre Ausdrücke werden in Schreibmaschinenschrift und zwischen
Schrägstriche gesetzt (\rx{regex}).
Strings werden in Schreibmaschinenschrift und Anführungszeichen gesetzt
(\str{string}).
Scheme-Code wird in Schreibmaschinenschrift gesetzt.
Bedeutungen werden in einfache Anführungszeichen gesetzt (`foo')
Phoneme werden kursiv zwischen Schrägstriche gesetzt (/\i{a}/).
Phone bzw. Phonemkandidaten werden kursiv gesetzt;
wird die Phonetik explizit betont, stehen Phone in eckigen Klammern ([\i{a}]).
Grapheme werden in spitzen Klammern gesetzt (<\i{a}>).
Rekonstrukte werden mit Sternchen markiert (\rek{a}).
In den formulaischen Lautgesetzen werden rekonstruierte
und belegte Laute nicht unterschieden.
Die Lautschrift ist an das IPA angelehnt,
folgt ihm aber nicht völlig
(teilweise um eine Festlegung der exakten Lautung zu vermeiden).
Lautgesetzliche Entwicklung wird mit > bzw. < bezeichnet.

\subsection*{Sprachen}
\begin{tabular}{l l}
\begin{tabular}{l l}
ach.	& achaisch\\
ahd.	& althochdeutsch\\
ai.	& altindisch\\
aiol.	& aiolisch\\
aitol.	& aitolisch\\
ark.	& arkadisch\\
att.	& attisch\\
av.	& avestisch\\
boiot.	& boiotisch\\
dor.	& dorisch\\
el.	& elisch\\
euboi.	& euboiisch\\
got.	& gotisch\\
gr.	& griechisch\\
hom.	& homerisch\\
idg.	& indogermanisch\\
inseldor.	& inseldorisch\\
inselion.	& inselionisch\\
ion.	& ionisch\\
kor.	& korinthisch\\
kret.	& kretisch\\
\end{tabular}
&
\begin{tabular}{l l}
kypr.	& kyprisch\\
lak.	& lakonisch\\
lat.	& lateinisch\\
lesb.	& lesbisch\\
lit.	& litauisch\\
lokr.	& lokrisch\\
meg.	& megarisch\\
myk.	& mykenisch\\
nwdor.	& nordwestdorisch\\
nwgr.	& nordwestgriechisch\\
oarg.	& ostargolisch\\
ostion.	& ostionisch\\
othess.	& ostthessalisch\\
pamph.	& pamphylisch\\
phok.	& phokisch\\
sardor. & saronisches dorisch\\
sdor.	& süddorisch\\
sogr.	& südostgriechisch\\
thess.	& thessalisch\\
warg.	& westargolisch\\
wthess.	& westthessalisch\\
\end{tabular}
\\
\end{tabular}

\subsection*{Grammatisch Abkürzungen}
\begin{tabular}{l l}
\begin{tabular}{l l}
A.	& Akkusativ\\
m.	& maskulinum\\
Aor.	& Aorist\\
D.	& Dativ\\
Du.	& Dual\\
f.	& femininum\\
G.	& Genitiv\\
I.	& Instrumental\\
\end{tabular}
&
\begin{tabular}{l l}
Imp.	& Imperativ\\
L.	& Lokativ\\
n.	& neutrum\\
N.	& Nominativ\\
Pass.	& Passiv\\
Pl.	& Plural\\
Sg.	& Singular\\
St.	& Stamm\\
\end{tabular}
\\
\end{tabular}

\subsection*{Coversymbole}
\begin{tabular}{l l}
\begin{tabular}{l l}
C	& Konsonant\\
D	& Media\\
H	& Laryngal\\
K	& Velar\\
Kʷ	& Labiovelar\\
L	& Liquida\\
\end{tabular}
&
\begin{tabular}{l l}
N	& Nasal\\
R	& Resonant\\
T	& Tenuis\\
Tʰ	& Tenuis aspirata\\
V	& Vokal\\
V̯	& Halbvokal\\
\end{tabular}
\\
\end{tabular}

\newpage

\section{Einleitung}

Ziel dieser Arbeit ist es,
den Lautwandel vom Urindogermanischen in die altgriechischen Dialekte
am Computer zu simulieren.
Seit den Junggrammatikern im 19. Jh. spricht man
in der historisch-vergleichenden Sprachwissenschaft von ``Lautgesetzen''.
Dieser Begriff entsprang dem Bestreben,
die Linguistik von der Geisteswissenschaft zur Naturwissenschaft zu machen
und suggeriert, dass der Lautwandel
einen ähnlichen Status habe wie die Naturgesetze in der Physik:
wann immer ein Gesetz wirken kann, wirkt es ohne Ausnahme.
Diese Lautgesetze müssen für eine der historischen Entwicklung möglichst
nahekommende Simulation präzise und in korrekter Chronologie formuliert werden,
was allein schon ein großer Nutzen wäre,
da alle Beschreibungen der Lautwandel in der Literatur
immer nach Lauten, nie zeitlich sortiert sind.
Dies wird gegenüber der traditionellen Arbeit noch dadurch erleichtert,
dass sich die Richtigkeit der Lautwandelmodellierung
an sicheren Beispielwörtern ohne großen Aufwand überprüfen lässt
und umgekehrt sich bei einer gegebenen Modellierung
ohne Aufwand die Lautgesetzlichkeit einzelner Wörter feststellen lässt.
Beides wäre bei manueller Arbeit fehleranfällig und mühsam;
eine Automatisierung ist also äußerst lohnend.

Das hier vorgestellte Programm \tt{lga} ist
der Nachfolger eines Prototyps,
der ein Skript für das UNIX-Programm \tt{sed} war.
Dieser war sehr primitiv und wies viele Mängel auf,
zeigte aber,
dass sich die Arbeit an einem besseren Programm lohnen würde.
Ebenso ist allerdings \tt{lga} nur als Prototyp
für ein elaborierteres Programm zu verstehen.
Im Laufe der Arbeit wird sich herausstellen,
welche Probleme, Mängel und weitere Anforderungen es gibt,
die in einer nächsten Version angegangen werden müsste.

Nach einigen methodisch-theoretischen Überlegungen zum Lautwandel
werde ich im ersten Teil dieser Arbeit \tt{lga}
in seinen Designideen und in seiner Funktionsweise vorstellen.
Im zweiten Teil stelle ich eine vorläufige Formulierung und relative
(teilweise absolute) Chronologie der Lautgesetze dar,
die die Entwicklung vom Urindogermanischen in die altgriechischen Dialekte
beschreiben.
Hierfür wäre \tt{lga} natürlich prinzipiell nicht nötig gewesen
und wird auch im Text nicht weiter erwähnt.
Ohne dieses nützliche Werkzeug hätte jedoch
die Aufstellung einige Fehler gehabt,
die man so schnell nicht gefunden hätte.

\section{Zum Lautwandel allgemein}

Die wahrscheinlich bedeutendste Entdeckung
in der historisch-vergleichenden Sprachwissenschaft ist
das junggrammatische Postulat der \emph{Ausnahmslosigkeit der Lautgesetze}.
Im Vorwort zu den \textit{Morphologischen Untersuchungen}
schreiben \textcite[XIII]{brugmann_osthoff}:

\begin{quote}
Die zwei wichtigsten von den methodischen grundsätzen der `junggrammatischen'
richtung sind folgende.

\emph{Erstens}. Aller lautwandel, so weit er mechanisch vor sich geht,
vollzieht sich nach \emph{ausnahmslosen gesetzen},
d. h. die richtung der lautbewegung ist bei allen angehörigen
einer sprachgenossenschaft, ausser dem fall, dass dialektspaltung
eintritt, stets dieselbe, und alle wörter, in denen der der
lautbewegung unterworfene laut unter gleichen verhältnissen erscheint,
werden ohne ausnahme von der änderung ergriffen.

\emph{Zweitens}. Da sich klar herausstellt,
dass die formassociation, d. h. die neubildung von sprachformen auf dem wege
der analogie, im leben der \emph{neueren}
sprachen eine sehr bedeutende rolle spielt,
so ist diese art von sprachneuerung unbedenklich auch für die \emph{älteren}
und \emph{ältesten} perioden anzuerkennen,
und nicht nur überhaupt anzuerkennen, sondern es ist dieses
erklärungsprincip auch in derselben weise zu verwerten,
wie zur erklärung von spracherscheinungen späterer perioden,
und es darf nicht im mindesten auffallen,
wenn analogiebildungen in den älteren und ältesten sprachperioden
\emph{in demselben umfange} oder gar \emph{in noch grösserem umfange}
uns entgegentretet wie in den jüngeren und jüngsten.
\end{quote}

Im darauf folgenden Absatz erklären sie,
dass durch die Ausnahmslosigkeit der Lautgesetze man überhaupt
erst ``\emph{festen boden unter den füssen}'' bekomme.
Dass Sprachwandel nicht allein aus ausnahmslosem Lautwandel besteht,
liegt auf der Hand.
Die Junggrammatiker stellen ihm die Formassoziation d.h. Analogie entgegen,
aber auch die Kombination von ausnahmslosem Lautwandel und Analogie kann nicht 
allen Sprachwandel erklären:
auch Dialektentlehnungen,
expressivie Bildungen
oder lexikalisch beschränkte Lautveränderungen
wie Haplologie oder Dissimilationen
können dem regulären Lautwandel entgegenwirken.

Im Kapitel ``The regularity of sound change'' deduziert \textcite{katicic}
auf Basis theoretischer Annahmen,
dass regulärer Lautwandel ausnahmslos sein muss
und alle scheinbaren Ausnahmen durch andere Sprachveränderungen
erklärt werden müssen.
Katičić unterscheidet externen und internen Sprachwandel.
Beim externen Sprachwandel haben die Sprecher einer Sprache den Anreiz,
eine neue Sprache zu lernen (z.B. die Sprache ihrer Eroberer).
Beim internen Sprachwandel
ist die neue Sprache innerhalb der Sprachgemeinschaft entstanden und
die Sprecher haben prinzipiell keinen Anreiz,
die Sprache der Innovatoren zu lernen.
Wenn sich daher die neue Sprache durchsetzen will,
muss sie mühelos von den Sprechern der alten Sprache gelernt werden können,
was impliziert,
dass die neue Sprache möglichst einfach aus der alten abgeleitet werden können muss.
Dies ist (auf die Lautebene bezogen) dann der Fall,
wenn sich der Lautwandel als eindeutige Abbildung (d.h. als Funktion)
von Lauten modellieren lässt,
so dass ein Sprecher anhand möglichst einfacher Regeln
aus jedem Laut der alten Sprache den entsprechenden Laut der neuen Sprache
herleiten kann.

Dieser ausnahmslose und mechanische Lautwandel ist es,
der von \tt{lga} simuliert wird.

\section{lga}

\subsection{Vorüberlegungen}

Wenn ein Lautgesetzt wirkt,
wird ein Laut oder eine Lautgruppe durch eine andere oder durch nichts ersetzt.
Da man Laute phonetisch notieren kann,
kann man auch ein Lautgesetz durch Ersetzung
von Zeichen durch andere Zeichen beschreiben.

Da Lautwandel meist durch die Lautumgebung bedingt ist
und sich oft nicht nur ein Laut, sondern Lautklassen verändern
(bspw. Stimmloswerdung von stimmhaften Lauten),
ist es sinnvoll,
eine Sprache zu entwicklen,
die Lautgesetze kompakt beschreiben
und von einem Computerprogramm verstanden werden kann.
Eine naheliegende Wahl ist daher eine Form von \emph{regulären Ausdrücken},
die in vielen Programmen implementiert und weithin bekannt sind.
Reguläre Ausdrücke sind zwar für diesen Zweck nicht perfekt geeignet,
aber für einen Prototypen ausreichend.
Wenn sich gezeigt hat,
welche Defizite reguläre Ausdrücke für diesen Anwendungsfall haben,
wird man eine besser geeignete Sprache entwickeln und implementieren können.

Als Programmiersprache habe ich die Scheme-Implementation CHICKEN
\footnote{\tt{http://www.call-cc.org/}, abgerufen am 10.8.2015}
und die \tt{irregex}-Engine für reguläre Ausdrücke
\footnote{\tt{http://wiki.call-cc.org/man/4/Unit\%20irregex}, abgerufen am 10.8.2015}
gewählt.

\subsection{Reguläre Ausdrücke}

Hier eine (informelle)
Beschreibung der hier verwendeten Untermenge regulärer Ausdrücke.
Die vollständige Dokumentation für die \tt{irregex}-Engine für
CHICKEN findet sich unter dem oben genannten Link.

Die grundlegende Funktion von regulären Ausdrücken ist das
Finden (engl. \bel{match}) von Zeichenketten (String) anhand eines Musters.
Der gematchte Text kann dann durch einen anderen ersetzt werden,
wobei Teile des Matches wieder aufgenommen werden können.

Ein regulärer Ausdruck matcht einen String aus einem Alphabet.
Dies ist hier die Menge der Unicode \emph{Codepoints}, nicht die
der Unicode \emph{Grapheme},
welche nämlich aus mehreren Codepoints bestehen können.

Ein Zeichen des Alphabets, das kein Metazeichen ist (dazu unten),
matcht sich selbst.
Metazeichen verlieren ihre spezielle Funktion,
wenn ihnen ein Backslash \tt{\textbackslash} vorangeht.
Der Ausdruck \rx{foo} matcht also den String \str{foo}
(und auch das \str{foo} in \str{foobar}).

Der Punkt . matcht ein beliebiges Zeichen.
Der Ausdruck \rx{...} matcht also alle Strings mit drei Zeichen,
\rx{\textbackslash.\textbackslash.\textbackslash.} matcht drei Punkte.

Eine in eckigen Klammern stehende Menge an Zeichen matcht eines dieser Zeichen,
oder alle außer dieser Zeichen,
wenn das erste Zeichen ein Zirkumflex \tt{\textasciicircum} ist.
Dabei können auch Zeichenbereiche mit Bindestrich angegeben werden.
\rx{[abc]} matcht also \str{a}, \str{b} oder \str{c},
\rx{[\textasciicircum{}abc]} alle Zeichen außer \str{a}, \str{b} und \str{c}.
Alternativ hätte man auch \rx{[a-c]} bzw. \rx{\textasciicircum[a-c]}schreiben können.

Der Zirkumflex \tt{\textasciicircum} und das Dollarzeichen \tt{\$}
matchen den Anfang bzw. das Ende einer Zeile.
\rx{\textasciicircum{}foo} matcht also \str{foo} am Anfang der Zeile,
\rx{bar\$} matcht \str{bar} am Ende der Zeile
und \rx{\textasciicircum{}quux\$} die Zeile, die nur \str{quux} enthält.
Da in \tt{lga} jedes Wort in einer eigenen Zeile steht,
kann man die Zeichen für den Anfang bzw. das Ende des Wortes benutzen.

\tt{?}, \tt{*} und \tt{+} sind Quantoren
und matchen, was ihnen vorangeht, null oder einmal, null bis unendlich mal
und ein bis unendlich mal.
Der Ausdruck \rx{.?[a-d]*x+} z.B.
matcht optional ein beliebiges Zeichen, dann eins der Zeichen \str{a}-\str{d}
null bis unendlich mal und schließlich mindestens ein \str{x}.

Mit Klammern \tt{()} werden analog zur mathematischen Notation
Matches enger gruppiert.
So würde \rx{[ab]c+} \str{a} oder \str{b} und dann mindestens ein
\str{c} matchen (also z.B. \str{accc} oder \str{bc}).
\rx{([ab]c)+} dagegen matcht \str{[ab]c} mindestens einmal
(also z.B. \str{acbcbc}).

Neben ihrer Gruppierungsfunktion werden die Submatches innerhalb
der Klammern für die Ersetzung gespeichert.
\rx{([0-9]+)foo([0-9]+)} würde also zwei Zahlen und \str{foo} dazwischen
matchen.
Die Zahlen wären dann in den Submatches \tt{1} und \tt{2} gespeichert
und können im Ersetzungstext wieder aufgegriffen werden.
Dabei gilt, dass verschachtelte Klammerausdrücke vor Klammern derselben
Ebene numeriert werden. Also wären die Submatches in z.B. \rx{(a(b)(c))(x)}
der Reihe nach \str{abc}, \str{b}, \str{c}, \str{x}.

Mit dem senkrechten Strich in z.B. \rx{e0|e1} wird \str{e0}
oder \str{e1} gematcht.
Der Senkrechtstrich hat die niedrigste Präzedenz,
\rx{foo|bar+} matcht also \str{foo} oder mindestens ein \str{bar}.
\rx{(foo|bar)+} dagegen matcht mindestens ein \str{foo} oder \str{bar}.

\subsection{Funktionsweise}

Im Kern arbeitet \tt{lga} mit einer Liste von Wörtern
und einer Liste von Lautgesetzen
(in der Form von Textersetzungen durch reguläre Ausdrücke)
und wendet der Reihe nach alle Lautgesetze auf die Liste der Wörter an.
Das Resultat sind die Wörter nach Anwendung aller Lautgesetze.

Die Liste der Lautgesetze aber hängt davon ab,
von welcher in welche Sprache die Wörter transformiert werden sollen
und wird aus einem \emph{Baum} von Lautgesetzen und einem
zugehörigen Sprachstammbaum generiert.

Die Liste der Wörter wird vorher durch eine weitere Liste von Regeln,
die von der Ausgangssprache abhängt,
von einer zugrundeliegenden
graphematisch/phono\-logischen
Form in eine interne phonetische gebracht
(was natürlich eine eindeutige Graphie voraussetzt).
Die Phoneme dieser Form sind als zugrundeliegende Segmente im
Sinne der generativen, nicht der strukturalistischen Phonologie zu verstehen.
Am Ende wird die phonetische Form durch eine Liste von Regeln,
die von der Zielsprache abhängig ist,
wieder in eine graphematische Darstellung gebracht.

Die Funktion \tt{(run-list from to words)}
generiert die drei Listen von Regeln anhand der Sprachen \tt{from} und \tt{to},
macht aus der Liste von Wörtern \tt{words} aus Effizienzgründen einen
einzigen String,
in dem jedes Element von \tt{words} in einer eigenen Zeile steht,
wendet darauf die drei Regellisten an,
und gibt das Ergebnis zurück.

Zunächst wird mit der Funktion \tt{(make-path tree start end)} anhand
eines Sprachstammbaums \tt{tree}
eine Liste aller Sprachen von der Ausgangs- zur Zielsprache generiert.
Der Aufruf \tt{(make-path lang-tree 'uridg 'lesb)} würde bspw.
mit dem Beispielbaum unten die Liste \tt{(uridg urgr nwgr aiol lesb)} erzeugen.

Der Sprachstammbaum (gespeichert in der globalen Variable \tt{lang-tree})
definiert das Verwandtschaftsverhältnis von Sprachen.
Wenn eine Sprache keine Untersprachen hat,
ist sie (in Scheme-Terminologie) ein Symbol
(also hat z.B. \tt{myk} keine Abkömmlinge).
Hat eine Sprache Untersprachen,
ist sie eine Liste, dessen erstes Element ein Symbol für die Sprache selbst ist;
die weiteren Elemente sind Untersprachen (wiederum Symbole oder Listen).
Ein vereinfachter Stammbaum könnte also bspw. folgendermaßen aussehen:
\begin{verbatim}
(urgr (nwgr (dor nwdor sardor sdor)
            (aiol boiot (thess wthess othess) lesb))
      (sogr (ach myk ark kypr pamph)
            (ion-att ion att euboi)))
\end{verbatim}
Es sei angemerkt,
dass die so beschriebenen Verhältnisse nicht streng als genetische
Verwandtschaft interpretiert
oder zumindest so verwendet werden müssen.
Auch nach einer Aufspaltung können nah verwandte Dialekte
noch gemeinsame Entwicklungen haben.
So passieren viele Lautveränderungen in allen griechischen Dialekten,
sind aber dennoch nicht urgriechisch.
Passiert also ein Lautwandel bspw. nach der Aufspaltung
von urion. in ion. und att.
trotzdem noch in beiden Dialekten
(dafür gibt es genug Beispiele),
kann man ihn nicht als urion. bezeichnen.
Für diese Fälle mag es sinnvoll sein, auf die Vorsilbe \textit{ur-}
zu verzichten oder z.B. statt von urion. von ion-att.
zu sprechen (wie im Beispielbaum oben).
Eine bessere Trennung von genetischer Verwandtschaft und ``Sprachgruppen''
wäre methodisch vielleicht angebracht.
Die Praxis wird zeigen,
ob hier noch Verbesserungsbedarf besteht.

Der Lautgesetzbaum ist eine Liste von vier möglichen Elementen, nämlich
Funktionen, die einen String als Argument nehmen und einen String zurückgeben
(also Funktionen, die Lautgesetze anwenden),
Symbole, die eine erreichte Sprachstufe markieren,
und zwei Arten von Listen des Typs \tt{(br|sub lang-lst rule-tree)},
deren zweites Element eine Liste von Sprachen ist,
für die sie gelten (bzw. nicht gelten, wenn das erste Element \tt{not} ist),
und deren drittes Element ein weiterer Lautgesetzbaum ist.
\tt{br} (für \emph{branch}) leitet einen neuen Sprachzweig ein,
der durch \tt{rule-tree} definiert ist
(d.h. dass alle Regeln nach einem erfolgreichen \tt{br} ignoriert werden).
\tt{sub} (für \emph{subrules}) funktioniert ähnlich,
macht aber nach Abarbeitung von \tt{rule-tree}
nach der Regel weiter.
Dies impliziert,
dass es sich bei \tt{rule-tree} nicht um einen echten Baum handelt.
\tt{sub} ist nur dazu gedacht eine Liste von Lautgesetzen bedingt anzuwenden.
Im Griechischen ist dies wie gerade erwähnt besonders nützlich,
da zwar alle Dialekte ähnliche Entwicklungen machen,
diese sich aber im einzelnen unterscheiden.
Mit \tt{sub} kann man einfach Regeln für bestimme Dialekte oder
Dialektgruppen definieren, ohne für jeden Dialekt einen eigenen Sprachzweig
haben zu müssen,
und so Sprachkontaktphänomene recht gut beschreiben.

Die Funktion \tt{(make-rules tree path)}
gibt anhand des Lautgesetz\emph{baums} \tt{tree} eine \emph{Liste}
von Lautgesetzen zurück,
die die Sprachentwicklung in \tt{path} (von \tt{make-path} erzeugt, s.o.)
reflektiert.
Von \tt{(apply-rules rules words)} wird dann diese Liste \tt{rules}
auf die Wörter \tt{words} angewandt.

Die Lautgesetzfunktionen werden von der Funktion \tt{(s pattern . subst)}
erzeugt,
die aus einem String \tt{pattern} einen regulären Ausdruck macht
und eine Funktion zurückgibt,
die \tt{irregex-replace/all} auf diesen regulären Ausdruck
und die Ersetzungen in der Liste \tt{subst} anwendet.
Matcht der reguläre Ausdruck einen Teil der Eingabe,
wird diese durch \tt{subst} ersetzt,
dessen Elemente Strings, Zahlen als Indizes der Submatches, sowie
Funktionen, die anhand eines Matches Strings zurückgeben, sind.
\tt{(irregex-replace/all "(foo)(bar)" "foobar" 2 (lambda (m)
    (string-reverse\\ (irregex-match-substring m 1))) "quux")}
z.B. würde den String \tt{"foobar"} durch \tt{"baroofquux"} ersetzen.

Die oben beschriebenen regulären Ausdrücke
sind für die Modellierung von Laugesetzen nicht ausreichend.
Neben anderen Unzulänglichkeiten,
die sich im Laufe der Arbeit herausstellen werden,
gibt es eindeutig Probleme, mit Lautklassen umzugehen,
und noch allgemeiner, Laute durch Zeichen zu kodieren.

Wenn jeder Laut durch genau ein Zeichen bzw. einen Unicode Codepoint kodiert wird,
gibt es bei der Lautkodierung keine Probleme.
Da es allerdings sinnvoll ist,
einige Laute mit mehreren Codepoints zu kodieren,
kann es zu Problemen kommen.
Will man bspw. ein Lautgesetz \tt{(s "b" "p")} formulieren,
so hätte dies die Nebenwirkung,
dass auch \str{bʰ} zu \str{pʰ} würde,
was im allgemeinen Fall unerwünscht ist.
Wenn man Laute wie \bel{bʰ} oder \bel{gʷ} mit einem einzigen Codepoint
repräsentieren würde,
hätte man dieses Problem natürlich zwar nicht,
jedoch bietet Unicode für solche Vorhaben keine definierten Codepoints.
Dafür müsste man die Private Use Area benutzen,
wodurch die Kodierung jedoch von der Schriftart abhängig wird,
eine unschöne Lösung.
Eine Möglichkeit, dieses Problem zu umgehen, ist,
Polygraphen sowohl in den zu transformierenden Wörtern als auch in den
die Lautgesetze beschreibenden regulären Ausdrücken durch
beliebige unbenutzte Codepoints zu ersetzen
und am Ende die umgekehrte Ersetzung durchzuführen,
so dass die nur intern verwendeten Codepoints nirgendwo graphisch auftauchen.
Tatsächlich benutze ich auch in den regulären Ausdrücken
in einigen Fällen eine einfachere Kodierung
(wenn eine gute Alternative vorhanden war, z.B. \str{w} statt \str{u̯}) wie in
\ref{sec:vorbemerkungen} beschrieben
(die Ersetzung der komplexen durch eine einfachere Kodierung in den Wörtern
findet in der Liste der phonologischen Regeln statt,
der umgekehrte Fall in den graphematischen).
In diesen Fällen ist jedoch die einfache Kodierung
im regulären Ausdruck direkt verwendet --
also keine Ersetzung von komplexer durch einfache Kodierung --
aber eine automatische Ersetzung könnte in \tt{(make-regex s)}
leicht implementiert werden.
Der Einfachheit halber und aus Effizienzgründen habe ich jedoch zunächst
darauf verzichtet,
da die problematischen Fälle wenige sind und vorläufig durch vorsichtige
Formulierung umgangen werden können
(der obere Fall also als \tt{(s "b([\textasciicircum{}ʰ])" "p" 1)}).

Bei der Formulierung von Lautgesetzen ist es ferner sinnvoll,
Laute in Lautklassen zusammenzufassen,
um eine konzise und leicht abstrahierende Darstellung zu haben.
Der POSIX-Standard definiert zwar Zeichenklassen wie \tt{[:digit:]}
für Ziffern oder \tt{[:lower:]} für Kleinbuchstaben,
aber die Bedürfnisse, die man als Linguist hat,
sind damit nicht abgedeckt,
da sich zur Laufzeit Zeichenklassen weder neu definieren noch verändern lassen.
Dieses Problem lässt sich vorläufig lösen,
indem die regulären Ausdrücke,
bevor sie von \tt{string->irregex} in eine interne Form kompiliert werden,
noch verändert werden,
so dass die tatsächlich im Quelltext vorkommenden regulären Ausdrücke
erst zu echten regulären Ausdrücken verarbeitet werden,
die von der \tt{irregex}-Engine verstanden werden.
Konkret funktioniert dies so,
dass in einer assoziativen Liste Lautklassennamen (als Symbole)
mit Strings assoziiert werden
und bei der Vorbearbeitung der regulären Ausdrücke
Strings der Form \tt{<lautklassenname>} durch den mit dem Symbol
\tt{lautklassenname} assoziierten String ersetzt werden.
Die Funktion \tt{(set-class key value)} assoziiert den Namen \tt{key}
mit dem String \tt{value} und \tt{(l key)} (kurz für \emph{lookup})
findet den zu \tt{key} passenden String in der assoziativen Liste.

So kann man bspw. mit \tt{(set-class 'kurz-vok "a|e|i|o|u")} und
\tt{(set-class 'lang-vok "ā|ē|ī|ō|ū")} Kurz- und Langvokale definieren,
mit \tt{(set-class 'vok (s+ (l 'kurz-vok) "|" (l 'lang-vok)))} dann die Menge
aller Vokale
(\tt{s+} ist eine Abkürzung für \tt{string-append} zum Verketten von Strings)
und diese in einem Lautgesetz wie \tt{(s "<vok>" "a")} verwenden.

Die Lautklassen sind jedoch im Laufe der Sprachentwicklung
nicht vernünftig veränderbar,
da sie zur Zeit der Auswertung von \tt{s} in den regulären Ausdruck
eingefügt werden (also wenn die Quelltextdatei gelesen wird),
man die Definition aber wohl am ehesten aus dem Lautgesetzbaum heraus
verändern wollen würde
(Lautklassen sollten also dynamisch statt lexikalisch gebunden sein).
Hier wird nur eine neuentwickelte Lautgesetzengine wirklich
Abhilfe schaffen können.

Es gibt jedoch noch ein weiteres Problem.
So kann man nicht ohne weiteres z.B. einen Langvokal durch seinen
entsprechenden Kurzvokal oder einen silbischen durch sein unsilbisches
Pendant ersetzen.
Beispiele für solche Vorgänge gibt es genug und es ist eine Methode notwendig,
mit der man diese Fälle unkompliziert ausdrücken kann.
In einer späteren Version wäre vielleicht eine Ersetzung wie
\tt{(s "<lang-vok>" "<kurz-vok>")} wünschenswert.
In der aktuellen Version können jedoch auch Funktionen,
die auf Submatches angewandt werden, denselben Effekt erzielen.
Als Ersetzungsargumente kann man an \tt{irregex-replace/all}
neben Strings und Zahlen, die das entsprechende Submatch bezeichnen,
auch Funktionen übergeben,
die auf das aktuelle Match angewandt werden und einen String zurückgeben.
Um also z.B. einen Langvokal in ein Kurzvokal zu verwandeln,
braucht man nur eine Funktion,
die auf ein Submatch eine Reihe von regulären Ausdrücken anwendet
und das Ergebnis zurückgibt.
Eine Funktion, die reguläre Ausdrücke auf einen String anwendet,
gibt es ja schon: \tt{apply-rules}.
Diese wird von der Funktion \tt{(match-rulelist rules)} benutzt,
die eine Liste von Ersetzungsfunktionen \tt{rules}
auf ein Submatch anwendet.
Dazu gibt sie eine Funktion zurück,
die einen Submatchindex \tt{i} bindet und eine Funktion zurückgibt,
auf die \tt{irregex-replace/all} angewandt werden kann.
Die Vokalkürzung könnte somit folgendermaßen definiert werden:
\begin{verbatim}
(define kuerzung
  (match-rulelist
    (list (s "ā" "a")
          (s "ē" "e")
          (s "ō" "o")
          (s "ī" "i")
          (s "ū" "u"))))
\end{verbatim}
Ein Lautgesetz, das alle Langvokale kürzt,
kann dann als \tt{(s "(<lang-vok>)" (kuerzung 1))} formuliert werden.

Die schon erwähnten Regeln zur Umwandlung von einer
graphematisch/phono\-logischen in eine phonetische und von
einer phonetischen in eine graphematisch/phono\-logische Darstellung
sind unter ihrem Sprachnamen ebenfalls in
assoziativen Listen abgelegt.

Hiermit ist die Funktionalität von \tt{lga} weitestgehend erläutert.
Eine exakte Beschreibung kann natürlich nur der Quelltext selbst liefern.

\section{Modellierung der Lautentwicklung des Altgriechischen}
\subsection{Vorbemerkungen}
\label{sec:vorbemerkungen}
Vor der Beschreibung der Lautentwicklung noch einige Anmerkungen,
die nicht speziell das Griechische betreffen.
Aufgrund von Fehlern in der Implementation von \tt{irregex}
können in regulären Ausdrücken keine höheren Unicodezeichen
in \tt{[]}-Sets benutzt werden.
Da diese aber notwendig sind, um gewisse Zeichen auszuschließen
-- insbesondere \tt{ʰ} zur Unterscheidung von Aspiraten --
müssen diese aus dem ASCII-Vorrat stammen.
Aus dem Grund wird für die Aspiration \tt{!} statt \tt{ʰ} verwendet.

Für einige komplex kodierte Laute (d.h. mehr als ein Codepoint)
wird wie schon beschrieben in den regulären Ausdrücken eine einfachere
Kodierung benutzt.
Für die Labiovelare wurden im Gegensatz zu den Velaren Großbuchstaben verwendet
(also z.B. \bel{K} statt \bel{kʷ}),
die silbischen Resonanten \bel{r̥l̥m̥n̥} werden ebenfalls
mit den Großbuchstaben \tt{RLMN} ausgedrückt.
Die Halbvokale \bel{i̯u̯} werden mit \tt{yw} und
die Laryngale \bel{h[₁₂₃]} mit \tt{H[123]} bezeichnet.
Akzente werden in der graphematischen Schreibweise
als combining diacritics geschrieben und intern mit \tt{'} und \tt{\textasciitilde}
kodiert.

Als phonologische Regeln für das Uridg.
sind Laryngalumfärbung (\rek{e} neben \rek{h₃} jedoch noch von \rek{o}
verschieden, was allerdings für das Gr. belanglos ist)
und Stimmhaftwerdung von \rek{s} neben stimmhaften Konsonanten beschrieben.
Weitere phonologische Regeln müssen ggf. ergänzt werden.
Lautgesetze wie \textsc{Stang}s Gesetz,
die vor dem Erreichen des Sprachzustandes, der mit \tt{uridg} bezeichnet wird,
durchlaufen wurden,
wurden nicht berücksichtigt und die Rekonstrukte bzw. Transponate,
die als Eingabe dienen, sollten einen entsprechenden Lautstand vorweisen
(um bei \textsc{Stang}s Gesetz zu bleiben wäre also \rek{di̯ēm} nicht \rek{di̯eu̯m} \tt{uridg}).
Insbesondere enthält \tt{uridg} den \textit{þ}-Laut,
dessen Herkunft und Phonetik hier ebenfalls nicht weiter hinterfragt wird.

Im Folgenden werden die Lautgesetze vom Urindogermanischen ins Griechische
in drei Etappen aufgestellt.
Zunächst wird die Lautentwicklung bis ins Urgriechische modelliert,
dann bis in mykenischer Zeit und schließlich in die Dialekte
der alphabetischen Zeit.
Die dazu verwendete Literatur ist im wesentlichen
\cite{bart_langvok},
\cite{bart_frueh},
\cite{bart_hand},
\cite{buck},
\cite{lejeune},
\cite{rix} und
\cite{sihler}.

Eine (relative) Chronologie aller Lautgesetze lässt sich nicht immer
mit Sicherheit aufstellen,
da viele Lautgesetze kaum mit anderen Lautgesetzen interagieren
oder mehrere Entwicklungen denkbar sind,
so dass man gerade in der frühesten Zeit,
in der das Griechische noch nicht belegt ist,
in diesen Fällen wenig mehr als nur raten kann.
Hier werden dann die Lautgesetze entweder mit ähnlichen
anderen Gesetzen oder nach Gefühl einsortiert.

Auch ist die Aufstellung hier keinesfalls als endgültig anzusehen.
Besonders Entwicklungen, die in die spätere dialektale Zeit fallen
wie Digammaschwund und Kontraktionen,
die zum Teil nach Einsetzen der Überlieferung stattfinden,
sind zu unübersichtlich und unklar,
um sie hier angemessen zu modellieren ohne den Rahmen der Arbeit zu sprengen
(Schwierigkeiten werden im Text genannt).
Aber auch über sicherere Entwicklungen lässt sich im Detail fast immer streiten.
Sowohl das Programm \tt{lga} als auch die Formulierung
der Lautgesetze sind somit als vorläufiges Ergebnis oder Gerüst zu sehen,
auf dem weitere Arbeiten aufbauen können.
Insbesondere sind Akzententwicklungen noch völlig unbeachtet geblieben.

Nach der Beschreibung und Diskussion der Lautgesetze folgt eine
formelhafte Darstellung,
 die mehr oder weniger genau den Regeln in \tt{lga} entspricht.
Zu jeder Regel werden einige Beispiele genannt,
die zum Großteil von Rix übernommen,
allerdings nicht erschöpflich sind.
Sie dienen mehr zur Veranschaulichung denn als Belege.
In besonders trivialen Fällen gebe ich daher auch weniger oder
gar keine Beispiele.
Auch in sehr unübersichtlichen Fällen bilden die Beispiele nicht
die ganze Komplexität ab.

Sofern bei Rekonstrukten keine Sprachbezeichnung angegeben ist,
ist irgendeine nicht weiter spezifizierte Vorform gemeint.
Die Sternformen sind auch nicht in allen Fällen als
Rekonstrukte zu verstehen.
Wenn eine Form als uridg. bezeichnet ist,
kann sich dies auch nur auf die einzelnen Morphe beziehen,
ohne dass damit gesagt würde,
dass schon das Uridg. die Form in dieser Zusammensetzung besessen hat.

\subsection{Das Lautsystem des Urindogermanischen}

Das hier angenommene Lautsystem des Uridg. sieht wie folgt aus.
Umstrittene Laute stehen in Klammern.
Teilweise ist nur der Phonemstatus umstritten,
aber obwohl Phoneme für Lautgesetze eigentlich uninteressant sind,
sind die hier aufgeführten Laute natürlich immer noch
eine Abstraktion über die tatsächlichen Laute,
was aber in der Natur der Rekonstruktion liegt
und auch aus pragmatischen Gründen kaum anders gehandhabt werden könnte.

\bigskip
\noindent
Vokale: \rek{a}, \rek{ā}, \rek{e}, \rek{ē}, \rek{o}, \rek{ō},
\rek{i}, (\rek{ī}), \rek{u}, (\rek{ū})

\noindent
Silbische Resonanten: \rek{r̥}, \rek{l̥}, \rek{m̥}, \rek{n̥}

\noindent
Resonanten und Halbvokale: \rek{r}, \rek{l}, \rek{m}, \rek{n}; \rek{i̯}, \rek{u̯}

\noindent
Verschlusslaute:\\
\begin{tabular}{l l l l l}
Labiovelare & \rek{kʷ} & \rek{gʷ} & \rek{gʷʰ} & (\rek{kʷʰ})\\
(Rein)velare & \rek{k} & \rek{g} & \rek{gʰ} & (\rek{kʰ})\\
Palatale & \rek{k̑} & \rek{g̑} & \rek{g̑ʰ} & (\rek{k̑ʰ})\\
Dentale & \rek{t} & \rek{d} & \rek{dʰ} & (\rek{tʰ})\\
Labiale & \rek{p} & \rek{b} & \rek{bʰ} & (\rek{pʰ})\\
\end{tabular}

\noindent
Frikative und sonstige: \rek{s}, \rek{z}, \rek{h₁}, \rek{h₂}, \rek{h₃}, (\rek{þ})

\bigskip
\rek{z} ist Allophon von \rek{s} vor stimmhaften Konsonanten.
Die beiden Laute werden der Einfachheit halber
jedoch in den Regeln und in \tt{lga}
unterschiedslos mit \rek{s} bezeichnet.

Die genaue Artikulation der Laryngale \rek{h₁}, \rek{h₂}, \rek{h₃}
ist nicht klar. Meist werden Frikative vermutet,
neben denen ggf. Sprossvokale eingefügt werden.

Der Laut \rek{þ} ist umstritten,
sowohl seine Existenz als auch seine Artikulation
(vielleicht dentale Affrikata?).

\subsection{Urindogermanisch bis Urgriechisch}

Die Lautveränderung,
die ins Urgr. führen, sind grob gesprochen
die Kentumvertretung der Tektale,
die Entwicklung von \rek{þ},
die Stimmloswerdung der Mediae aspiratae,
jegliche Laryngalentwicklungen,
die Entwicklung der silbischen Resonanten (teilweise aber auch erst dialektal),
einige Okklusivassimilationen,
die Entwicklung von \rek{s}
und die Anfänge diverser Palatalisierungen.

Da in dieser Phase die Lautveränderungen weniger ineinander greifen,
ist eine relative Chronologie oft nicht aufzustellen.

\mypara{Kentum}
Das Griechische ist eine Kentumsprache,
es fallen also die uridg. Velare und Palatale zusammen (Rix §92-94).
Da das Griechische keinerlei Anhaltspunkte dafür bietet,
dass es je einen Unterschied zwischen diesen Lauten gab,
wird der Zusammenfall früh datiert.

\lex{\i{k̑}, \i{g̑}, \i{g̑ʰ}, (\i{k̑ʰ})}{\i{k}, \i{g}, \i{gʰ}, (\i{kʰ})}

uridg. \rek{k̑}: \rek{k̑m̥tóm} > gr. ἑκατόν; ai. \bel{śatáṃ} `hundert'

uridg. \rek{k}: \rek{leu̯kós} > gr. λευκός; ai. \bel{rocás} `leuchtend'

uridg. \rek{g̑}: \rek{h₂ég̑onti} > att. ἄγουσι; av. \bel{azənti} `sie führen'

uridg. \rek{g}: \rek{gerh₂nós} > gr. γερανός; cf. lit. \bel{gervė} `Kranich'

uridg. \rek{g̑ʰ}: zu \rek{u̯eg̑ʰ-/u̯og̑ʰ-}: att. ὄχος `Wagen'; av. \bel{vazāmi} `ich fahre'

uridg. \rek{gʰ}: \rek{h₃migʰ-/h₃moi̯gʰ-}: gr. ὀμίχλη `Nebel'; ai. \bel{meghás} `Wolke'

\mypara{Thorn}
\rek{þ} (genauer Lautwert unklar) tritt nach Velaren und Labiovelaren auf.
Nach Tenues entsprich \rek{t}, nach Media aspirata \rek{dʰ} (Rix §81).

\lex{\i{þ}}{\i{t}}{\{Tenuis\}\_}

uridg. \rek{k̑þ}: \rek{ték̑þō(n)} > gr. τέκτων; ai. \bel{tákṣā} `Zimmermann'

\lex{\i{þ}}{\i{dʰ}}{\{Media asp.\}\_}

uridg. \rek{gʷʰþ}: \rek{ń̥gʷʰþitom} > gr. ἄφθιτον; ai. \bel{ákṣitam} `unvergänglich'

\mypara{Media aspirata > Tenuis aspirata}
Mediae aspiratae werden zu Tenues aspiratae (Rix §94).
Aufgrund der Nähe zum Makedonischen und Phrygischen,
in denen der Reflex vermutlich Media ist,
ist dieser Wandel vielleicht später anzusetzen.
Spätestens im Myk. ist die Stimmlosigkeit belegt.

\lex{\i{bʰ}, \i{dʰ}, \i{gʰ}, \i{gʷʰ}}{\i{pʰ}, \i{tʰ}, \i{kʰ}, \i{kʰʷ}}

uridg. \rek{bʰ}: \rek{bʰéronti} > dor. φέροντι; ai. \bel{bháranti} `sie tragen'

uridg. \rek{dʰ}: zu \rek{dʰeh₁-}: gr. τίθημι; ai. \bel{dadhā́mi} `ich setze/stelle/lege'

uridg. \rek{g̑ʰ}: zu \rek{stei̯g̑ʰ-}: gr. στείχω; got. \bel{steigan} `steigen'

uridg. \rek{gʷʰ}: zu \rek{gʷʰen-}: gr. φόνος `Mord'; ai. \bel{gʰnánti} `sie erschlagen'

\mypara{\rek{(H)i̯-}}
\label{Hy}
Im Gr. hat \rek{i̯-} scheinbar zwei Reflexe,
nämlich Verschärfung zu einem Laut, der mit \rek{di̯-} und \rek{gi̯-}
zusammenfällt und einfaches \rek{i̯-}, das im Myk. noch belegt ist.
Mit der Laryngaltheorie versucht man die beiden Reflexe als \rek{Hi̯-}
und \rek{i̯-} zu unterscheiden.
Welcher Anlaut verschärft wurde,
ist unklar -- nach Rix §68, 80e ist es \rek{Hi̯-} --
und nach dem Datenmaterial kann für beides argumentiert werden,
wenn man aber ὑγιής `gesund' aus \rek{h₂i̯u-gʷih₃-ḗs} `langes Leben habend'
herleiten will, darf man \rek{Hi̯-} > \rek{i̯-} vermuten
(\rek{su-} `gut', das Rix (§97) rekonstruiert und mit ai. \bel{su-} `gut' verbindet,
ist als \rek{h₁su-} anzusetzen und würde εὐ- ergeben).
Auch lautlich plausibel ist ein
früher Zusammenfall von uridg. \rek{i̯-} und \rek{di̯-}
(cf. lat. \bel{Iūlius} > ital. \bel{Giulio} und \bel{diūrnus} > \bel{giorno} `Tag'),
danach Wegfall aller Laryngale (also keine Vokalisierung wie vor \rek{u̯}),
wodurch \rek{Hi̯-} als \rek{i̯-} stehenbleibt,
wie es im Frühmyk. belegt ist.

\lex{\i{i̯-}}{\i{di̯-}}

uridg. \rek{i̯-}: \rek{i̯ugóm} > gr. ζυγόν; ai. \bel{yugám} `Joch'

\newpage
\lex{\i{Hi̯-}}{\i{i̯-}}

uridg. \rek{h₂i̯-}: ὑγιής ist mit
\rek{h₂i̯u-} `langes Leben (→ lange Zeit → Zeit)'
wegen Vollstufe \rek{h₂ei̯u-} in z.B.
hom. αἰεί `immer' < \rek{h₂ei̯u̯esi} `zu allen Zeiten' zu rekonstruieren.

\mypara{Laryngale}
\label{laryngale}
Die Entwicklung der Laryngale folgt Rix §79-85.
Teilweise sind die Lautgesetze weniger explizit als bei Rix formuliert,
um eine lautlich plausiblere Entwicklung zu modellieren.
So wird z.B. intervokalischer Laryngal nach \rek{i} oder \rek{u}
nicht zum Gleitlaut sondern dieser entsteht erst als Hiattilger
nach dem Laryngalschwund.
Entgegen Rix glaube ich nicht an Vokalisierung im Anlaut vor \rek{u} (§79b),
da die Beispiele nicht überzeugen und er immerhin selber ὑδέω als Gegenbeispiel nennt.
Der Schwund der Laryngale dürfte früh passiert sein.
Der Wandel, der hier als \i{H} > \i{V} geschrieben ist,
ist als \i{h₁}, \i{h₂}, \i{h₃} > \i{e}, \i{a}, \i{o} zu verstehen.

\lex{\i{H}}{\i{V}}{\i{C}\_\#} (Rix §85d)

uridg. \rek{-h₁} > gr. -ε N.Du.m.

uridg. \rek{-h₂} > gr. -α N.Pl.n.

\lex{\i{(i, u)H}}{\i{(i, u)V}}{\i{CC}\_\#} (Rix §85b)

uridg. \rek{trí-h₂} > gr. τρία; ai. \bel{trī́} `drei'

\lex{\i{(i, u)H}}{\i{(i̯, u̯)V}}{\i{C}\_\#} (Rix §85b)

uridg. \rek{h₃ókʷ-ih₁} > \rek{okʷi̯e} > hom. ὄσσε `beide Augen'

uridg. \rek{dóru-h₂} > \rek{dóru̯a} > hom. δοῦρα `Speere'

\lex{\i{VH}}{\i{V̄}}{\_(\i{C}, \i{R̥}, \#)} (mit \i{R̥} > \i{R}) (Rix §82b, 85a)

uridg. \rek{eh₁}: zu \rek{dʰeh₁-}: gr. τίθημι `setze/stelle/lege'

uridg. \rek{eh₂}: zu \rek{steh₂-}: dor. ἵστᾱμι `stehe'

uridg. \rek{eh₃}: zu \rek{deh₃-}: gr. δίδωμι `gebe'

uridg. \rek{o/eh₂}: zu \rek{bʰoh₂-}/\rek{bʰeh₂-}: dor. φωνᾱ́ `Stimme' bzw. φᾱτί `spricht'

uridg. \rek{-eh₂} (N.Sg. eh₂-St.) > ion-att. -η, sonst -ᾱ; ai. \bel{-ā}

uridg. \rek{-oh₁} (N.Du. o-St.) > -ω; ai. \bel{-ā}

\lex{\i{H}}{\i{V}}{(\#, \i{C})\_\i{C}} (Rix §80d, 82c)

uridg. \rek{h₂u̯-}: \rek{h₂u̯eh₁ti} > hom. ἄησι; ai. vā́ti `weht'

uridg. \rek{h₁r-}: \rek{h₁regʷos} > gr. ἔρεβος `Finsternis'; ai. \bel{rájas} `Dunst'

uridg. \rek{dʰh₁-tós} > gr. θετός `gesetzt/-stellt/-legt'

uridg. \rek{ph₂tḗr} > gr. πατήρ; ai. \bel{pitā́} `Vater'

uridg. \rek{dh₃-tós} > gr. δοτός `gegeben'

\lex{\i{n̥h₂}}{\i{na}}{\_\#} (Rix §85c)

uridg. \rek{k̑r̥h₂sn̥-h₂} (N.Pl.n.) > hom. κάρηνα; ai.  \bel{śīrṣā́} `Köpfe'

\newpage
\lex{\i{HR̥}}{\i{VR}}{\i{C}\_} (Rix §84γ)

uridg. \rek{h₁n̥}: zu \rek{dʰidʰh₁-n̥t-}: gr. τιθέ-ντ-ος; ai. \bel{dádh-at-as}

uridg. \rek{h₃n̥}: zu \rek{didh₃-n̥t-}: gr. διδό-ντ-ος; ai. \bel{dád-at-as}

\lex{\i{R̥H}}{\i{RV̄}}{\_\i{C}}

uridg. \rek{l̥h₁}: zu \rek{gʷl̥h₁-}: gr. βλητός `geworfen'; ai. \bel{ud-gūr-ṇas} `emporgehoben'

uridg. \rek{l̥h₂}: \rek{tl̥h₂tós} > dor. τλᾱτός; lat. lātus `getrageń

uridg. \rek{r̥h₃}: \rek{str̥h₃tós} > gr. στρωτός; lat. strātus `hingebreitet'

\lex{\i{Ŕ̥H}}{\i{V́RV}}{\_\i{C}}

uridg. \rek{ń̥h₁}: \rek{g̑ń̥h₁tis} > gr. γένεσις `Erzeugung'

uridg. \rek{ŕ̥h₂}: \rek{k̑ŕ̥h₂sn̥h₂} > \rek{kárasna} > hom. κάρηνα `Köpfe'

uridg. \rek{ŕ̥h₃}: zu \rek{ste-stŕ̥h₃-}: aiol. ἐ-στόρο-ται

\lex{\i{HR̥}}{\i{VR}}{\#\_} (Rix §79c)

uridg. \rek{h₁r̥-}: zu \rek{h₁r̥-sk̑e-}: gr. ἔρχεται; ai. ṛcháti `kommt'

uridg. \rek{h₃n̥-}: zu \rek{h₃n̥bʰl̥-}/\rek{h₃nobʰl̥-}: gr. ὀμφαλός, ahd. \bel{nabulo} `Nabel'

\lex{\i{H}}{\i{Ø}}

\mypara{\rek{u̯eu̯} > \bel{u̯ei̯}}
\rek{u̯eu̯} wird zu \rek{u̯ei̯} dissimiliert,
wie εἶπον < \rek{e-u̯e-u̯kʷ-om} (vgl. ai. \bel{avocam}) lehrt.
Dies muss ferner vor \ref{labvel-vel} passieren,
wie εἶπον ebenfalls zeigt.

\lex{\i{u̯}}{\i{i̯}}{\i{u̯e}\_}

\mypara{Labiovelar > Velar}
\label{labvel-vel}
Neben \rek{ū̆}, \rek{u̯} und vor \rek{i̯}
fallen die Labiovelare mit den Velaren zusammen (Rix §97).

\lex{\i{kʷ}, \i{gʷ}, \i{kʷʰ}}{\i{k}, \i{g}, \i{kʰ}}{(\i{u}, \i{ū}, \i{u̯})\_; \_(\i{u}, \i{ū}, \i{}u̯, \i{i̯})}

uridg. \rek{u̯kʷ}: zu \rek{-kʷólos}: gr. βουκόλος `Kuhhirte' vs. αἰπόλος `Ziegenhirte' (zu Labiovelar > Labial s. \ref{labvel})

uridg. \rek{ugʷ}: zu \rek{-gʷih₃-}: gr. ὑγιής `gesund' vs. βίος `Leben'

uridg. \rek{gʷʰu}: zu \rek{h₁ln̥gʷʰ-}: gr. ἐλαχύς vs. ἐλαφρός `schnell'

uridg. \rek{kʷi̯}: zu \rek{h₃okʷ-}: hom. ὄσσε < \rek{okʷ-i̯e} `beide Augen';
lit. akis `Auge'

\mypara{Silbische Resonanten}
Die Entwicklung der silbischen Resonanten folgt Rix §75 und §76.
Für das Urgr. wird ein Sprossvokal \rek{ə} angesetzt,
welcher sich dialektal zu \bel{a} oder \bel{o} entwickelt (s. \ref{a-o}).

Hierunter fallen auch die Fälle, wo ursprünglich Laryngal zwischen
Resonant und Vokal stand.
Diese sind bei Rix gesondert behandelt,
was aber unnötig ist, wenn man den Laryngalschwund einfach vorher ansetzt.

Genauere Datierung unklar. Nach Laryngalschwund.
Eine etwas spätere Datierung wäre auch möglich,
aber vor \ref{pal} wegen z.B. \rek{gʷami̯ō} > βαίνω.

\lex{\i{N̥}}{\i{aN}}{(\i{V}, \i{V̯})}

\rek{m̥V}: zu uridg. \rek{k̑m̥h₂-}: gr. ἔκαμον `mühte sich'

\rek{n̥i̯}: uridg. \rek{tek̑þn̥ih₂} > \rek{tektani̯a} > gr. τέκταινα `Erbauerin'

\lex{\i{N̥}}{\i{ə}}

\rek{n̥}: uridg. \rek{tn̥tós} > gr. τατός; ai. \bel{tatás}, lat. \bel{tentus} `gespannt'

\rek{m̥}: uridg. \rek{dék̑m̥} > att. δέκα, ark. δέκο; lat. \bel{decem} `zehn'

\lex{\i{L̥}}{\i{əL̥}}{(\#, \i{V}, \i{V̯})}

uridg. \rek{Hi̯ḗkʷr̥} > gr. ἧπαρ `Leber'

\rek{ekstr̥-i̯ō} > \rek{ekstari̯ō} > gr. ἐχθαίρω `hasse (den Feind)'

zu uridg. \rek{gʷl̥h₁-}: \rek{egʷalon} > gr. ἔβαλον `sie warfen'

\lex{\i{L̥}}{\i{Lə}}

uridg. \rek{str̥tós} > att. στρατός, lesb. boiot. στροτός `Truppe, Heer'

uridg. \rek{plth₂ús} > att. πλατύς; ai. \bel{pṛthús} `breit'

\mypara{Nasal vor Okklusiv}
Vor Okklusiven wird der homorgane Nasal realisiert (Rix §78).
Gegen Rix jedoch nicht \rek{ms} > \rek{ns} wegen
ἔνειμα < \rek{enemsa}.
Hier kann \bel{m} zwar analog wieder eingeführt worden sein,
aber ohne ein Beispiel, das bei Rix fehlt, ist die Regel zunächst unnötig.
Die Nasalassimilation passiert auch später noch und
dürfte tatsächlich wohl über lange Zeit eine synchrone phonologische Regel sein.

\lex{\i{N}}{\i{m}}{\{Labial\}}

\lex{\i{N}}{\i{n}}{\{Dental, Velar, Labiovelar\}}

\mypara{\rek{mi̯} > \rek{ni̯}}
Rix §77.
Datierung unklar.
Vielleicht nach \rek{m̥} > \rek{am}, aber \rek{m̥} > \rek{n̥} \textbackslash\ \rek{i̯}
ist ebenfalls als zusätzliche Regel denkbar.
Vielleicht auch im Zuge der Palatalisierung (s. \ref{pal}).

\lex{\i{m}}{\i{n}}{\i{i̯}}

zu uridg. \rek{gʷem-}: \rek{gʷm̥-i̯ō} > \rek{gʷami̯ō} > \rek{gʷani̯ō} > gr. βαίνω `ich komme'; lat. \bel{veniō}

\mypara{Benachbarte Vokale nach Laryngalschwund}
Kontraktion von \rek{e}, \rek{a}, \rek{o},
zwischen denen Laryngal geschwunden ist (Rix §81).
Hier entsteht zuerst der Intonationsunterschied zwischen Akut und Zirkumflex.
Trägt der erste Vokal den Akzent, ist das Kontraktionsprodukt zirkumflektiert.
Trägt ihn der zweite, ist es akutiert.
Kontrahierte \rek{ee}, \rek{oo} fallen mit den alten (offenen) Langvokalen zusammen;
die geschlossenen Langvokale entstehen erst bei \ref{ED1}.
Datierung unklar.

\lex{\i{VV}}{\i{V̄}} (\i{V} = \i{a}, \i{e}, \i{o})

uridg. \rek{-eh₂ei̯} D.Sg. eh₂-St. > \rek{-aai̯} > ion-att. -ῃ, sonst -ᾳ

\bigskip\noindent
Nach Vokal werden \rek{i}, \rek{u} zu Halbvokalen (es entstehen also Diphthonge).

Im umgekehrten Fall entsteht ein Gleitlaut zwischen den beiden Lauten.
Dieser ist streng genommen nicht gesichert, da er nicht geschrieben wird,
da aber die Silbengrenze erhalten bleibt,
ist er phonetisch äußerst wahrscheinlich (und sollte vielleicht vordatiert werden).

\lex{\i{i}, \i{u}}{\i{i̯}, \i{u̯}}{\i{V}\_}

uridg. \rek{pléh₁istom} > gr. πλεῖστον `am meisten'

\lex{\i{i}, \i{u}}{\i{ii̯}, \i{uu̯}}{\_\i{V}}

uridg. \rek{tri-h₂} > gr. τρία `drei'

zu uridg. \rek{h₃bʰruH-}: gr. ὀφρύος `Braue'

\mypara{Okklusivassimilation}
\label{okklass}
Okklusive werden an folgende Okklusive in
Aspiration und Stimmhafigkeit assimiliert (Rix §106a).

Hier dürfte es sich wohl eigentlich um eine phonologische Regel handeln.

\lex{\{Okklusiv\}}{\{Media\}}{\{Media\}}

gr. ἐμπλέγ-δην zu ἐμπλέκω

gr. κρύβ-δην zu κρύφα

\lex{\{Okklusiv\}}{\{Tenuis\}}{\{Tenuis\}}

gr. ἑκ-τός zu ἔχω

gr. ληπ-τός zu λαμβάνω

\lex{\{Okklusiv\}}{\{Aspirata\}}{\{Aspirata\}}

gr. ἐδιώχ-θην zu διώκω

gr. ἐλήφ-θην zu λαμβάνω

\mypara{\rek{Ku̯} > \rek{Kʷ(:)}}
\label{kw}
Velar + \rek{u̯} wird zum Labiovelar, zwischen Vokalen als Geminate (Rix §104),
um das Silbengewicht zu erhalten (vielleicht nur nach Kurzvokal?).

\lex{\i{Ku̯}}{\i{Kʷ:}}{\i{V}\_\i{V}}

uridg. \rek{h₁ék̑u̯os} > \rek{ékʷkʷos} >> gr. ἵππος (mit rätselhaftem Anlaut);
ai. \bel{áśvas} `Pferd'

\lex{\i{Ku̯}}{\i{Kʷ}}

uridg. zu \rek{g̑ʰu̯eh₁r-}: hom. θήρ, aiol. φήρ; lit. žvėris `wildes Tier'


\mypara{Beseitigung von Geminaten}
Das Gr. muss eine Zeit lang Geminaten vermieden haben.
Entweder generell wie das Uridg.
-- dann müsste diese Regel vor \ref{kw} kommen
und dort die Gemination nur nach Kurzvokal passieren --
oder nur im Silbenanlaut (also nur nicht nach Kurzvokal)
wie αἰπόλος < \rek{ai̯kʷolos} < \rek{ai̯g-kʷolos} zeigt.
Weitere Beispiele könnten die Verhältnisse klarer machen.

Datierung unklar,
vermutlich nach \ref{kw}.

\lex{\i{C:}}{\i{C}}{.\_}

\rek{ai̯g-kʷólos} > \rek{ai̯kʷkʷólos} > \rek{ai̯kʷólos} > gr. αἰπόλος `Ziegenhirte'
zu gr. αἴξ, αἰγός `Ziege'


\mypara{Okklusive neben \rek{s}}
\label{okklu-s}

Vor \rek{s} werden alle Okklusive zu Tenues,
wobei gleichzeitig Stimmhaftigkeit und Aspiration auf einen evtl. folgenden
Okklusiv übertragen werden (Rix §105).

Da die Formulierung etwas komplex ist,
modelliere ich es in zwei Schritten:
zuerst Assimilation, dann Tenuiswerdung.

Vor \ref{CsC>sC} zu datieren.

\lex{\{Tenuis\}}{\{Media\}}{\{Media\}s\_}

\rek{mig-skō} > \rek{migsgō} > \rek{miksgō} > gr. μίσγω `ich vermische'

\lex{\{Tenuis\}}{\{Aspirata\}}{\{Aspirata\}\i{s}\_}

\rek{pʰatʰ-skō} > \rek{pʰatʰskʰō} > \rek{pʰatskʰō} > gr. πάσχω `ich erleide'

\lex{\{Okklusiv\}}{\{Tenuis\}}{\_\i{s}}

s.o.

\rek{ai̯g-si} > αἰξί `den Ziegen'

\rek{tʰrikʰ-si} > θριξί `den Haaren'


\mypara{CsC > sC}
\label{CsC>sC}
Steht \rek{s} zwischen zwei gleichen Konsonanten, schwindet der erste (Rix §104).
Dabei werden Stimmhaftigkeit und Aspiration ignoriert,
da vor \rek{s} nach \ref{okklu-s} nur Tenuis stehen kann.

\lex{\i{C₁}}{\i{Ø}}{\_\i{C₂}} (\i{C₁} = \i{C₂})

s. \ref{okklu-s}

\rek{dikskos} > δίσκος `Scheibe'

\subsection*{Urgriechisch I}
Das hier erreichte Sprachstadium ist eine Vorstufe des Urgriechischen,
das als Ausgangssprache bei der Lautwandelsimulation sehr nützlich ist,
da man nicht alle Vorformen auf das Urindogermanische zurückführen will,
das Urgriechische andererseits in der Entwicklung oftmals
schon zu weit fortgeschritten ist.
Was Rix als urgr. bezeichnet, entspricht oftmals eher meinem urgr. I,
manchmal sogar noch früheren Sprachstufen.

\mypara{\rek{s} > \rek{h}}
\label{sh}
\rek{s} wird nach Vokal oder im Anlaut und vor Vokal, Resonant oder \rek{u̯},
nach unbetonter Silbe zwischen Resonant oder \rek{u̯} und Vokal
und zwischen Nasal und Vokal zu \rek{h} verhaucht (Rix §86, 89).
Dieses \rek{h} schwindet in fast allen Fällen später ganz
(teilweise mit Ersatzdehung oder Gemination, s. \ref{ED1}, \ref{gem}).

In einigen Fällen ist \rek{s} auf unerklärte Weise erhalten,
teilweise beim selben Wort:
σῦς neben ὗς `Schwein', σμικρός neben μικρός `klein'.

Der Wandel ist vor \textsc{Osthoff} (\ref{osthoff}) zu datieren.

\lex{\i{s}}{\i{h}}{(\#, \i{V})\_(\i{V}, \i{R}, \i{u̯}}

\rek{\#sV}: uridg. \rek{septḿ̥} > gr. ἑπτά; ai. \bel{saptá} `sieben'

\rek{\#sr}: zu uridg. \rek{sreu̯-}: gr. ῥέω; ai. \bel{srávati} `es fließt'

\rek{\#sn}: zu uridg. \rek{snei̯gʷʰ-}: gr. νείφει; ahd. \bel{snīwit} `es schneit'

\rek{\#su̯}: uridg. \rek{su̯ekurós} > gr. ἑκυρός; ai. \bel{svaśurás} `Schwiegervater'

\rek{VsV}: uridg. \rek{g̑enh₁esos} > hom. γένεος; ai. \bel{janasas} G.Sg. `Geburt/Erzeugung'

\rek{Vsr}: \rek{kʰésras} > ion-att. χεῖρας, lesb. χέρρας A.Pl. `Hand'

\rek{Vsn}: \rek{selásnā} > dor. σελᾱ́νᾱ, lesb. σελάννᾱ `Mond' (Anlaut unklar)

\rek{Vsu̯}: \rek{nasu̯ós} > lak. νᾱϜός, lesb. ναῦος `Tempel, Gotteshaus'

\lex{\i{s}}{\i{h}}{(\i{R}, \i{u̯})\_\i{V}} (nach unbetonter Silbe)

\rek{au̯sṓs} > hom. ἠώς, dor. ἀϜώς, lesb. αὔως `Morgenröte'

\rek{korsā́} > att. κουρᾱ́ `das Abschneiden'

\lex{\i{s}}{\i{h}}{\i{N}\_\i{V}}

\rek{kʰánsas} > att. χῆνας, boiot. χᾶνας A.Pl. `Gans'

\mypara{\rek{s} neben \rek{i̯}}
Nach Rix §89g ist \rek{s} neben \rek{i̯} über \rek{h} assimilert worden.
Die Beispiele belegen jedoch nur eine Assimilation bei \rek{si̯}
und der Weg über \rek{h} scheint nur aus systematischen Gründen gewählt zu sein.

\rek{si̯} wird intervokalisch zu \rek{i̯i̯}
und anlautend möglicherweise zu \rek{i̯}.
Für den Anlaut is gr. ὑμήν `dünne Haut, Membran' das einzige Beipiel,
das mit ai. \bel{syū́man} `Band, Zügel' zusammengestell wird.
Da aber sowohl \rek{s} als auch \rek{i} auch alleine im Anlaut zu \bel{h} werden und anlautendes \rek{u} ohnehin immer aspiriert ist (s. \ref{hu}),
ist die Zwischenstufe \rek{i̯} zwischen \rek{si̯-} und \rek{h-} hypothetisch.

\rek{i̯s} kommt intervokalisch im L.Pl. \rek{-oi̯si} vor,
der myk. <-o-i> geschrieben wird.
Die Lautung ist unklar -- vielleicht \rek{-oi̯i̯i} oder \rek{-oi̯hi} --
da später das \rek{s} analogisch restituiert wurde.

\lex{\i{si̯}}{\i{i̯i̯}}{\i{V}\_\i{V}}

\rek{gelós-i̯os} > gr. γελοῖος `lustig, lächerlich'

\lex{\i{si̯}}{\i{i̯}}{\#\_\i{V}}

s.o.

\lex{? \i{i̯s}}{\i{i̯i̯}}{\i{V}\_\i{V}}

s.o.

\mypara{\textsc{Osthoff}}
\label{osthoff}
Die klassische Formulierung von \textsc{Osthoff}s Gesetz ist
Kürzung von Langvokalen vor Resonant oder Halbvokal und Konsonant
(Rix §58, 64).
\textsc{Osthoff}s Gesetz ist jedoch umstritten und es gibt mindestens
ein sicheres Gegenbeispiel,
das vermutlich auf eine andere Syllabifizierung deutet.
Dann wäre die Formulierung aber einfach die,
dass Langvokal vor tautosyllabischem Resonanten oder Halbvokal gekürzt wird
(im Auslaut ist die antevokalische Sandhivariante verallgemeinert,
so dass Langvokale erhalten bleiben).
Für eine detaillierte Untersuchung s. \textcite{simkin}.

In att. G.Sg. μηνός < \rek{mēnsós} ist der Langvokal geblieben
(Ersatzdehnung hätte geschlossenes \bel{ē} <ει> ergeben).
Die beste Lösung ist die s-Verhauchung (\ref{sh})
vor \textsc{Osthoff} zu datieren
und eine Syllabifizierung \rek{mē.nhós} anzunehmen,
bei der das Gesetz nicht gilt.
In welchen Fällen genau eine andere Syllabifizierung angenommen werden sollte,
werde ich hier nicht detailliert behandeln.
Simkin nimmt neben \rek{nh} auch noch \rek{nm} und \rek{u̯i̯} an,
aber ich werde mich vorerst auf \rek{nh} beschränken.

Das Lautgesetz muss vor \ref{konsausl} datiert werden
wie z.B. hom. ἔσταν < \rek{estānt} < \rek{e-steh₂-n̥t} zeigt.

\lex{\i{V̄}}{\i{V}}{\_(\i{R}, \i{}V̯).} (außer im Auslaut)

uridg. \rek{-ōi̯s} I.Pl. o-St. > gr. -οις; ai. \bel{-āis}

gr. -θεντ-, Stamm des Partizips Aor.Pass. auf -θη-. 

uridg. \rek{e-steh₂-n̥t} > \rek{estānt} > hom. ἔσταν `sie stellten sich'

\mypara{Konsonanten im Auslaut}
\label{konsausl}
Wortauslautende Okklusive gehen verloren (Rix §100).

Dies muss nach der Vokalisierung der silbischen Liquiden passiert sein,
da ὑπόδρα < \rek{upodrak} < \rek{upodr̥k̑} den auslautenden Konsonanten
voraussetzt.
Außerdem nach \textsc{Osthoff} (\ref{osthoff}) wie dort gezeigt.

\lex{\{Okklusiv\+\}}{\i{Ø}}{\_\#}

uridg. \rek{-t}: uridg. \rek{e-bʰeret} > gr. ἔφερε; ai. \bel{abharat} `er trug'

uridg. \rek{-k̑}: \rek{upo-dr̥k̑} > \rek{upodrak} > ὑπόδρα `von unten guckend'

\bigskip\noindent
Auslautendes \rek{m} erscheint als \bel{n} (Rix §77).

Wann dies passierte, ist vollkommen unklar.
Es wird hier nur zusammen mit dem obigen Gesetz gruppiert,
da beide den Auslaut betreffen.

\lex{\i{m}}{\i{n}}{\_\#}

uridg. \rek{-om} A.Sg. o-St: gr. -ον, ai. \bel{-am}, lat. \bel{-om} > \bel{-um}


%TODO: geminaten im anlaut?
\mypara{Palatalisierung}
\label{pal}
Die Anfänge der Palatalisierungen liegen wohl in urgr. Zeit
und die genaue Entwicklung in den Dialekten ist kompliziert.
Zu diesem Thema siehe \cite{Allen}, \cite[140ff.]{bart_hand} und Rix §102, 103.

Linear B hat zwei Reihen für die Schreibung der späteren Zischlaute,
deren Konsonanten mit <z> und <s> transliteriert werden.
Welche Laute mit diesen Zeichen bezeichnet wurden, ist unklar.
Oft wird <s> als einfacher Sibilant und <z> als dentale Affrikate verstanden,
jedoch ist es sehr gut möglich,
dass es sich tatsächlich um einen Unterschied der Palatalität handelt
und beide Zeichenreihen sowohl Sibilanten als auch Affrikaten bezeichnen können.
Die Z-Reihe würde dann einen palataleren Laut als die S-Reihe bezeichnen
(<z> also [\bel{tš}] und [\bel{dž}], <s> [\bel{s}], [\bel{z}] und [\bel{ts}]).
Mit dieser Annahme lassen sich die mykenischen Schreibungen
und die Lautentwicklung fast zufriedenstellend erklären.
Noch besser ist vielleicht die Annahme,
dass es sich bei der Z-Reihe um eine Art präpalatalen Plosiv
und bei der S-Reihe um einen Sibilanten oder eine Affrikate handelt
(<z> also [\bel{č}], [\bel{ǰ}], <s> [\bel{s}], [\bel{z}], [\bel{ts}], [\bel{tš}]).

Dentale Tenuis (aspirata) + \rek{i̯} im Anlaut ergibt in allen Dialekten \bel{s},
das urgr. sein könnte oder erst mit \ref{ts} im Silbenanlaut zu \bel{s} wird.
Im Inlaut wird hier \rek{ts} angesetzt, das mit altem \rek{ts} zusammenfällt
und sich dann dialektal unterschiedlich entwickelt (s. \ref{ts}).
In Linear B wird für diesen Laut die S-Reihe benutzt,
weswegen eine palatale Qualität sehr unwahrscheinlich ist.

\lex{(\i{t}, \i{tʰ})\i{i̯}}{\i{s}}{\#\_}

zu uridg. \rek{ti̯egʷ-}: gr. σέβονται `sie sind erfürchtig'; ai. \bel{tyajante} `sie verlassen'

\lex{(\i{t}, \i{tʰ})\i{i̯}}{\i{ts}}{\_}

\rek{toti̯os}: ion-att. τόσος, lesb. τόσσος `so groß'; cf. boiot. ὁπόττος, kret. ὀπόττος `so groß wie'

\rek{panti̯a} > \rek{pantsa} > \rek{pansa} (s. \ref{ts}) >
ark. πάνσα, ion-att. πᾶσα, lesb. παῖσα `jede'

uridg. \rek{médʰi̯os} > ion-att. μέσος, lesb. μέσσος, boiot. kret. μέττος `in der Mitte'

%TODO: klären
\bigskip\noindent
Dieselbe Lautkombination an morphologisch transparenter Stelle
ergibt einen anderen Laut,
der in alphabetischer Zeit mit dem Resultat der Palatalisierung der
Velare zusammenfällt,
in Linear B aber mit der S-Reihe geschrieben wird.
Vielleicht wurde an dieser Stelle das \rek{i̯} restituiert
und die Gruppe \rek{tsi̯} nachmyk. weiterpalatalisiert,
so dass sie mit Resultat der palatalisierten Velare zusammenfiel.
Möglicherweise handelt es sich hier aber auch um denselben Laut,
den ich unten für die Substratwörter ansetze,
da Schreibung und Entwicklung mit diesem völlig übereinstimmen.

Stimmhafte Dentale und Velare fallen vor \rek{i̯} in einem Laut zusammen,
der in Linear B mit der Z-Reihe und im Alphabetgriechischen mit <ζ>
geschrieben wird.
Der Laut dürfte im Urgr. entweder eine stimmhafte Affrikate oder ein
(vermutlich geminierter) stimmhafter präpalataler Plosiv gewesen sein
(dafür schreibe ich hier abstrahierend <ǰǰ>).
Dass hier Dentale und Velare zusammenfallen ist eigenartig.
Später wird der Laut zu \rek{dz} (s. \ref{dz}).

\lex{(\i{d}, \i{g})\i{i̯}}{\i{ǰǰ}}

uridg. \rek{di̯ḗu̯s} > gr. ζεύς; ai. \bel{dyáus} `Himmel(sgott)'

zu ion-att. ἁρπαγή `Raub': ἁρπάζω `ich raube'

zu uridg. \rek{gʷieh₃-}: hom. ζωή `Leben'

uridg. \rek{i̯ugóm}: gr. ζυγόν; ai. \bel{yugám} `Joch' (s. \ref{Hy})

\bigskip\noindent
Velare Tenuis (aspirata) + \rek{i̯} scheint eine spätere Palatalisierung
als die der Dentale zu sein.
Das Ergebnis wird in Linear B ebenfalls mit der Z-Reihe geschrieben,
was eine stimmlose Version des obigen Lautes nahelegt,
den ich mit <čč> bezeichne.
Später wird dieser Laut (vermutlich über so etwas wie \rek{tš})
zu \bel{ss} bzw. \bel{tt} (s. \ref{tš})

\lex{(\i{k}, \i{kʰ})\i{i̯}}{\i{čč}}

\rek{ki̯-ā́meron} > att. τήμερον, ion. σήμερον, dor. σᾱ́μερον `heute'

\rek{pʰuláki̯ō} > att. φύλαττω, ion. dor. φυλάσσω `ich behüte' zu φύλαξ `Wächter'

\bigskip\noindent
Ein anderer Laut, der in Substratwörtern vorkommt,
fällt im Alphabetgriechischen völlig mit diesem \rek{čč} zusammen,
wird aber im Linear B mit der S-Reihe geschrieben.
Da zwischen \rek{ts} und dem zuerst vorgeschlagenen \rek{tš} (statt \rek{čč})
als Ergebnis
der Velarpalatalisation jedoch wenig Spielraum für eine weitere Affrikate bleibt,
die zwar später mit \rek{tš} zusammenfiele, aber dental genug wäre,
um in Linear B mit der S-Reihe geschrieben zu werden,
ist die Bestimmung der Laute der Z-Reihe als irgendwie palatale Plosive
vielleicht die beste Lösung.
Dann könnte man den Substratlaut als \rek{tš} ansetzen
und eine spätere Entwicklung \rek{ǰǰ}, \rek{čč} > \rek{dž}, \rek{tš} annehmen.
Heteromorphemisches \rek{ti̯} könnte sich ebenfalls zu \rek{tš} entwickelt
und somit die Möglichkeit der Einordnung des Substratlautes in das
Lautsystem überhaupt erst ermöglicht haben.

Labial + \rek{i̯} ergibt \bel{pt} in allen Dialekten,
muss jedoch eine Vorstufe urgr. \rek{pč}
(ohne eine genaue phonetische Interpretation nahelegen zu wollen) gehabt haben,
da das Myk. ein eigenes Zeichen für diesen Laut kennt,
der erst später auch für ererbtes \bel{pt} verwendet wird (s. \ref{pt}).

\lex{\{Labial\}\i{i̯}}{\i{pč}}

\rek{klepi̯ō} > gr. κλέπτω (cf. lat. \bel{clepō}) `ich stehle'

\rek{tʰapʰi̯ō} > gr. θάπτω `ich begrabe' (cf. τάφος `Grab')

\bigskip\noindent
Die Resonanten \rek{l}, \rek{r}, \rek{n}
(\rek{m} in dieser Position schon mit \rek{n} zusammengefallen)
werden vor \rek{i̯} zu palatalisierten Geminaten.
Ob hier Anlaut oder Silbengewicht eine Rolle spielen, ist noch zu überprüfen.
Später werden die vorangehenden Vokale diphthongiert bzw. gelängt oder
die Geminaten entpalatalisiert (s. \ref{palgem}).

\lex{(\i{l}, \i{r}, \i{n})\i{i̯}}{(\i{ĺĺ}, \i{ŕŕ}, \i{ńń})}

\rek{katʰariō} > ion-att. καθαίρω `ich reinige'

\rek{gʷami̯ō} > ion-att. βαίνω `ich komme'

\rek{áli̯os} > kypr. αἴλος sonst ἄλλος `ein anderer'

\rek{pʰtʰeri̯ō} > ion-att. φθείρω, lesb. φθέρρω `ich zerstöre'

\rek{krini̯ō} > ion-att. κρῑ́νω, lesb. κρίννω `ich entscheide'

\mypara{\rek{dl} > \rek{ll}}
Rix §104. Datierung unklar.

\lex{\i{d}}{\i{l}}{\_\i{l}}

uridg. \rek{sed-leh₂}: lak. ἕλλᾱ (cf. got. \bel{sit-ls} mit \rek{-los} im Suffix)


\mypara{Metathese von \rek{t} neben Okklusiven}
Nach Rix (§106b) werden \rek{tk} und \rek{tp} metathetiert.
Für \rek{tp} > \rek{pt} gibt er als Beispiel \rek{kʷid-pe} > τίπτε,
das aber wohl besser als \rek{kʷid-kʷe} > \rek{kʷikʷte} > τίπτε zu erklären ist
(\cite{lillo}).
Will man den in einigen Wörtern rätselhaften Anlaut
\bel{pt-} neben \bel{p-}
als Sandhi aus \rek{-t\#\#p-} erklären,
so wäre auch eine Metathese neben Labialen anzunehmen.
So kann man die allgemeine Regel aufstellen, dass \bel{t} + Tenuis
metathetiert wird (keine Beispiele für Media und Aspirata).

Datierung unklar.

\lex{\i{t}(\i{k}, \i{kʷ}, \i{p})}{(\i{k}, \i{kʷ}, \i{p})\i{t}}

\rek{tk}: \rek{ti-tk-ō} zur Wurzel uridg. \rek{tek̑-} `gebären' > τίκτω `gebiert'

\rek{tkʷ}: \rek{kʷid-kʷe} > \rek{kʷikʷte} > τίπτε `warum denn?'

? \rek{tp}: \rek{-t\#\#pólis} > πτόλις

\mypara{Nasal vor Liquida}
Zwischen Nasal und Liquida wird die homorgane Media eingeschoben.
Im Anlaut ist solch eine Lautverbindung unmöglich
und der Nasal wird stattdessen schlicht zur Media (Rix §78).

Die Datierung hier ist noch unsicher.
Sogar die Einordnung als urgr. ist nur vorläufig.

\lex{\i{NR}}{\i{DR}}{\#\_}

uridg. \rek{mr̥tós} > aiol. βροτός `Sterblicher'; ai. \bel{mṛtás} `tod'

\rek{ml̥h₃sk̑oh₂} > βλώσκω `ich komme' zu μολεῖν `kommen'

\lex{\i{NR}}{\i{NDR}}

uridg. \rek{h₂n̥rós} > \rek{anrós} > ἀνδρός G.Sg. zu ἀνήρ `Mann'

hom. μέ-μβλεται `er ist besorgt' zu μέλω `ich sorge mich'

\mypara{\rek{u-} > \bel{hu-}}
\label{hu}
Im Anlaut ist \rek{u} immer behaucht.
Da alle (nicht psilotischen, s. \ref{psilose}) Dialekte,
dieses Phänomen zeigen, setze ich es als urgr. an.
Dass es nach \textsc{Graßmann} (s. \ref{grassmann}) zu datieren ist,
wie Lejeune (§320) wegen ὑφαίνω (cf. ai. \bel{ubhnā́ti}) behauptet,
ist nicht zwingend,
da bei \textsc{Graßmann} der erste Laut deaspiriert wird
und ein bloßes \rek{u} im Anlaut im Gr. nicht vorkommt.

\lex{\i{u}}{\i{hu}}{\#\_}

uridg. \rek{upo} > gr. ὑπό; ai. \bel{úpa} `unter'

\subsection*{Das Lautsystem des Urgriechischen}
Das urgr. Lautsystem stellt sich nach den obigen Lautgesetzen nun wie folgt dar:

\bigskip
\noindent
Vokale: \rek{a}, \rek{ā}, \rek{e}, \rek{ē}, \rek{o}, \rek{ō},
\rek{i}, \rek{ī}, \rek{u}, \rek{ū}, \rek{ə}

\noindent
Resonanten und Halbvokale: \rek{r}, \rek{ŕŕ}, \rek{l}, \rek{ĺĺ},
\rek{m}, \rek{n}, \rek{ńń}; \rek{i̯}, \rek{u̯}

\noindent
Verschlusslaute:\\
\begin{tabular}{l l l l}
Labiovelare & \rek{kʷ} & \rek{gʷ} & \rek{kʷʰ}\\
Velare & \rek{k} & \rek{g} & \rek{kʰ}\\
Dentale & \rek{t} & \rek{d} & \rek{tʰ}\\
Labiale & \rek{p} & \rek{b} & \rek{pʰ}\\
\end{tabular}

\noindent
Palatale: \rek{ǰǰ}, \rek{čč}, \rek{pč}

\noindent
Frikative und Affrikaten: \rek{h}, \rek{s}, \rek{z}, \rek{ts}, \rek{tš}

\subsection{Urgriechisch bis mykenische Zeit}

In diesem Abschnitt wird die Lautentwicklung bis zur Zeit,
in der die mykenische Überlieferung endet, modelliert.
Am Anfang dieser Zeit steht die erste dialektale Aufspaltung.
Da außer dem recht einheitlichen myk. Dialekt keine weiteren aus dieser
Zeit überliefert sind,
wird der Lautstand des Myk.
-- sofern es keinen Grund gibt, etwas anderes anzunehmen --
als repräsentativ für das gesamte gr. Sprachgebiet gesehen.

Die dialektale Gliederung ist im wesentlichen aus \cite{bart_frueh}
Tab. B genommen und sieht in Scheme wie folgt aus
(es wird jedoch nicht auf alle Sprachen auch eingegangen):
\begin{verbatim}
(urgr1 (urgr (nwgr (dor (nwdor phok lokr aitol)
                        (sardor kor meg oarg)
                        (sdor el lak inseldor warg kret))
                   (aiol boiot (thess wthess othess) lesb))
             (sogr (ach myk ark kypr pamph)
                   (ion-att (ostion hom) inselion euboi att))))
\end{verbatim}

\mypara{\rek{ti} > \bel{si}}
Dieser Lautwandel wird als grundlegende Unterscheidung zwischen Südost-
und Nordwestgriechisch gesehen (Rix §101),
ist aber weniger Lautgesetz als einem lieb wäre.
Auch nwgr. Dialekte haben \bel{si} in \emph{ti}-Abstrakta (z.B. βάσις vs. ai. \bel{gátis} `Gang')
und das Myk. geht in z.B. \bel{ko-ri-si-o} weiter als andere Dialekte,
die alle Κορίνθιοι zeigen (\cite[144]{bart_hand}).
In dieser Darstellung wird der Lautwandel wegen z.B. ἔστι, κτίσις
nach \rek{s} und \rek{k} ausgesetzt.
Eine genauere Lautumgebung -- falls man sie überhaupt angeben kann --
ist unsicher.
Vermutlich ist der Wandel schlicht nicht lautgesetzlich
sondern als eine dialektal unterschiedliche
Verallgemeinerung von Sandhivarianten zu sehen.

\lex{\i{t}}{\i{s}}{\_\i{i}} außer nach \i{s}, \i{k} (nwgr.)

uridg. \rek{bʰéronti} > ion-att. φέρουσι, dor. φέροντι `sie tragen'

uridg. \rek{bʰeh₂ti} > ion-att. φησί, dor. φᾱτί `er sagt'

s.o.

\mypara{\rek{ə} > \rek{a}/\rek{o}}
\label{a-o}
Der Sprossvokal, der neben/aus Liquida bzw. Nasalis sonans entstanden ist,
wird dialektal zu \bel{a} bzw. \bel{o}.
Da die genaue dialektale Verteilung schwierig ist
(im Ark.-kypr. und Myk. finden sich beide Vokalisationen),
nehme ich hier als ungefähre Annäherung für aiolisch \bel{o},
ansonsten \bel{a} an.

\lex{\i{ə}}{\i{o}} (aiol.)

\lex{\i{ə}}{\i{a}}

lesb. boiot. στροτός vs. att. στρατός `Truppe, Heer'

thess. πετρο- myk. \bel{qe-to-ro} vs. att. τετρα- `vier'

myk. \bel{a-mo-ta} vs. att. ἅρματα `Wagen'

lesb. ark. δέκοτον myk. \bel{de-ko-to} vs. att. δέκατον `das zehnte'

\mypara{\rek{pč} > \bel{pt}}
\label{pt}
Nach \textcite[106]{bart_hand} waren die Laute im Myk. schon zusammengefallen,
da das PTE-Zeichen oft mit <pe-te> alterniert.
Dies kann man jedoch auch so deuten,
dass die Laute eben noch nicht ganz zusammengefallen waren
(<pte> also noch für etwas wie \rek{pč} stand)
und erst im Laufe der myk. Überlieferung eine Schreibung <pe-te> für ursprüngliches \rek{pč}
bzw. <pte> für ursprüngliches \bel{pte} möglich wurde.
Wie auch immer die genauere Datierung sein mag,
lange vor myk. Zeit kann sie nicht sein,
da die Palatalisierungen eher spät in urgr. Zeit sein dürften und ein
allzu plötzlicher Wandel \rek{pi̯} > \bel{pt} unwahrscheinlich scheint.

\lex{\i{pč}}{\i{pt}}

s. Beispiele bei \ref{pal}

\mypara{\rek{i̯} > \rek{h}}
\label{yV-}
Anlautendes und intervokalisches \rek{i̯} wird im Myk. teilweise mit der J-Reihe,
teilweise (wenn vorhanden) mit einem H-Zeichen,
und teilweise gar nicht geschrieben.
Daraus ist zu schließen,
dass die Verhauchung im Laufe der mykenischen Überlieferung passierte.

\lex{\i{i̯}}{\i{h}}{(\#, \i{V})\_\i{V}}

myk. \bel{jo} und \bel{o} = att. ὡς `wie'

\subsection{Nachmykenische/alphabetische Zeit}
In dieser Zeit bilden sich die meisten Unterschiede der Dialekte heraus.
Durch die Fülle an Dialekten und der dazu im Vergleich eher
spärlichen Belegsituation sind die genauen Verhältnisse
in vielen Fällen unklar und die hier aufgestellten Regeln
entsprechend mit mehr oder weniger großer Unsicherheit behaftet.
In fast jedem Fall müsste man sich eingehend mit den Dialektinschriften
auseinandersetzen, um das Phänomen so gut es geht beschreiben zu können.
In vielen Fällen sind die Lautwandel zwar recht klar,
aber wie sie dialektal gestreut sind und wann sie zu datieren sind,
ist oft sehr unsicher.

Ein Hauptproblem bei der Überlieferung ist,
dass viele Dialekte erst in späterer Zeit überliefert sind,
in der man schon mit Beeinflussung durch die ion-att. Koiné rechnen muss.
Daher weiß man oft nicht,
ob ein Merkmal in einem Dialekt ererbt oder entlehnt ist.

Letztendlich ist es nicht mein Ziel,
die exakte Lautung jedes belegten Dialektes aus dem Urgr. herzuleiten,
da diese Aufgabe wegen des zeitlich und räumlich gesäten Variantenreichtums
in dieser Arbeit nicht zu schaffen ist.
Stattdessen werde ich mich auf den Lautstand der etwas größeren Dialektgebiete
zur Zeit des Einsetzens ihrer Überlieferung konzentrieren
(auch wenn sich dieser Zeitpunkt von Dialekt zu Dialekt recht stark unterscheiden kann).

Die eingeschobenen ungefähren Jahreszahlen habe ich anhand einiger Termini
post quos für die Dialektaufspaltung,
die ich aus den von den Lautwandeln betroffenen Dialekten geschlossen habe,
aus dem Dialektstammbaum von \textcite{bart_frueh} eingefügt.
Einige Zeiträume scheinen merkwürdig ereignislos, andere wiederum sehr bewegt.
Auch hier ist also wohl noch keine Wahrheit gefunden und
an Datierung sowohl der dialektalen Aufspaltung als auch der Lautgesetze
sollte noch gearbeitet werden.

\mypara{\rek{čč}, \rek{ǰǰ}}
\label{dz}
\rek{čč} fällt mit \rek{tš} zusammen,
die im Myk. (unter der Annahme, dass meine Deutung der Schreibungen korrekt ist) noch getrennt sind.
Da es außerhalb des Myk. keinen Hinweis auf ursprünglich zwei verschiedene
Laute gibt,
setze ich dieses und das nächste Lautgesetz so früh wie möglich an.

\lex{\i{čč}}{\i{tš}}

\bigskip\noindent
\rek{ǰǰ} wird (vielleicht über \rek{dž}) zur dentalen Affrikata \rek{dz},
die in den Regeln, da wir bei \bel{s} die Stimmhaftigkeit ignorieren,
mit \bel{ds} bezeichnet ist.
Die genaue spätere Lautung ist umstritten.
Geschrieben wird der Laut mit <ζ>,
das auf jeden Fall [\bel{zd}] bezeichnen konnte,
was Wörter zeigen, bei denen \rek{zd} ursprünglich ist
(z.B. ὄζος = got. \bel{asts} `Ast').
Eine Metathese \rek{zd} > \rek{dz} ist unwahrscheinlich
und einer Erhaltung von \rek{dz} < \rek{(d, g)i̯} widerspricht
ἔρδω `machen' < \rek{u̯erzdō} < \rek{u̯erdzō} < \rek{u̯erǰǰō} < \rek{u̯ergi̯ō}.
\rek{dz} wird also nur als Zwischenstufe zwischen \rek{ǰǰ} und \rek{zd}
angenommen.

\lex{\i{ǰǰ}}{\i{ds}}

\mypara{el. \bel{ē} > \bel{ǣ}}

Im Elischen ist \bel{ē} mit <ε> aber auch oft mit <α> wiedergegeben,
was eine sehr offene Aussprache nahelegt.
Für \textcite[89ff.]{bart_langvok} ist dieser Lautwandel sehr alt und er
setzt ihn vor dem 1. Jt. an.

\lex{\i{ē}}{\i{ǣ}} (el.)

zu uridg. \rek{di̯ḗm} A.Sg. `Himmel(sgott)':
el. Ζᾶνες N.Pl. `Zeusstatuen' vs. hom. Ζῆνα A.Sg. `Zeus'

el. μᾱ́ neben μή `nicht'

el. Ϝρᾱ́τρᾱ vs. lak. ῥήτρᾱ `Verabredung'

\mypara{Labiovelare}
\label{labvel}
Die genaue Entwicklung der Labiovelare ist in einigen Dialekten etwas unklar.
Gemeingriechisch ist der Wandel \{Labiovelar\} > \{Labial\}.
Darüber hinaus haben die meisten Dialekte einen Wandel
\{Labiovelar\} > \{Dental\} in bestimmten Umgebungen.

Das Aiolische hat nur ausnahms- und unerklärterweise dentalen Reflex.
Das Ark. und Kypr. haben von den restlichen Dialekten etwas andere bzw.
weitergehende Entwicklungen,
auf die ich allerdings in dieser vereinfachten Darstellung
nicht detailliert eingehen werde. Rix §96-99.

Da sich alle aiol. Dialekte gleich verhalten,
ist der Terminus ante quem für die Entwicklung zu Dentalen
wohl vor der Aufspaltung des Uraiol. zu sehen,
den \textcite{bart_frueh} auf etwa 1100 datiert.
Da sich das Ark. und Kypr. vor \bel{e} anders verhalten,
dürfte der Wandel nach Aufspaltung dieser beiden Dialekte
passiert sein, den \textcite{bart_frueh} auf etwa 1300 datiert.
Die Entwicklung zu Labialen muss etwas später sein,
erfasst aber den gesamten Sprachraum,
und zwar vor Beginn der alphabetischen Überlieferung.
Damit ist hierfür der Terminus ante quem in etwa das 9. Jh.

\lex{\i{kʷ}}{\i{t}}{\_\i{i}} (nicht aiol.)

uridg. \rek{kʷis} > gr. τίς (auch aiol.!);  lat. \bel{quis} `wer?'

ion-att. τίσις `Vergeltung' neben ποινή `Buße'

\lex{\{Labiovelar\}}{\{Dental\}}{\_\i{e}} (nicht aiol. kypr.)

uridg. \rek{-kʷe} > gr. -τε (auch aiol.!); lat. \bel{-que} `und'

uridg. \rek{pénkʷe} > aiol. πέμπε, sonst πέντε `fünf'

uridg. \rek{gʷelbʰús} > gr. δελφύς `Mutterleib';
Δελφοί neben boiot. Βελφοί

zu uridg. \rek{gʷʰen-/gʷʰon-}: gr. θείνω `schlagen' neben φόνος `Mord'

\lex{\{Labiovelar\}}{\{Labial\}}

uridg. \rek{pénkʷtos} > gr. πέμπτος `der fünfte'

\rek{gʷḿ̥i̯ō} > \rek{gʷámi̯ō} > gr. βαίνω; lat. \bel{veniō} `ich komme'

s.o. φόνος

\bigskip\noindent
Nach dem Wandel assimilieren sich die Nasale wieder an den folgenden Laut
(z.B. in πέμπτος).
Dies ist wohl auch als phonologische Regel anzusetzen.

\mypara{\rek{b}, \rek{g} > m, ŋ}

\rek{b} und \rek{g} assimilieren sich an ein folgendes \bel{n}
und werden zu Nasalen (Rix §105),
wobei zu beachten ist,
dass der velare Nasal von der velaren Media graphisch nicht unterschieden ist
(beide <γ>).

Dies muss, wenn man keinen labiovelaren Nasal annehmen will,
nach der Beseitigung der Labiovelare passiert sein.

\lex{\i{b}, \i{g}}{\i{m}, \i{ŋ}}{\_\i{n}}

\rek{ti̯egʷnós} > \rek{sebnós} > σεμνός `ehrfürchtig'

\subsection*{ca. 1000}

\mypara{\rek{\{Dental\}u̯}}
\rek{tu̯} verhält sich im Anlaut wie \rek{ti̯}, sonst wie \rek{ki̯} (Rix §104).
Eine Hesychglosse erweist unverändertes \bel{tu̯e} (<τρε>) fürs Kretische.
Dass \rek{tu̯} nicht komplett mit einer der beiden Lautgruppen zusammenfällt,
ist etwas überraschend.

Nach Rix bleibt \rek{tʰu̯} zunächst erhalten.
Da Beispiele fehlen,
könnte man aber auch (zumindest im Inlaut) als Ergebnis \bel{stʰ} annehmen,
wodurch man die Endung der 2.Pl. -σθε mit ai. \bel{-dhvam} verbinden kann.

Eine genaue Datierung ist unklar.

\lex{\i{tu̯}}{\i{s}}{\#\_} (nicht kret.?)

uridg. \rek{tu̯e} > ion-att. σε; cf. ai. \bel{tvām} `dich'

\rek{tu̯ei̯sō} > gr. σείω `erschüttern'

\lex{\i{tu̯}}{\i{tš}}

hom. ἐσσείοντο zu σείω, cf. ai. \bel{atviṣanta} `sie gerieten in Erregung'.

\rek{kʷetu̯r̥es} > ion. τέσσαρες, att. τέτταρες, boiot. πέτταρες `vier'

\lex{? \i{tʰu̯}}{\i{sth}}

s.o.

\mypara{\rek{dz} > \bel{zd}}
\label{zd}
Wie schon bei \ref{dz} erklärt, wird \rek{dz} zu \bel{zd} <ζ>.

Dies muss wegen ἔρδω vor \ref{CsC} passieren.

\lex{\i{ds}}{\i{sd}}

\mypara{\rek{(N, \{Dental\})sC}}
\label{NTsC}
Nasal und Dental schwinden vor \rek{sC}.
Dies muss vor \ref{CsC},
wie att. ἴσος < \rek{u̯itsu̯os} `vielleicht' und δεσπότης < \rek{dems-pótās} `Hausherr' zeigen,
und nach \ref{zd} passieren, wie
σύζυγος < \rek{sun-zdugos} < \rek{sun-dzugos} `zusammengespannt' zeigt.
Rix §78.

\lex{(\i{N}, \{Dental\})}{\i{Ø}}{\_\i{sC}}

\rek{kent-tós} > \rek{kenstós} > gr. κεστός `bestickt'

s.o.

\mypara{\rek{CsC}}
\label{CsC}
Zwischen Konsonanten wird \rek{s} in der Regel zu \rek{h} verhaucht und
schwindet dann ggf. mit Hauchumsprung (s. \ref{hauchumsprung}) (Rix §87c, Lejeune §132, 133).
Die genaue Entwicklung ist nicht völlig klar,
aber zwischen Resonant oder Halbvokal und stimmlosem Okklusiv
scheint \rek{s} erhalten zu sein.
Bestimmte Lautgruppen sind dann jedoch dialektal teilweise vereinfacht worden.
Eine Tenuis vor dem \rek{s} wurde aspiriert,
woran sich eine folgende Tenuis assimilierte.
Eine folgende Media hatte Assimilation der Aspirata an die Media als Folge.

Dies muss -- wie schon gesagt -- nach \ref{zd} und \ref{NTsC} passieren.
Im Myk. ist die Lautverbindung noch intakt.

\lex{\i{s}}{\i{h}}{\i{C}\_\i{C}} (außer in (\i{R}, \i{V̯})\_(\i{T}, \i{Tʰ}))

uridg. \rek{pérsneh₂}: ion-att. πτέρνη; got. \bel{faírzna} `Ferse'

\rek{orsmā́} > ion-att. ὁρμή `Ansturm, sich in Bewegung setzen'

aber:

\rek{u̯oi̯d-tʰa} > \rek{u̯oi̯stʰa} > att. οἶσθα

\rek{par-stádes} > παστάδες neben παρτάδες `Vorhallen'

\rek{tʰurstʰen} > θύρθεν neben θύσθεν `draußen (vor der Tür)'

\lex{\i{Th}}{\i{Tʰ}}

myk. \bel{ai-ka-sa-ma} \bel{/ai̯ksmā́/} = hom. αἰχμή `Speer'

\lex{\i{T}}{\i{Tʰ}}{\i{Tʰ}\_}

\rek{eks-trós} > gr. ἐχθρός `Fremder, Feind'

\lex{\i{Tʰ}}{\i{D}}{\_\i{D}}

zu uridg. \rek{pesd-/psd-}: gr. βδέω `furzen'

\mypara{\rek{u̯i̯} > \rek{ẅẅ}}
\rek{u̯i̯} wird zu einer palatalen Geminate.
Damit gliedert sie sich in die Reihe der palatalen \rek{ĺĺ}, \rek{ŕŕ}, \rek{ńń}
ein, mit denen sie sich weitgehend parallel entwickelt (s. \ref{palgem}, \ref{ED1}).
Rix §73.

\lex{\i{u̯i̯}}{\i{ẅẅ}}

\rek{kau̯-i̯ō} > hom. καίω, att. κᾱ́ω `ich entzünde'; cf. καῦμα `brennende Hitze'

\rek{au̯i̯-etós} > hom. αἰετός, att. ᾱ̓ετός `Adler'; cf. lat. \bel{avis} `Vogel'

\bel{díu̯i̯os} (myk. \bel{di-wi-jo}) > ion-att. δῖος `göttlich'

\mypara{Palatale Geminaten}
\label{palgem}
Nach \bel{a} oder \bel{o} spalten sich \rek{ŕŕ}, \rek{ńń} (Rix §70a), \rek{ẅẅ} (Rix §73)
in \bel{i̯} und ihre ungeminierte unpalatale Entsprechung
(\bel{i̯u̯} entwickelt sich dialektal später zu \bel{i̯i̯} weiter, s. \ref{digamma}).
Im Kypr. ist das auch bei \rek{ĺĺ} der Fall (zumindest nach \bel{a}).
Ansonsten wird \rek{ĺĺ} entpalatalisiert und bleibt Geminate (Rix §70c).
Im Thess. und Lesb. werden \rek{ŕŕ}, \rek{ńń} in allen anderen Fällen entpalatalisiert (in den anderen Dialekten Ersatzdehnung, s. \ref{ED1}) (Rix §70b).

\lex{\i{ŕŕ}, \i{ńń}, \i{ẅẅ}}{\i{i̯r}, \i{i̯n}, \i{i̯u̯}}{(\i{a}, \i{o})\_}

\rek{katʰari̯ō} > καθαίρω `ich reinige'

\rek{gʷm̥i̯ō} > \rek{gʷani̯ō} > βαίνω `ich komme'

\rek{au̯i̯etós} > αἰβετός (Hesychglosse), hom. αἰετός `Adler'

\rek{kom-i̯ós} (cf. lat. \bel{cum} `mit') > κοινός `gemein(sam)'

\lex{\i{ĺĺ}}{\i{i̯l}}{(\i{a}, \i{o})\_} (kypr.)

\lex{\i{ĺĺ}}{\i{ll}}

uridg. \rek{áli̯os} > kypr. αἶλος, sonst ἄλλος; lat. \bel{alius} `ein anderer'

\lex{\i{ŕŕ}, \i{ńń}}{\i{rr}, \i{nn}} (thess. lesb.)

\rek{pʰtʰeri̯ō} > lesb. φθέρρω, ion-att. φθείρω `ich zerstöre'

\rek{oi̯ktiri̯ō} > lesb. οἰκτίρρω, ion-att. οἰκτῑ́ρω `ich bedauere'

\rek{kteni̯ō} > lesb. κτέννω, ion-att. κτείνω `ich töte'

\rek{krini̯ō} > lesb. κρίννω, ion-att. κρῑ́νω `ich entscheide, wähle'

\mypara{Erste Ersatzdehnung}
\label{ED1}
Die erste Ersatzdehnung dehnt grob gesprochen Vokale vor palatalen Geminaten (Rix §70b, 73),
\rek{h} + Resonant (Rix §88d) und \rek{ln} (Rix §77) außer im Thess. und Lesb.,
wo es zur Gemination kommt (s. \ref{gem}).
Gedehnte \bel{e}, \bel{o} fallen dabei nur in einigen Dialekten
mit alten \bel{ē}, \bel{ō} zusammen.
Im Ion-att., Nwdor. und Sardor.
entstehen dabei neue geschlossene Langvokale \rek{ẹ̄}, \rek{ọ̄}
(\cite{bart_langvok}).

Datiert werden muss der Wandel vor der Hebung von \rek{ā} (\ref{a>e})
im Ion-att.

\lex{(\i{e}, \i{i}, \i{u})\i{ŔŔ}}{(\i{ẹ̄}, \i{ī}, \i{ū})\i{R}} (mit \i{Ŕ} = \i{ŕ}, \i{ń}, \i{ẅ})

s.o. \ref{palgem}

\rek{pluni̯ō} > ion-att. πλῡ́νω `ich wasche'

\bel{díu̯i̯os} (myk. \bel{di-wi-jo}) > ion-att. δῖος `göttlich'

zu εὐρύς: \rek{eu̯réu̯-i̯a} > hom. att. εὐρεῖα (wohl \bel{/ẹ̄/}, nicht \bel{/ei̯/}), lesb. εὔρηα `breit'

\lex{\i{VRh}, \i{VhR}}{\i{V̄R}}{\_\i{V}} (mit \i{R} = Resonant oder \i{u̯}; nicht thess. lesb.)

s. die Beispiele bei \ref{sh}

\lex{\i{Vln}}{\i{V̄l}}{\_\i{n}} (nicht thess. lesb.)

\lex{\i{Vln}}{\i{Vll}}

\rek{gʷolnā́} > ion-att. βουλή, ark. βωλᾱ́, lesb. βόλλᾱ `Wille, Rat'

\mypara{\rek{u̯s} > \rek{u̯h}}
\label{uh}
Der Unterschied, den die Betonung bei \rek{u̯s} offenbar ausmacht,
verdient eine gesonderte Betrachtung.
Die Rekonstrukte uridg. \rek{h₂eu̯sṓs} `Morgenröte'
und \rek{h₁éu̯soh₂} `ich senge' haben trotz ihrer fast gleichen Lautung
die sehr unterschiedlichen Reflexe att. ἕως, hom. ἠώς, dor. ἀϜώς, lesb. αὔως
und att. εὕω.
die Morgenröte hat außer in att. ἕως keine Behauchung
und ist für \textcite[12]{sommer} von ἑσπέρᾱ `Abend' übernommen
(wobei das Lesb. und ursprünglich wohl auch das Hom. psilotisch sind --
s. \ref{psilose} -- und daher keine sehr zuverlässige Aussage machen).
Ansonsten ist Ersatzdehnung (außer im Lesb., wo immer geminiert wird) zu beobachten.
Att. εὕω hingegen hat keine Ersatzdehnung sondern Gemination
und auch die Behauchung kann kaum von einem anderen Wort kommend erklärt werden.
Die einzig plausible Lösung scheint, den Unterschied im Akzent zu sehen,
was \textcite{wackernagel} zuerst schon bei \bel{rs} beobachtete
(vgl. κόρση (att. κόρρη) `Tempel, Teil der Stirn' vs. κουρᾱ́ `Schnitt').
Dass in der Stellung nach Akzent das \rek{s} stabiler ist als vor Akzent,
zeigt bei \bel{r} das Beispiel oben.
Daher liegt die Vermutung nahe,
dass in \rek{u̯s} das \rek{s} nach unbetonter Silbe
schon früh verhaucht wurde (s. \ref{sh}),
dann der Ersatzdehnung (bzw. Gemination) unterliegt (s. \ref{ED1}),
nach betonter Silbe aber zunächst erhalten bleibt.
Später wird erhaltenes \rek{u̯s} zu \bel{u̯u̯} geminiert (s. \ref{gem}),
wie att. εὕω zeigt.
Da hier offenbar aber zusätzlich (!) der Hauchumsprung (s. \ref{hauchumsprung})
stattfindet,
ist eine vorherige Verhauchung des \rek{s} anzunehmen.

Eine nette Bestätigung dieses Ergebnisses sind die
Wörter ion-att. ἀκούω \bel{/akóu̯u̯ō/} < \rek{akóu̯sō} `ich höre'
und hom. ἀκουή \bel{/akọ̄ε̄́/}, att. ἀκοή \bel{/akoε̄́/} < \rek{akou̯sā́} `Gehör'.

\lex{\i{s}}{\i{h}}{\i{u̯}\_}

\mypara{Hauchumsprung}
\label{hauchumsprung}
Die genauen Bedingungen für den Hauchumsprung sind etwas unklar.
Grob gesagt springt ein \bel{h} direkt nach der ersten Silbe
in den Anlaut derselben.
Wenn das \bel{h} intervokalisch ist,
geschieht dies nur, wenn die zweite Silbe unbetont ist (\cite[10]{sommer}).
Zwischen Resonant oder \bel{u̯} und Vokal scheint \bel{h}
nicht nur in den Anlaut der ersten Silbe zu springen,
sondern auch am urpsprünglichen Ort zu bleiben,
anders ist att. εὕω < \rek{éu̯sō} nicht verständlich.

Die Formulierung klingt  nicht besonders überzeugend und
möglicherweise sind die Regeln dialektal auch unterschiedlich.

\lex{\i{V₁hV₂}}{\i{hV₁V₂}} (wenn \i{V₂} unbetont)

\rek{é-serpont} > εἷρπον `sie schlichen'

\rek{iserós} > ἱερός `heilig'

aber: \rek{īháomai̯} > att. ἰάομαι `ich heile'

\lex{\i{V}(\i{R}, \i{u̯})\i{hV}}{\rek{hV}(\i{R}, \i{u̯})\i{hV}}

\rek{orsmā́} > ion-att. ὁρμή `Ansturm, in Bewegung setzen'

\rek{éu̯sō} > att. εὕω `ich senge'

\mypara{Gemination}
\label{gem}
Zwischen Vokalen assimiliert sich \rek{h} neben Resonanten oder \bel{u̯} an diesen Laut,
so dass eine Geminate entsteht.
Hauptsächlich betrifft dieses Gesetz das Lesb. und Thess.,
da diese Dialekte nicht von der 1. Ersatzdehnung (\ref{ED1}) betroffen waren,
aber da \rek{u̯h} < \rek{u̯s} auch in anderen Dialekten wirkt,
scheint es sinnvoll, das Gesetz für alle Dialekte zu formulieren.

\lex{(\i{R}, \i{u̯})\i{h}}{(\i{RR}, \i{u̯u̯})}{\i{V}\_\i{V}}

s. die Beispiele bei \ref{sh}

s. \ref{uh} zu \rek{éu̯sō} > att. εὕω `ich senge'

\mypara{\rek{h} > \bel{Ø}}
\rek{h} schwindet außer im Anlaut.

\lex{\i{h}}{\i{Ø}} (außer \#\_)

\mypara{\rek{ts}}
\label{ts}
\rek{ts} wird nach Konsonant (also auch nach Diphthong) zu \bel{s} vereinfacht.
Vor Konsonant ist dies schon bei \ref{NTsC} passiert.

Nach Vokal wird es Im Boiot. und Kret. zu \bel{tt},
sonst überall zunächst zu \bel{ss},
das dann in allen Dialekten nach Langvokal,
im Ion-att. und Ark. überall vereinfacht wird (bei Homer als Archaismus oder Aiolismus aber oft nicht). Rix §102, Sihler 196.

\lex{\i{ts}}{\i{s}}{\i{C}\_}

\rek{pánti̯a} > \rek{pántsa} > \rek{pánsa} >
thess. kret. ark. πάνσα, lesb. παῖσα, ion-att. πᾶσα (s. \ref{ED2}) `jede'

\lex{\i{ts}}{\i{tt}} (boiot. kret.)

\lex{\i{ts}}{\i{ss}}

boiot. ὁπόττος, kret. ὀπόττος `so groß wie';
ion-att. τόσος, lesb. τόσσος `so groß'

uridg. \rek{médʰi̯os} > ion-att. μέσος, lesb. μέσσος, boiot. kret. μέττος `in der Mitte'

\lex{\i{ss}}{\i{s}}{\i{V̄}\_}

hom. τάπησι (ταπητ- + -σι) `auf den Teppichen' vs. hom. ποσσί/ποσί (ποδ- + -σι) `an den Füßen'

\lex{\i{ss}}{\i{s}} (ion-att. ark.)

s.o. μέσος \&c.

\mypara{\bel{zd} > \bel{d(d)}}
In einigen Dialekten
wird \bel{zd} zu \bel{dd}, im Anlaut zu \bel{d} (Rix §102, 103).

\lex{\i{zd}}{\i{dd}} (boiot. kret. lak. el.)

Präsentien in -άδδω, -ίδδω vs. -άζω, -ίζω

\lex{\i{dd}}{\i{d}}{\#\_}

boiot. Δεύς = ion-att. Ζεύς; kret. Δῆνα = hom. Ζῆνα

\subsection*{ca. 900}

\mypara{\bel{ā} > ion-att. \bel{ǣ}}
\label{a>e}
Das offensichtlichste Merkmal der ion-att. Dialekte.
\bel{ā} wird gehoben, fällt jedoch noch nicht mit ererbtem \bel{ē} zusammen,
da die beiden Laute im Inselion. noch mit unterschiedlichen Zeichen geschrieben werden.

Der Lautwandel ist zwischen der ersten und zweiten Ersatzdehnung anzusetzen,
da durch die erste Ersatzdehnung entstandenes \bel{ā} den Lautwandel mitmacht
und bei der zweiten ein neues \bel{ā} entsteht.

\lex{\i{ā}}{\i{ǣ}} (ion-att.)

ion-att. μήτηρ, sonst μᾱ́τηρ `Mutter'

In der Nikandre-Inschrift: <Νικανδρη> mit <η> für \bel{/ǣ/} aber <εθεκεν> (klassisch ἔθηκεν) mit <ε> für \bel{/ē/} (sonst beide ion. <η>)

\subsection*{ca. 800}

\mypara{Zweite Ersatzdehnung}
\label{ED2}

Bei der zweiten Ersatzdehnung schwindet \bel{n}
in der Lautgruppe Vokal + \bel{ns},
das entweder im Auslaut steht und somit nicht
der ersten Ersatzdehnung (s. \ref{ED1}) unterlag,
oder im Inlaut mit neuentstandenem \bel{s} < \rek{ts}.

Die zweite Ersatzdehnung betrifft nicht alle Dialekte in gleicher Art.
In einigen wird der Vokal gedehnt,
in anderen wird er zum i-Diphthong,
und in einigen unterbleibt der n-Schwund und damit die Ersatzdehnung völlig
(Rix §78).
Das genaue Ergebnis kann in- und auslautend unterschiedlich sein
und auch vom Vokal abhängen.
Auslautendes \bel{-Vns} konnte schon durch \ref{NTsC} als antekonsonantische
Sandhivariante zu \bel{-Vs} werden und verallgemeinert werden,
so dass einige Dialekte zur Zeit der zweiten Ersatzdehnung \bel{-Vns}
schon ganz aufgegeben hatten.

Die Ergebnisse von gedehnten \bel{e}, \bel{o} sind
im Ion-att., Nwdor. und Inseldor. die geschlossenen Langvokale
\bel{ẹ̄}, \bel{ọ̄}, ansonsten fallen sie mit den alten offenen zusammen.

Die genauen Regeln sind also nicht ganz einfach zu formulieren.
Im wesentlichen halte ich mich an die Karte,
die bei \textcite{bart_frueh}
abgedruckt ist
(ursprünglich von Hainsworth, \textit{Cambridge Ancient History} III 1, 863).

\lex{\i{Vns}}{\i{Vi̯s}}{\_\i{V}} (lesb.)

\lex{\i{Vns}}{\i{V̄s}}{\_\i{V}} (ion-att. boiot. el. lak. inseldor. kypr.)

\rek{-onti̯a} (Partizip N.Sg.f.) > \rek{-onsa} >
kret. -ονσα; lesb. -οισα; ion-att. -ουσα, lak. -ωσα

\lex{\i{Vns}}{\i{Vi̯s}}{\_\#} (lesb. el.)

\lex{\i{Vns}}{\i{V̄s}}{\_\#} (ion-att. boiot. lak. inseldor.)

uridg. \rek{tóns} (Artikel A.Pl.) >
kret. arg. τόνς; lesb. τοίς; ion-att. τούς, lak. τώς

\mypara{Kontraktionen}
\label{kontr1}
Einige Kontraktionen passieren vor dem Digammaschwund,
sind aber teilweise unklar und schwer zu formulieren
und werden daher hier übersprungen.

\mypara{Digamma}
\label{digamma}
Bei \bel{u̯} ist die dialektale Entwicklung äußerst schwierig.
Teilweise geht es vor der Überlieferung verloren (z.B. ion-att.),
teilweise ist es bis heute im Tsakonischen bewahrt.
Grundsätzlich schwindet \bel{u̯} in vielen Dialekten
zwischen Vokalen (dort besonders oft), im Anlaut (vor Vokalen und Konsonanten)
und nach Konsonanten (unter \ref{ED3} behandelt)
(Rix §71, 72, 73).
Nur bei den Dialekten,
in denen seit dem Beginn der Überlieferung \bel{u̯} fehlt,
wird hier der Schwund angesetzt.
Die hier formulierten Regeln beruhen hauptsächlich auf
\textcite{buck} und \textcite{lejeune}
und ihre Vorläufigkeit und Unvollständigkeit kann nicht genug betont werden.
Ehrlicherweise müsste man diesen Abschnitt zunächst überspringen,
jedoch ist er zu wichtig um völlig ignoriert zu werden.
Eine genaue Untersuchung wäre gerade zu diesem Thema sehr wünschenswert.
Der (intervokalische) Digammaschwund hängt eng mit den Kontraktionen zusammen;
eine genauere Untersuchung muss also beide Phänomene betrachten.

Manchmal wird \bel{u̯} im Anlaut zu \bel{h} statt ganz zu schwinden,
ohne dass die Bedingungen genau bekannt wären.

\lex{\i{u̯}}{\i{Ø}}{\i{V}\_\i{V}} (ion-att. inseldor. lesb. kret. ark.)

uridg. \rek{néu̯os}: myk. \bel{ne-wo}, kypr. νεϜοσ-, sonst νέος `jung, neu'

\lex{\i{u̯}}{\i{h}}{\#\_\i{VsC}} (att.)

\rek{u̯εspérā} > att. ἑσπέρᾱ `Abend'; cf. lat. \bel{vesper}

\lex{\i{u̯}}{\i{Ø}}{\#\_\i{V}} (ion-att. inseldor. lesb.)

\rek{u̯óikos} > οἶκος, sonst Ϝοῖκος `Haus' (cf. lat. \bel{vīcus} `Dorf')

\lex{\i{i̯u̯}}{\i{i̯i̯}}{\i{V}\_\i{V}} (ion-att. inseldor. lesb.)

\rek{ai̯u̯esi} > kypr. lokr. αἰϜεί, hom. αἰεί `immer (← zu allen Zeiten)'

\rek{élai̯u̯on} (entlehnt in lat. \bel{oleum}) >
myk. \bel{e-ra₃-wo}, ion-att. ἔλαιον `Olivenöl'

\rek{elái̯u̯ā} (entlehnt in lat. \bel{olīva}) >
myk \bel{e-ra-wa}, hom. ἐλαίη `Olive'

\rek{au̯i̯-etós} > \rek{ai̯u̯etós} > hom. αἰετός; αἰβετός (Hesychglosse)

\lex{\i{u̯}}{\i{Ø}}{\i{h}\_\i{V}} (ion-att. inseldor. lesb.)

uridg. \rek{su̯ekurós} > ἑκυρός, ai. \bel{svaśurás} `Schwiegervater'

\lex{\i{u̯}}{\i{h}}{\#\_\i{r}} (ion-att. inseldor.)

\rek{u̯rḗtrā} > att. ῥήτρᾱ, el. Ϝρᾱ́τρᾱ `Verabredung'

\mypara{\bel{\#hR}}
Vor \bel{l}, \bel{m}, \bel{n} schwindet \bel{h} im Ion-att.
Wann und in welchen Dialekten dies noch geschieht,
bleibt herauszufinden.

\lex{\i{h}}{\i{Ø}}{\#\_(\i{l}, \i{m}, \i{n})} (ion-att.)

ion-att. λαβών neben aigin. λhαβο̄́ν `genommen habend'

ion-att. μία < \rek{smía} `eine' zu uridg. \rek{sem-} `ein'

ion-att. νείφει `es schneit' zu uridg. \rek{snei̯gʷʰ-} `schneien'


\mypara{att. \bel{ǣ} > \bel{ā}}
Das wichtigste Merkmal, das das Attische vom Ionischen unterscheidet.
Nach \bel{e}, \bel{i}, \bel{r} wird gehobenes \bel{ǣ} zurück zu \bel{ā}
gesenkt.
Dieser Wandel muss zwischen  \ref{digamma} und \ref{ED3}
datiert werden, wie die Beispiele zeigen.

\lex{\i{ǣ}}{\i{ā}}{(\i{e}, \i{ē}, \i{ẹ̄}, \i{i̯}, \i{i}, \i{ī}, \i{r})\_}

\rek{néu̯ā} > att. νέᾱ, ion. νέη `jung, neu (f.)'

\rek{-íā} > att. -ίᾱ ion. -ίη in z.B. σοφίᾱ `Klugheit'

\rek{kʰṓrā} > att. χώρᾱ ion. χώρη `Platz, Raum, Land(schaft)'

aber \rek{kóru̯ā} > att. κόρη, ostion. κούρη `Mädchen'

%TODO: vielleicht etwas früher?
\subsection*{ca. 700}

\mypara{Dritte Ersatzdehnung und Kontraktion}
\label{ED3}
In der Lautgruppe
Resonant, \bel{d}, \bel{tʰ} oder \bel{s} + \bel{u̯} schwindet das \bel{u̯} in vielen Dialekten.
In einigen Dialekten geht das \bel{u̯} spurlos verloren,
in anderen wird vorangehender Vokal ersatzgedehnt
und in anderen bleibt die Lautkombination intakt (Rix §72, 104b; Lejeune §71, 131).
Das Ergebnis von gedehnten \bel{e}, \bel{o} sind die geschlossenen
\bel{ẹ̄}, \bel{ọ̄} im Ion-att., Nwdor. und Sardor (\cite{bart_langvok}).

Dasselbe Ergebnis haben die Kontraktionen von \bel{e} + \bel{e}
und \bel{o} + \bel{o}.

\lex{(\#, \i{V})(\i{r}, \i{l}, \i{n}, \i{d}, \i{tʰ}, \i{s})\i{u̯}}{(\#, \i{V̄})(\i{r}, \i{l}, \i{n}, \i{d}, \i{tʰ}, \i{s})}
(inselion., ostion., inseldor., kret.)

\lex{\i{u̯}}{\i{Ø}}{(\i{r}, \i{l}, \i{n}, \i{d}, \i{tʰ}, \i{s})\_} (ion-att. lesb. lak.)

\rek{kóru̯ā} > att. κόρη, ion. κούρη, kor. κόρϜᾱ, `Mädchen'

\rek{kalu̯ós} > att. καλός, ostion. κᾱλός, boiot. καλϜός, `schön'

\rek{ksénu̯os} (myk. \bel{ke-se-nu-wo}) >
att. ξένος, ostion. ξεῖνος, kor. ξένϜος `Fremder, Gast'

\rek{dédu̯imen} > att. δέδιμεν, hom. δείδιμεν `wir fürchten'

\rek{du̯ís} > hom. δίς; lat. \bel{bis}, ai. \bel{dvis} `zweimal'

\rek{u̯ítsu̯os} > \rek{u̯ísu̯os} > att. ἴσος, ostion. ἶσος, kret. ϜίσϜος `vielleicht'

\lex{\i{ee}}{\i{ẹ̄} bzw. \i{ē}}

\rek{é-deu̯e} > ion-att. ἔδει `es musste'

\lex{\i{oo}}{\i{ọ̄} bzw. \i{ō}}

\rek{-oso} > ion-att. -ου, inseldor. -ω (G.Sg. o-St.)

\mypara{\rek{tš}}
\label{tš}
\rek{tš} scheint in einigen ionischen Dialekten länger erhalten geblieben
zu sein, da ein extra Zeichen für diesen Laut benutzt wird,
auch wenn die genaue Lautung unsicher ist.
Daher setze ich diesen Lautwandel eher spät an
(und ungenauerweise in allen Dialekten gleichzeitig).
In den meisten Dialekten wird \rek{tš} zu \bel{ss}, das im Anlaut vereinfacht wird.
Im Att., Euboi., Boiot. und Kret. wird es jedoch zu \bel{tt} bzw. \bel{t}
im Anlaut.

\lex{\i{tš}}{\i{t}}{\#\_} (att. euboi. boiot. kret.)

\lex{\i{tš}}{\i{tt}} (att. euboi. boiot. kret.)

\lex{\i{tš}}{\i{s}}{\#\_}

\lex{\i{tš}}{\i{ss}}

s. die Beispiele bei \ref{pal}

\mypara{Hauchdissimilation (\textsc{Graßmann}s Gesetz)}
\label{grassmann}
\textsc{Graßmann}s Gesetz wird in der Literatur nie präzise formuliert
und meist ungenau oder gar nicht datiert
(Rix §107, \cite[147]{bart_hand}, \cite[§45]{lejeune}, \cite[§138]{sihler}).
Da die lautgesetzlichen Reflexe durch morphologischen Druck
ausgeglichen werden konnten,
ist die genaue Wirkungsweise auch nicht mehr mit Sicherheit festzustellen.
Nach einer möglichen Formulierung wird ein aspirierter Silbenanlaut
(\rek{h} oder Tenuis aspirata)
vor Tenuis aspirata in der folgenden Konsonantenverbindung deaspiriert.
Da im In- und Auslaut meist Derivations- und Flexionsmorpheme stehen,
die von dem Gesetz (vielleicht aber aus Systemzwang)
nicht betroffen zu sein scheinen,
beschränke ich meine Formulierung auf den Anlaut.
Die Dissimilation \rek{-tʰē-tʰi} > -θη-τι im Imp.2.Sg.Aor.Pass.
wird von der obigen Formulierung nicht abgedeckt,
da hier der zweite Laut deaspiriert wird.

Nach Bartoněk
ist nicht klar, ob das Gesetz vor- oder nachmyk. ist,
aber wie Lejeune mit ὄφρα < \rek{i̯ópʰra} (cf. Stamm des Relativums \rek{i̯o-})
`bis; damit' zeigt,
muss es nach \rek{i̯V-} > \bel{hV-} (s. \ref{yV-}) datiert werden.
In den Absätzen zu diesem Gesetz liefert Mayrhofer (\citeyear[§4.10.2.1.3]{mayrhofer})
sogar inschriftliche Belege für nicht-dissimilierte Wörter.
Daher datiere ich dieses Gesetz erst spät.

\lex{(\i{Tʰ}, \i{h})}{(\i{T}, \i{Ø})}{\#\_\i{C*VC*Tʰ}}

zu uridg. \rek{seg̑ʰ-}: ἔχω `ich habe' vs. ἕξω `ich werde haben'

τάφος `Grab' vs. θάπτω `ich bestatte'

τριχός `des Haares' vs. θρίξ `Haar'

nicht-dissimiliert: θυφλός statt τυφλός `Blind', ἀνεθέθη statt ἀνετέθη `er weihte' \&c.

\mypara{Psilose}
\label{psilose}
In einigen Dialekten geht anlautendes \bel{h} verloren (die einzige Position,
in der es überhaupt vorkommt).
Die genaue Datierung (gerade bei später belegten Dialekten wie dem Lesb.)
ist unsicher.

\lex{\i{h}}{\i{Ø}} (el. lesb. ostion. inselion. kret.)

\mypara{\bel{ǣ}}
\label{ae>e}
Das aus \bel{ā} gehobene \bel{ǣ} fällt mit \bel{ē} zusammen,
ist jedoch im Inselion. noch länger erhalten, wo die beiden Laute graphisch
unterschieden sind.

\lex{\i{ǣ}}{\i{ē}} (inselion. später)

\mypara{Kontraktionen}
\label{kontr2}
Einige Kontraktionen passieren nach dem Digammaschwund
(und vermutlich nach \ref{ae>e}),
sind aber ebenso schwer genau zu formulieren
wie die bei \ref{kontr1} übersprungenen
und werden deswegen ebenfalls übergangen.

\section{Schluss und Ausblick}

Die Arbeit hat gezeigt,
wie nützlich ein Programm wie \tt{lga} bei der
Sortierung und Formulierung von Lautgesetzen ist.
Viele Fehler, die einem dabei passieren,
können oftmals schnell aufgedeckt werden und die Gnadenlosigkeit
eines Computerprogramms zwingt einen zu Exaktheit und Strenge.
Diese sind (sofern sie so gut es geht der Realität entsprechen wollen)
allerdings nur auf Basis einer vernünftigen Beleglage möglich;
die Literatur führt jedoch Formen oft ohne Belegstellen an,
wodurch sie schwer überprüfbar sind
(Fundort und Datierung wären z.B. oft wünschenswert zu wissen)
und belegen außerdem nie alle beschriebenen Fälle.
Die Beispiele, die ich bei der Beschreibung der Lautgesetze gegeben habe,
sollten also in Zukunft zu wirklichen Belegen werden.

Offensichtlich wurden in dieser Arbeit viele Lautentwicklungen
noch völlig ignoriert oder noch nicht angemessen beschrieben.
Besonders die Veränderungen ab dem frühen 1. Jt.,
in dem sich die Dialekte stark differenzieren,
sind schwer in den Griff zu bekommen.
Sie bietet aber immerhin schon ein Gerüst für weitere Forschung
bzw. die Einarbeitung des in der weitergehenden Literatur
detaillierter beschriebenen
Lautwandels.

Obwohl die Resultate automatisch erzeugt wurden,
erfolgte die Überprüfung immer noch manuell.
Auch hier ist noch methodischer Verbesserungsbedarf.
In der Softwareentwicklung bedient man sich heutzutage meist Unittests,
mit denen man automatisch überprüfen kann,
ob bei einer Reihe von Testfällen für eine bestimmte Eingabe die
zu erwartende Ausgabe herauskommt.
Da wir es allerdings aufgrund der vielen Dialekte
nicht mit einer, sondern vielen Ausgaben zu tun haben,
wird die Situation komplexer bzw. unübersichtlicher und
für eine neue Version von \tt{lga} sollte man sich über eine elegante
Lösung Gedanken machen.

Neben methodischen sind aber auch technische Unzulänglichkeiten deutlich geworden.
Eine sinnvolle Behandlung von Digraphen (bzw. Polygraphen) ist nötig.
Außerdem ist die Art, auf die Laut- bzw. Zeichenklassen implementiert sind,
weder konzeptionell sauber,
da man sie im Laufe der Sprachentwicklung nicht ändern kann,
noch performant, da jeder reguläre Ausdruck vorbearbeitet werden muss.
Das erste Problem kann man, wie ja schon erwähnt, lösen,
indem man Polygraphen in jeder Zeichenkette
(d.h. regulären Ausdrücken und Eingabewörtern) durch ansonsten unbenutze
Unicode Codepoints ersetzt und vor der Ausgabe wieder zurückersetzt.
Das zweite Problem ist mit einer bestehenden Regex-Engine nicht zu lösen;
hier ist eine eigene Implementation mit einer eigenen Sprache nötig,
die auf die Bedürfnisse der Lautgesetzformulierung optimiert ist.
Ein Feature, das so eine Lautgesetzsprache haben sollte,
wäre eine Möglichkeit, Akzente wenn nötig zu ignorieren,
da ein regulärer Ausdruck \rx{ab} z.B. nicht \str{áb} matcht und man so zu
teilweise sehr umständlichen Ausdrücken gezwungen ist.
Für einige Sprachen (für das Griechische allerdings eher weniger)
wäre auch die Möglichkeit, mit Silben zu arbeiten, nützlich.
Darüber, wie die Beschreibungssprache genau aussehen sollte
und welche weiteren Features wünschenswert, notwendig
und vernünftig implementierbar sind,
wird man sich noch Gedanken machen müssen.
Für die meisten Fälle funktioniert \tt{irregex} schon erstaunlich gut,
aber die Lesbarkeit ist (wie oft bei regulären Ausdrücken) nicht optimal.

Ein weiterer Punkt ist die Anwendung der Regellisten,
um eine phonetische und graphematische Form zu erhalten.
Diese müssen z.Z. noch für jede Sprache explizit vergeben werden,
was unschön ist, aber nicht wichtig genug war,
sofort verbessert zu werden.

Was die Benutzung betrifft ist \tt{lga} noch völlig unintuitiv;
die nächste Version sollte eine graphische Benutzeroberfläche haben.
Lautgesetze sollten kommentierbar und mit Literaturverweisen
und Belegen versehen werden können.
Um die Sortierung zu erleichtern,
wäre es vielleicht nützlich,
Lautgesetze mit Beschränkungen zu versehen,
die verhindern,
dass zwei Lautgesetze, deren relative Chronologie zueinander klar ist,
nicht fälschlicherweise in die falsche Reihenfolge kommen.
Dass einem bei komplizierteren Fällen,
in denen mehrere Lautwandel ineinandergreifen, solche Fehler schnell passieren,
hat die Erfahrung gezeigt.

Schließlich ist es mein Wunsch, mit \tt{lga} und seinen zukünftigen Nachfolgern
die historisch-vergleichende Sprachwissenschaft
mehr wie freie Softwareentwicklung zu betreiben
und (hoffentlich) als Gemeinschaftsprojekt eine Sammlung aller Lautgesetze zu erstellen,
die die Lautentwicklung vom Urindogermanischen in die Einzelsprachen
möglichst gut modellieren.
Das Projekt soll als dann Referenz und Grundlage für weitere Forschung dienen können.
Aktuell findet sich sowohl die Arbeit als auch
\tt{lga} samt den bisher eingetragenen Lautgesetzen unter 
\tt{https://github.com/aap/lga}.
Diese Arbeit ist nur der erste Schritt.

\nocite{beekes}
\printbibliography

\appendix

\section{Quelltext}

Der Unterschied zwischen vorkombinierten Zeichen (also mit einem Codepoint)
und kombinierten Zeichen (mehrere Codepoints)
ist leider im Abdruck nicht mehr erkenntlich.
Grundsätzlich wurden außer bei Akzenten
wenn möglich vorkombinierte Zeichen verwendet.
Auf
\tt{https://github.com/aap/lga}
ist die korrekte und ausführbare digitale Form zu finden.

\footnotesize
\singlespacing
\subsection{main.scm}
\begin{verbatim}
;;;;
;;;; main file
;;;;

(require-extension utf8)
(require-extension irregex)
(require-extension srfi-1)

(define s+ string-append)

;;;
;;; assoc lists
;;;

(define sound-classes '())
(define phonological-rules '())
(define prettyprint-rules '())

;;; TODO: clean up the mess
;; Associates a name with a set of sounds
(define (set-class key value)
  (let ((x (assoc key sound-classes)))
    (if x
        (set-cdr! x value)
        (set! sound-classes (cons (cons key value) sound-classes)))))

;; Associates a name with a list of rules to get to phonetic representation
(define (set-phono-rules key value)
  (let ((x (assoc key phonological-rules)))
    (if x
        (set-cdr! x value)
        (set! phonological-rules (cons (cons key value) phonological-rules)))))

;; Associates a name with a list of rules to pretty-print words
(define (set-pretty-rules key value)
  (let ((x (assoc key prettyprint-rules)))
    (if x
        (set-cdr! x value)
        (set! prettyprint-rules (cons (cons key value) prettyprint-rules)))))


(define (lookup key alist)
  (let ((x (assoc key alist)))
    (if x
        (cdr x)
        '())))

(define (l key)
  (lookup key sound-classes))

;;;
;;; tree and rule functions
;;;

;; makes a list of all languages between start and end language based on a tree
(define (make-path tree start end)
  (define (find-start tree)
    (cond ((atom? tree)
           (if (eq? tree start)
               tree
               '()))
          ((eq? (car tree) start)
           tree)
          (else (let loop ((t (cdr tree)))
                  (if (null? t)
                      '()
                      (let ((s (find-start (car t))))
                        (if (null? s)
                            (loop (cdr t))
                            s)))))))
  (define (to-end tree)
    (cond ((atom? tree)
           (if (eq? tree end)
               (cons tree '())
               '()))
          ((eq? (car tree) end)
           (cons end '()))
          (else (let loop ((t tree))
                  (if (null? t)
                      '()
                      (let ((p (to-end (car t))))
                        (if (null? p)
                            (loop (cdr t))
                            (cons (car tree) p))))))))
  (to-end (find-start tree)))

;; make a list of rules that apply to the language development in path
(define (make-rules tree path)
  (define start (car path))
  (define end (car (reverse path)))
  (define (rule-applies? langlist)
    (or (null? langlist)
        (if (equal? (car langlist) 'not)
            (eq? '() (lset-intersection eq? path (cdr langlist)))
            (not (eq? '() (lset-intersection eq? path langlist))))))
  (define (find-start tree)
    (cond ((null? tree)
           '())
          ((eq? (car tree) start)
           (cdr tree))
          ((and (list? (car tree))
                (eq? 'br (caar tree)))
           (let ((l (find-start (caddar tree))))
             (if (null? l)
                 (find-start (cdr tree))
                 l)))
          (else (find-start (cdr tree)))))
  (define (to-end tree)
    (cond ((or (null? tree)
               (eq? (car tree) end))
           '())
          ((procedure? (car tree))
           (cons (car tree) (to-end (cdr tree))))   ; better use tail recursion
          ((and (list? (car tree))
                (rule-applies? (cadar tree)))
           (case (caar tree)
             ((br) (to-end (caddar tree)))
             ((sub) (append (to-end (caddar tree)) (to-end (cdr tree))))
             (else (to-end (cdr tree)))))
          (else (to-end (cdr tree)))))
  (to-end (find-start tree)))

;;;
;;; helper functions
;;;

;; convert list of strings to single string with one original string per line
(define (make-words lst)
  (if (null? lst)
      '()
      (let ((s (make-words (cdr lst))))
        (if (null? s)
            (car lst)
            (s+ (car lst) "\n" s)))))

;; make regex substitute function
(define (s pattern . subst)
  (let ((pat (string->irregex (make-regex pattern) 'utf8 'fast 'm)))
    (lambda (word)
      (apply irregex-replace/all (cons pat (cons word subst))))))

;; compile regex containing custom character classes
(define (make-regex s)
  (let loop ((classes sound-classes)
             (s s))
    (if (null? classes)
        s
        (loop (cdr classes)
              (irregex-replace/all (s+ "<"
                                       (symbol->string (caar classes))
                                       ">")
                                   s
                                   (cdar classes))))))

;; apply a list of rules to a string, return transformed string
(define (apply-rules rules words)
  (if (null? rules)
      words
      (apply-rules (cdr rules)
                   ((car rules) words))))

;; runs 'words' from language 'from' through language 'to'
(define (run-list from to words)
  (let ((r1 (make-rules rules (make-path lang-tree from to)))
        (r2 (lookup from phonological-rules))
        (r3 (lookup to prettyprint-rules)))
    (apply-rules r3
                 (apply-rules r1
                              (apply-rules r2 (make-words words))))))

;;;
;;; data
;;;

;; tree of languages
;;  an atom is a leaf language
;;  a list's car is a language, the cdr its children
(define lang-tree
  '(uridg (urgr1 (urgr (nwgr (dor (nwdor phok lokr aitol)
                                  (sardor kor meg oarg)
                                  (sdor el lak inseldor warg kret))
                             (aiol boiot (thess wthess othess) lesb))
                       (sogr (ach myk ark kypr pamph)
                             (ion-att (ostion hom) inselion euboi att))))
          ))

;; sound classes
;; TODO: these should be changeable by the rules
(set-class 'kurz-vok "e|E|o|O|a|A|i|u|ə")
(set-class 'lang-vok "ē|Ē|ō|Ō|ā|Ā|ī|ū")
(set-class 'vok (s+ (l 'kurz-vok) "|" (l 'lang-vok)))
(set-class 'liquid "r|ŕ|l|ĺ")
(set-class 's-liquid "R|L")
(set-class 'nasal "m|n|ṇ|ň|ñ|ń|ṅ")
(set-class 's-nasal "M|N")
(set-class 'halb-vok "y|w")
(set-class 'res (s+ (l 'liquid) "|" (l 'nasal)))
(set-class 'res-yw (s+ (l 'res) "|" (l 'halb-vok)))
(set-class 's-res (s+ (l 's-liquid) "|" (l 's-nasal)))
(set-class 'lary "H1|H2|H3|H")
(set-class 'sibil "s|š|ṣ|ś|z!|z|ž!|ž|ẓ!|ẓ")
(set-class 'labial "b!|p!|b|p")
(set-class 'dental "d!|t!|d|t")
(set-class 'retroflex "ḍ!|ṭ!|ḍ|ṭ")
(set-class 'palatal "ǰ!|č!|ǰ|č")
(set-class 'palatal2 "j!|c!|j|c")
(set-class 'velar "g!|k!|g|k")
(set-class 'labiovelar "G!|K!|G|K")
(set-class 'okklu (s+ (l 'labial) "|" (l 'dental) "|" (l 'retroflex)
                      "|" (l 'palatal) "|" (l 'palatal2) "|" (l 'velar)
                      "|" (l 'labiovelar)))
(set-class 'kons (s+ (l 'res) "|" (l 'halb-vok) "|" (l 'sibil) "|"
                     (l 'okklu) "|" (l 'lary)))
(set-class 'media "b|d|ḍ|ǰ|j|g|G")
(set-class 'tenuis "p|t|ṭ|č|c|k|K")
(set-class 'mediaasp "b!|d!|ḍ!|ǰ!|j!|g!|G!")
(set-class 'tenuisasp "p!|t!|ṭ!|č!|c!|k!|K!")

(load "subfuncs.scm")
(load "rules.scm")
\end{verbatim}

\subsection{subfuncs.scm}
\begin{verbatim}
;;;;
;;;; functions that transform submatches
;;;;

;; applies list of rules to the submatch i in a match
(define (match-rulelist rules)
  (lambda (i)
    (lambda (match)
      (let ((m (or (irregex-match-substring match i)
                   "")))
        (apply-rules rules m)))))

(define (match-do f)
  (lambda (i)
    (lambda (match)
      (let ((m (irregex-match-substring match i)))
        (f m)))))

(define match-print
  (match-do
    (lambda (s)
      (display "   ")
      (display s)
      (newline)
      s)))

(define laryngal->vokal
  (match-rulelist
    (list (s "H1" "e")
          (s "H2" "a")
          (s "H3" "o")
          (s "H" "a"))))

(define laryngal->langvokal
  (match-rulelist
    (list (s "H1" "ē")
          (s "H2" "ā")
          (s "H3" "ō")
          (s "H" "ā"))))

(define dehnung
  (match-rulelist
    (list (s "a" "ā")
          (s "e" "ē")
          (s "o" "ō")
          (s "i" "ī")
          (s "u" "ū"))))

(define dehnung2
  (match-rulelist
    (list (s "a" "ā")
          (s "e" "Ē")
          (s "o" "Ō")
          (s "i" "ī")
          (s "u" "ū"))))

(define kuerzung
  (match-rulelist
    (list (s "ā" "a")
          (s "ē|Ē" "e")
          (s "ō|Ō" "o")
          (s "ī" "i")
          (s "ū" "u"))))

(define dental->retroflex
  (match-rulelist
    (list (s "t" "ṭ")
          (s "d" "ḍ"))))

(define palatal->velar
  (match-rulelist
    (list (s "č" "k")
          (s "ǰ" "g"))))

(define labiovelar->velar
  (match-rulelist
    (list (s "K" "k")
          (s "G" "g"))))

(define labiovelar->labial
  (match-rulelist
    (list (s "K" "p")
          (s "G" "b"))))

(define labiovelar->dental
  (match-rulelist
    (list (s "K" "t")
          (s "G" "d"))))

(define velar->labiovelar
  (match-rulelist
    (list (s "k" "K")
          (s "g" "G"))))

(define sonans->consonans
  (match-rulelist
    (list (s "i" "y")
          (s "u" "w")
          (s "R" "r")
          (s "L" "l")
          (s "N" "n")
          (s "M" "m"))))

(define consonans->sonans
  (match-rulelist
    (list (s "y" "i")
          (s "w" "u")
          (s "r" "R")
          (s "l" "L")
          (s "n" "N")
          (s "m" "M"))))

(define stimmhaft
  (match-rulelist
    (list (s "K" "G")
          (s "k" "g")
          (s "č" "ǰ")
          (s "c" "j")
          (s "t" "d")
          (s "ṭ" "ḍ")
          (s "p" "b")
          (s "s" "z")
          (s "š" "ž"))))

(define stimmlos
  (match-rulelist
    (list (s "G" "K")
          (s "g" "k")
          (s "ǰ" "č")
          (s "j" "c")
          (s "d" "t")
          (s "ḍ" "ṭ")
          (s "b" "p")
          (s "z" "s")
          (s "ž" "š"))))

(define ->F
  (match-rulelist
    (list (s "k(!?)" "χ")
          (s "g(!?)" "γ")
          (s "t(!?)" "θ")
          (s "d(!?)" "δ")
          (s "p(!?)" "f")
          (s "b(!?)" "β"))))

(define depala
  (match-rulelist
    (list (s "ẅẅ" "w")
          (s "ŕŕ" "r")
          (s "ĺĺ" "l")
          (s "ńń" "n"))))

(define ->acute
  (match-rulelist
    (list (s "~" "'"))))

(define ->circumflex
  (match-rulelist
    (list (s "'" "~"))))

(define ->A
  (match-rulelist
    (list (s "(!|$)" "!"))))

(define ->!A
  (match-rulelist
    (list (s "!"))))

(define ->M
  (match-rulelist
    (list (s "(.*)" (->!A 1))
          (s "(.*)" (stimmhaft 1)))))

(define ->MA
  (match-rulelist
    (list (s "(.*)" (->A 1))
          (s "(.*)" (stimmhaft 1)))))

(define ->T
  (match-rulelist
    (list (s "(.*)" (->!A 1))
          (s "(.*)" (stimmlos 1)))))

(define ->TA
  (match-rulelist
    (list (s "(.*)" (->A 1))
          (s "(.*)" (stimmlos 1)))))
\end{verbatim}

\subsection{rules.scm}
Diese Datei definiert den Lautgesetzbaum und die phonologischen
und graphematischen Regeln.
Alle anderen Zweige außer dem Griechischen sind hier nicht abgedruckt.
\begin{verbatim}
(load "griechisch.scm")

(define rules
  (list
    'uridg
    `(br (urgr1) ,rules-urgr)
    ))

(set-phono-rules 'uridg
  (list (s "ʰ" "!")
        (s "kʷ" "K")
        (s "gʷ" "G")
        (s "k̑" "č")
        (s "g̑" "ǰ")
        (s "i̯" "y")
        (s "u̯" "w")
        (s "m̥" "M")
        (s "n̥" "N")
        (s "r̥" "R")
        (s "l̥" "L")
        (s "h₁" "H1")
        (s "h₂" "H2")
        (s "h₃" "H3")
        (s "\u0301" "'")	; combining acute
        ;;;
        (s "e('?)H2" "a" 1 "H2")
        (s "H2e" "H2a")
        ;;; umgefärbtes e noch von o verschieden
        (s "e('?)H3" "O" 1 "H3")
        (s "H3e" "H3O")
        ;;; TODO: bedingungen für stimmhaftwerdung
        (s "s(G|g|ǰ|d|b|r|l|m|n)" "z" 1)
        (s "(G|g|ǰ|d|b|r|l|m|n)(!?)s" 1 2 "z")
        ;;; TT -> TsT
        (s "(<dental>)(<dental>)" 1 "s" 2)
        ))

(let ((rules (list (s "ʰ" "!")
                   (s "kʷ" "K")
                   (s "gʷ" "G")
                   (s "i̯" "y")
                   (s "u̯" "w")
                   (s "\u0301" "'"))))
  (set-phono-rules 'urgr1 rules)
  (set-phono-rules 'urgr rules))

(let ((rules (list (s "!" "ʰ")
                   (s "G" "gʷ")
                   (s "K" "kʷ")
                   (s "w" "u̯")
                   (s "y" "i̯")
                   (s "ā" "ā")
                   (s "ē" "ɛ̄")
                   (s "ī" "ī")
                   (s "ō" "ɔ̄")
                   (s "ū" "ū")
                   (s "Ā" "ǣ")
                   (s "Ē" "ē")
                   (s "Ō" "ō")
                   (s "~" "\u0303")	; combining tidle
                   (s "'" "\u0301"))))	; combining acute
  (set-pretty-rules 'urgr1 rules)
  (set-pretty-rules 'urgr rules)
  (set-pretty-rules 'lesb rules)
  (set-pretty-rules 'ion-att rules)
  (set-pretty-rules 'ostion rules)
  (set-pretty-rules 'inselion rules)
  (set-pretty-rules 'hom rules)
  (set-pretty-rules 'euboi rules)
  (set-pretty-rules 'att rules)
  (set-pretty-rules 'kret rules)
  (set-pretty-rules 'dor rules)
  (set-pretty-rules 'sdor rules)
  (set-pretty-rules 'sardor rules)
  (set-pretty-rules 'inseldor rules)
  (set-pretty-rules 'kypr rules)
  (set-pretty-rules 'thess rules)
  (set-pretty-rules 'boiot rules)
  (set-pretty-rules 'el rules)
  (set-pretty-rules 'kor rules)
  (set-pretty-rules 'warg rules)
  (set-pretty-rules 'oarg rules)
  (set-pretty-rules 'ark rules)
  )

; Set to #t to have Greek printed out in Greek alphabet
(define usegrkalphabet #f)

(define (togrk str)
 (define rules (list
                    (s "u̯u̯" "u̯")
                    (s "i̯i̯" "i̯")
                    (s "\u301" "'")	; combinding acute

                    (s "a(['~]?)i̯" "αι" 1)
                    (s "a" "α")
                    (s "ā(['~]?)i̯" "ᾱ" 1 "ι")
                    (s "ā" "ᾱ")
                    (s "e(['~]?)i̯" "ει" 1)
                    (s "e" "ε")
                    (s "ē" "ε̄ι")
                    (s "ɛ̄(['~]?)i̯" "η" 1 "ι")
                    (s "ɛ̄" "η")
                    (s "o(['~]?)i̯" "οι" 1)
                    (s "o" "ο")
                    (s "ō" "ο̄")
                    (s "ɔ̄(['~]?)i̯" "ω" 1 "ι")
                    (s "ɔ̄" "ω")
                    (s "u(['~]?)i̯" "υι" 1)
                    (s "u̯" "ϝ")
                    (s "u" "υ")
                    (s "ū" "ῡ")
                    (s "i" "ι")
                    (s "ī" "ῑ")

                    (s "b" "β")
                    (s "g" "γ")
                    (s "d" "δ")
                    (s "sd" "ζ")
                    (s "tʰ" "θ")
                    (s "k" "κ")
                    (s "l" "λ")
                    (s "m" "μ")
                    (s "n" "ν")
                    (s "ks" "ξ")
                    (s "p" "π")
                    (s "r" "ρ")
                    (s "s$" "ς")
                    (s "s" "σ")
                    (s "t" "τ")
                    (s "pʰ" "φ")
                    (s "kʰ" "χ")
                    (s "ps" "ψ")
                    (s "'" "\u0301")   ; combining acute
                    (s "~" "\u0303")   ; combining tilde
                    ))
 (if usegrkalphabet
     (apply-rules rules str)
     str))

\end{verbatim}

\subsection{griechisch.scm}
Diese Datei enthält sämtliche für das Griechische definierte Lautgesetze
inklusive stichwortartiger Notizen.

\begin{verbatim}
(define rules-urgr
  (list
    ;;; Ein Laut der Einfachheit halber
    (s "z" "s")

    ;;; *e neben *H3 und *o fallen zusammen
    (s "O" "o")

    ;;; Kentum -- Rix §92-94
    (s "č" "k")
    (s "ǰ" "g")

    ;;; Thorn -- Rix §91 -- Schindler 'A thorny problem'
    (s "(k|K)þ" 1 "t")
    (s "(g!|G!)þ" 1 "d!")

    ;;; MA > TA -- Rix §94
    (s "(<mediaasp>)" (stimmlos 1))

    ;;; (H)i̯ im Anlaut -- gegen Rix §68,80e
    (s "^y" "dy")
    (s "^(<lary>)y" "y")

    ;;; Laryngale -- Rix §79-85
    ; Rix §85d
    (s "(<kons>)(H1|H2)$" 1 (laryngal->vokal 2))
    ; Rix §85
    (s "(<kons>)(<kons>)(i|u)(H1|H2)$"
       1 2 3 (laryngal->vokal 4))
    (s "(i|u)(H1|H2)$" (sonans->consonans 1) (laryngal->vokal 2))
    ; Dehnung -- Rix §82b,84,85a
    (s "(<vok>)('?)(<lary>)(<kons>|<s-res>|$)"
       (dehnung 1) 2 (sonans->consonans 4))
    ; Rix §80d, §82c
    (s "(<kons>|^)(<lary>)(<kons>)" 1 (laryngal->vokal 2) 3)
    ; Rix §85c
    (s "NH2$" "na")
    ; Rix §84
    (s "(<kons>)(<lary>)(<s-res>)" 1 (laryngal->vokal 2) (sonans->consonans 3))
    ; Rix §83e
    (s "(<s-res>)(<lary>)(<kons>)"
       (sonans->consonans 1) (laryngal->langvokal 2) 3)
    (s "(<s-res>)'(<lary>)(<kons>)"
       (laryngal->vokal 2) "'" (sonans->consonans 1) (laryngal->vokal 2) 3)
    ; Rix §79bc
    (s "^(<lary>)(<s-res>)" (laryngal->vokal 1) (sonans->consonans 2))
    ; kompletter Verlust -- Rix §79a,81a,82d,84
    (s "<lary>")

    ;;; vor Labiovelar > Velar wegen εἶπον -- nicht bei Rix
    (s "wew" "wey")

    ;;; Labiovelar > Velar -- Rix §97
    (s "(w|u|ū)('?)(<labiovelar>)" 1 2 (labiovelar->velar 3))
    (s "(<labiovelar>)(w|u|ū|y)" (labiovelar->velar 1) 2)

    ;;; silbische Resonanten
    ; Rix §76
    (s "(<s-nasal>)(<vok>|<halb-vok>)" "a" (sonans->consonans 1) 2)
    (s "(<s-nasal>)" "ə")
    ; Rix §75
    (s "(<s-liquid>)($|<vok>|<halb-vok>)" "ə" (sonans->consonans 1) 2)
    (s "(<s-liquid>)" (sonans->consonans 1) "ə")

    ;;; Assimilationen (nach Labiovelarwandel noch mal!) -- Rix §78
    (s "(<nasal>)(<labial>)" "m" 2)
    ; gegen Rix nicht *ms > *ns wegen ἔνειμα < *enemsa (oder analog?)
    (s "(<nasal>)(<dental>|<velar>|<labiovelar>)" "n" 2)
    ; Rix §77
    (s "my" "ny")

    ;;; 1. Kontraktion (nach Laryngalverlust)
    ;; TODO: Langvokale? gibt es hier überhaupt welche?
    ; Rix §81
    (s "ee" "ē")
    (s "e'e" "ē~")
    (s "aa" "ā")
    (s "a'a" "ā~")
    (s "oo" "ō")
    (s "o'o" "ō~")
    ; i,u > y,w
    (s "(i|u)('?)(<vok>)" 1 2 (sonans->consonans 1) 3)
    (s "(<vok>)(['~]?)(i|u)" 1 2 (sonans->consonans 3))

    ;;; unmittelbare Assimilation -- Rix §106a
    ; TODO: früher (idg); phonologisch?
    (s "(<okklu>)(<media>)([^!])" (->M 1) 2 3)
    (s "(<okklu>)(<tenuis>)([^!])" (->T 1) 2 3)
    (s "(<okklu>)(<tenuisasp>)([^!])" (->TA 1) 2 3)

    ;; Kw > Kʷ (Beibehaltung des Silbengewichts)
    (s "(<vok>)(['~]?)(<velar>)w" 1 2 (velar->labiovelar 3) (velar->labiovelar 3))
    (s "(<velar>)w" (velar->labiovelar 1))
    (s "(<velar>)(<labiovelar>)" 2 2)

    ;; Beseitigung von Geminaten (αἰπόλος < *ai̯g-kʷolos) -- nicht Rix
    (s "(<kurz-vok>)(['~]?)(<kons>)" 1 2 3 "#")
    (s "GG" "G") (s "KK" "K") (s "K!K!" "K!")
    (s "gg" "g") (s "kk" "k") (s "k!k!" "k!")
    (s "bb" "b") (s "pp" "p") (s "p!p!" "p!")
    (s "dd" "d") (s "tt" "t") (s "t!t!" "t!")
    (s "#")

    ;;; Okklusive neben s -- Rix §105
    ;; Assimilation über s; Tenuis vor s
    (s "(<media>)s(<tenuis>)([^!])" 1 "s" (stimmhaft 2) 3)
    (s "(<tenuisasp>)s(<tenuis>)([^!])" 1 "s" (->A 2) 3)
    (s "(<okklu>)s" (->T 1) "s")

    ; CsC -- Rix §87c
    ; s zwischen gleichen Konsonanten -- TODO: alle Fälle abgedeckt?
    (s "ks(<velar>)" "s" 1)
    (s "Ks(<labiovelar>)" "s" 1)
    (s "ts(<dental>)" "s" 1)
    (s "ps(<labial>)" "s" 1)
    (s "rsr" "sr") (s "lsl" "sl")

    ; vor s > h und Palatalisierung
    'urgr1

    ;;; s > h
    ; s > h \ ($|V)_(V|R) -- Rix §88,89
    (s "(^|(<vok>)(['~]?))s(<vok>|<res>|w)" 1 "h" 4)
    ;; VRsV > VRhV in unbetonten Silben (Datierung?)
    (s "(<vok>)(<res>|w)s(<vok>)" 1 2 "h" 3)
    ;; VNsV > VNhV
    (s "(<vok>)(['~]?)(<nasal>)s(<vok>)" 1 2 3 "h" 4)

    ;; ab hier substrat-s

    ; sy,ys > yy -- Rix §89g, Lejeune §127
    ; TODO: Datierung
    ;       was neben Konsonant (Csy)? vermutlich verlust
    ;       was im Anlaut sy- (ὑμήν)
    (s "(<vok>)(['~]?)sy" 1 2 "yy")
    (s "^sy" "y")
    ; TODO: beispiele (-oi̯si > myk. -o-i)
    (s "ys(<vok>)" "yy" 1)

    ;;; Osthoff -- Rix §58,64
    ; Wohl nach *ns > *nh wegen μηνός < *mēnsós und
    ;      vor Verlust von Okklusiven im Auslaut
    ; *nh (und *nm, *u̯i̯?) nicht betroffen
    (s "nh" "#nh")
    (s "(<lang-vok>)(['~]?)(<res-yw>)(<kons>)" (kuerzung 1) 2 3 4)
    (s "#")

    ;;; Konsonanten im Auslaut (Datierung unklar) -- Rix §100
    ; nach *r̥C > *rəC wegen ὑπόδρα < *upo-dr̥k
    (s "(<okklu>)+$")
    ; Rix §77
    (s "m$" "n")

    ;;; Palatalisierung -- Rix §102
    (s "^(t!|t)y" "s")
    (s "(t!|t)y" "ts")
    (s "(d|g)y" "ǰǰ")
    (s "(k!|k)y" "čč")
    (s "(<labial>)y" "pč")

    (s "ly" "ĺĺ")
    (s "ry" "ŕŕ")
    (s "ny" "ńń")

    ;;; Datierung? -- Rix §105
    (s "dl" "ll")

    ;;; tk, tkw, tp Metathese -- Rix §106 (etwas anders)
    (s "(t)(k|K|p)" 2 1)

    ;;; Datierung unklar, vermutlich später und gestaffelt -- Rix §78
    (s "^m(<liquid>)" "b" 1)
    (s "m(<liquid>)" "bm" 1)
    (s "^n(<liquid>)" "d" 1)
    (s "n(<liquid>)" "nd" 1)

    ;; u- > hu-
    ; datierung unklar
    (s "^u" "hu")

    ;;
    ;; Urgriechisch
    ;;
    'urgr

    ; ti > si -- Rix §101
    ;  Kein wirkliches (ausnahmsloses) Lautgesetz;
    ;    schwierig einigermaßen zufriedenstellend zu formulieren.
    `(sub (sogr) (,(s "([^sk])ti" 1 "si")))

    ; TODO: wann o?
    `(sub (aiol) (,(s "ə" "o")))
    (s "ə" "a")

    (s "pč" "pt")

    ; heteromorphemisches ty
    ; TODO

    (s "^y(<vok>)" "h" 1)
    (s "(<vok>)(['~]?)y(<vok>)" 1 2 "h" 3)

    ;;
    ;; Mykenisch
    ;;
    'myk

    (s "čč" "tš")
    (s "ǰǰ" "ds")

    `(sub (el) (,(s "ē" "Ā")))

    ;;; Labiovelare > T/P -- Rix §96-99
    `(sub (not aiol) (,(s "K(i|ī)" "t" 1)))
    `(sub (not kypr aiol) (,(s "(<labiovelar>)(e|ē)" (labiovelar->dental 1) 2)))
    (s "(<labiovelar>)" (labiovelar->labial 1))
    ;; Assimilationen an Labiovelarreflexe
    (s "(<nasal>)(<labial>)" "m" 2)
    (s "(<nasal>)(<dental>)" "n" 2)
    (s "(<nasal>)(<velar>)" "ŋ" 2)

    ;; b,g > m,ŋ \ _n -- Rix §105
    (s "bn" "mn")
    (s "gn" "ŋn")

    ;;;; ca. 1000

    ;;; tw -- Rix §104
    `(sub (not kret) (,(s "^tw" "s")))
    (s "(.)tw" 1 "tš")
    ;;; tʰw -- unklar, aber vgl. -σθε ai. -dhvam
    (s "(.)t!w" 1 "st!")
    ;(s "^t!w" "t!")

    ;;; ds > sd
    ; vor CsC > ChC wegen ἔρδω < *u̯erzdō < *u̯erg̑i̯e/o-
    (s "ds" "sd")

    ;;; (N|T)sC > sC -- Rix §78, Lejeune §134
    ; vor CsC > ChC wegen ἴσος < u̯idsu̯os und δεσπότης < *demspot-
    ; nach dz > zd wegen σύζυγος < *sun-dzugos
    (s "(<nasal>|<dental>)s(<kons>)" "s" 2)

    ;;; CsC -- Rix §87c, Lejeune §132-133
    ; TODO: πασταδ-/παρταδ- < parstad-
    ;       θύσθεν < *tʰurstʰen
    (s "(<res-yw>)s(<tenuis>|<tenuisasp>)" 1 "#s" 2)
    (s "(<kons>)s(<kons>)" 1 "h" 2)
    (s "(<tenuis>)h" (->A 1))
    (s "(<tenuisasp>)(<tenuis>)([^!])" 1 (->A 2) 3)
    (s "(<tenuisasp>)(<media>)([^!])" (->M 1) 2 3)
    (s "#")

    ;;; u̯i̯ > ẅẅ -- Rix §73
    (s "wy" "ẅẅ")

    ;;; Palatale ŕŕ ĺĺ ńń -- Rix §70
    ; TODO: Datierung? wohl nach myk.
    (s "(a|o)(['~]?)(ŕŕ|ńń|ẅẅ)" 1 2 "y" (depala 3))
    `(sub (kypr) (,(s "a(['~]?)ĺĺ" "a" 1 "yl")))
    (s "ĺĺ" "ll")
    `(sub (thess lesb) (,(s "(ŕŕ|ńń)" (depala 1) (depala 1))))

    ;;; 1. Ersatzdehnung/Gemination
    ;; Palataldehnung -- Rix §73
    (s "(e|i|u)(['~]?)(ŕŕ|ńń|ẅẅ)" (dehnung2 1) 2 (depala 3))
    ;; Rh/hR-Dehnung
    `(sub (not thess lesb) (,(s "(<vok>)(['~]?)(<res>|w)h(<vok>)"
                                (dehnung2 1) 2 3 4)
                            ,(s "(<vok>)(['~]?)h(<res>|w)(<vok>)"
                                (dehnung2 1) 2 3 4)))
    ;; ln-Dehnung
    `(sub (not thess lesb) (,(s "(<vok>)(['~]?)ln(<vok>)" (dehnung2 1) 2 "l" 3)))
    (s "(<vok>)(['~]?)ln(<vok>)" 1 2 "ll" 3)
    ;; 7 vs. 5 Langvokale
    `(sub (not ion-att nwdor sardor) (,(s "Ē" "ē")
                                      ,(s "Ō" "ō")))

    ;; ws > wh
    ; TODO: dialektal rs > rh (oder später rs > rr ?)
    (s "(<vok>)(['~])(w)s(<vok>)" 1 2 3 "h" 4)

    ;; Gemination und Hauchumsprung
    (s "^(<vok>)(['~]?)h(<vok>)([^'~])" "h" 1 2 3 4)
    (s "^(<vok>)(['~]?)(<res>|w)h" "h" 1 2 3 "h")
    ; TODO: prohorā > φρουρά aspirieren
    (s "(<vok>)(['~]?)(<res>|w)h(<vok>)" 1 2 3 3 4)
    (s "(<vok>)(['~]?)h(<res>|w)(<vok>)" 1 2 3 3 4)
    ;; h > 0 (nicht im anlaut)
    (s "(.)h" 1)

    ;;; Cts > Cs
    (s "(<kons>)ts" 1 "s")
    ;;; ts > ss/tt -- Rix §102,87
    `(sub (boiot kret) (,(s "ts" "tt")))
    (s "ts" "ss")
    ; ss > s nach Langvokal
    (s "(<lang-vok>)(['~]?)ss" 1 2 "s")
    `(sub (ion-att ark) (,(s "ss" "s")))

    ;;; ds -- Rix §102
    `(sub (boiot kret lak el) (,(s "sd" "dd")))
    (s "^dd" "d")

    ;;;; ca. 900

    ;;; ion. ā > ǣ
    `(sub (ion-att) (,(s "ā" "Ā")))

    ;;;; ca. 800

    ;;; 2. Ersatzdehnung/Diphthongierung
    ;; TODO: bei allen Vokale gleich? (nicht im Kyrenischen)
    ;; VnsV
    ;; TODO: dialekte? sardor, arg, achaisch?
    `(sub (lesb) (,(s "(<vok>)(['~]?)ns(<vok>)" 1 2 "ys" 3)))
    `(sub (ion-att boiot el lak inseldor kypr)
          (,(s "(<vok>)(['~]?)ns(<vok>)" (dehnung2 1) 2 "s" 3)))
    ;; Vns$
    `(sub (lesb el) (,(s "(<vok>)(['~]?)ns$" 1 2 "ys")))
    `(sub (ion-att boiot lak inseldor)
          (,(s "(<vok>)(['~]?)ns$" (dehnung2 1) 2 "s")))
    ;; Vokaleinreihung
    ;; TODO: dialekte? inseldorisch einteilung (ost west)? thera?
    `(sub (not ion-att nwdor inseldor) (,(s "Ē" "ē")
                                        ,(s "Ō" "ō")))


    ;;; ältere Kontraktionen, TODO: datieren
    ;; wegen Unsicherheit vorläufig auskommentiert
    ; a,ā + e
;    `(sub (ion-att) (,(s "(a|ā)'e" "ā~")
;                     ,(s "(a|ā)e" "ā")))
;    (s "(a|ā)'e" "ē~")
;    (s "(a|ā)e" "ē")
;    ; a+O
;    (s "a'(o|ō|Ō)" "ō~")
;    (s "a(o|ō|Ō)" "ō")
;    ; att. a+e
;    `(sub (att) (,(s "e'a" "ē~")
;                 ,(s "ea" "ē")))
;    ; att. e+o,ō
;    `(sub (att) (,(s "e'o" "Ō~")
;                 ,(s "e'ō" "ō~")
;                 ,(s "eo([^'])" "Ō" 1)
;                 ,(s "eō([^'])" "ō" 1)))
;    ; e + ē
;    (s "e'ē" "ē~")
;    (s "eē" "ē")

    ;;; w > 0
    ; TODO: dialekte. vieles unklar
    ; V_V früh -- Buck §53
    `(sub (ion-att inseldor lesb kret ark) (,(s "(<vok>)(['~]?)w(<vok>)" 1 2 3)))
    ; _V später (TODO: wann > h?)
    `(sub (att) (,(s "^w(<vok>)(['~]?)s(<kons>)" "h" 1 2 "s" 3)))
    `(sub (ion-att inseldor lesb) (,(s "^w(<vok>)" 1)
                                   ,(s "(<vok>)(['~]?)yw(<vok>)" 1 2 "yy" 3)
                                   ,(s "^hw" "h")))
    `(sub (ion-att inseldor) (,(s "^wr" "hr")))

    ;;; hR-
    ;; TODO: dialekte
    `(sub (ion-att) (,(s "^h(l|m|n)" 1)))

    ;;; att. ǣ > ā \ eir_
    `(sub (att) (,(s "(((e|ē|Ē|i|ī)(['~]?))|r|y)Ā" 1 "ā")))

    ;;;; ca. 700

    ;;; 3. Ersatzdehnung/Schwund; e+e, o+o Kontraktion
    ;; TODO: dialekte nach Rix §72; Bartoněk, Buck?
    ; [wo]arg?
    ; sw kret. erhalten
    `(sub (inselion ostion inseldor kret) (,(s "(^|(<vok>)(['~]?))(r|l|n|d|t!|s)w" (dehnung2 1) 4)))
    `(sub (ion-att lesb lak) (,(s "(r|l|n|d|t!|s)w" 1)))
    ; Kontraktion
    (s "e'e" "Ē~")
    (s "ee" "Ē")
    (s "o'o" "Ō~")
    (s "oo" "Ō")
    ;; TODO: dialekte?
    `(sub (not ion-att nwdor sardor) (,(s "Ē" "ē")
                                      ,(s "Ō" "ō")))


    `(sub (att euboi boiot kret) (,(s "tš" "tt")))
    (s "^tt" "t")
    (s "tš" "ss")
    (s "^ss" "s")

    ;; Graßmann -- Rix §107
    (s "(<tenuisasp>)((<kons>)*)(<vok>)(['~]?)((<kons>)*)(<tenuisasp>)"
       (->T 1) 2 4 5 6 8)
    (s "h((<kons>)*)(<vok>)(['~]?)((<kons>)*)(<tenuisasp>)"
       1 3 4 5 7)

    ;;; Psilose -- Rix §68
    `(sub (el lesb ostion inselion kret) (,(s "h")))

    ;;;; ca. homerisch; Sprachstufe der ionischen Archaismen
    'hom

    `(sub (not el inselion) (,(s "Ā" "ē")))

    ;;; jüngere Kontraktionen; teilweise nach Einsetzen der Überlieferung
    ;; wegen Unsicherheit vorläufig auskommentiert
;    ; o + a > ō (nach w > 0)
;    (s "o'a" "ō~")
;    (s "oa" "ō")
;    ; ē + o,ō; TODO: genauere Bedinungen
;    `(sub (att) (,(s "ē'ō" "ō~")))
;    (s "ē(['~]?)(o|ō)" "e" (->acute 1) "ō")
;    ; e + ē (nach ion. ā > ē)
;    (s "e'ē" "ē~")
;    (s "eē" "ē")
;    ; ā + o,ō; TODO: datierung? dialekte (aiol, hom)?
;    `(sub (dor) (,(s "ā(['~])ō" "ā~")
;                 ,(s "āō" "ā")))
;    `(sub (dor) (,(s "ā(['~])o" "ā~")
;                 ,(s "(['~])(.*)āo" 1 2 "ā")
;                 ,(s "āo([^'])" "ā" 1)))

    ))

\end{verbatim}

\subsection{run-gr.scm}
Diese Datei ist mit \str{csi run-gr.scm} auszuführen
und führt die Simulation an den entsprechenden Wörter aus.

\begin{verbatim}
(load "main.scm")

(define run
  (lambda () (load "run-gr.scm")))

(define (run-grassmann dest)
  (print "  " (symbol->string dest))
  (print (run-list 'urgr1 dest (list
    "sékʰō"
    "tʰápʰi̯ō"
    "tʰápʰos"
    "tʰítʰēmi"
    "tʰrikʰós"
    "su̯étʰos"
    "i̯ópʰra"
    )))
  (newline))
(run-grassmann 'att)

(define (run-contr dest)
  (print "  " (symbol->string dest))
  (print (run-list 'urgr1 dest (list
    "enīkae"
    "nīkáei̯"
    "tīmáomen"
    " ́-āo"
    "i̯ā~u̯os"
    "lāu̯ós"
    "nasu̯ós"
    "au̯sṓs"
    "lāu̯okrínēs"
    "-ā́sōm"
    "tʰeāu̯ōrós"
    "potei̯dā́u̯ōn"
    "koināu̯ṓn"
    "u̯étesa"
    "kréu̯as"
    "-éās"
    "génesos"
    "tʰesós"
    "su̯ādéu̯os"
    "pʰiléi̯ō nt i"
    "pʰiléi̯oi̯"
    "-ē~os"
    "hikʷkʷóu̯anaks"
    )))
  (newline))
;(run-contr 'att)
;(run-contr 'ostion)
;(run-contr 'hom)
;(run-contr 'lesb)
;(run-contr 'boiot)
;(run-contr 'kret)

(define (run-misc dest)
  (print "  " (symbol->string dest))
  (print (run-list 'uridg dest (list
    "u̯eu̯r̥h₁mh₁nos"
    "u̯r̥h₁mn̥"
    "h₂ensíi̯eh₂"
    "demspot-"
    "u̯idsu̯os"
    "orsmā́"
    "h₂eu̯sriom"
    "u̯erg̑i̯oh₂"
    "ph₂n̥tih₂"
    "kʷelesi̯etai̯"
    "gelosi̯os"
    )))
  (print (run-list 'urgr1 dest (list
    "plangi̯ō"
    "si̯umēn"
    )))
  (print (run-list 'urgr dest (list
    "sunǰǰugos"
    )))
  (newline))
(run-misc 'urgr)
(run-misc 'att)
(run-misc 'oarg)

(define (run-y dest)
  (print "  " (symbol->string dest))
  (print (togrk (run-list 'urgr1 dest (list
    "káu̯i̯ō"
    "au̯i̯etós"
    "glukéu̯i̯a"
    "eu̯réu̯i̯a"
    "díu̯i̯os"
    "élai̯u̯om"
    "elái̯u̯ā"
    "ai̯u̯ési"
    "pói̯u̯ā"
    ))))
  (newline))
(run-y 'urgr)
(run-y 'att)
(run-y 'lesb)

(define (run-labvel dest)
  (print "  " (symbol->string dest))
  (print (run-list 'uridg dest (list
    "kʷis"
    "kʷe"
    "kʷidkʷe"
    "gʷou̯kʷolos"
    "gʷih₃os"
    "h₁ln̥gʰus"
    "kʷukʷlos"
    "penkʷe"
    "gʷelbʰus"
    "gʷʰeni̯oh₂"
    "gʷʰonos"
    "gʷm̥i̯oh₂"
    "kʷoineh₂"
    "kʷinu̯oh₂"
    "snigʷʰm̥"
    "ogʷʰis"
    "penkʷtos"
    "kʷl̥Him"
    "-gʷʰnet"
    "sens"
    )))
  (newline))
(run-labvel 'urgr1)
(run-labvel 'urgr)
(run-labvel 'att)
(run-labvel 'lesb)
(run-labvel 'ark)

(define (run-ky dest)
  (print "  " (symbol->string dest))
  (print (run-list 'urgr1 dest (list
    "ki̯ā́meron"
    "pʰuláki̯ō"
    "glṓki̯a"
    "tu̯éi̯sō"
    "tʰu̯ari̯ós"
    )))
  (print (run-list 'uridg dest (list
    "kʷétu̯r̥es"
    "-dʰu̯e"
    )))
  (newline))
(run-ky 'ostion)
(run-ky 'euboi)
(run-ky 'att)
(run-ky 'boiot)
(run-ky 'kret)
(run-ky 'sdor)

(define (run-s-test dest)
  (print "  " (symbol->string dest))
  (print (run-list 'urgr1 dest (list
    "asa"
    "a'sa"
    "asra"
    "a'sra"
    "asla"
    "a'sla"
    "asma"
    "a'sma"
    "asna"
    "a'sna"
    "aswa"
    "a'swa"
    "sa"
    "sra"
    "sla"
    "sma"
    "sna"
    "swa"
    )))
  (newline))
(run-s-test 'urgr)
(run-s-test 'att)

(define (run-2ed dest)
  (print "  " (symbol->string dest))
  (print (run-list 'urgr1 dest (list
    "-ánti̯a"
    "-ónti̯a"
    "-énti̯a"
    "-áns"
    "-óns"
    "-éns"
    )))
  (newline))
(run-2ed 'urgr)
(run-2ed 'lesb)
(run-2ed 'thess)
(run-2ed 'boiot)
(run-2ed 'ostion)
(run-2ed 'el)
(run-2ed 'ark)
(run-2ed 'kret)
(run-2ed 'inseldor)

(define (run-wau dest)
  (print "  " (symbol->string dest))
  (print (run-list 'urgr1 dest (list
    "kalu̯ós"
    "ksénu̯os"
    "kóru̯ā"
    "néu̯ā"
    "kʰṓrā"
    "du̯ei̯nós"
    "dédu̯imen"
    )))
  (newline))
(run-wau 'att)
(run-wau 'ostion)
(run-wau 'inseldor)
(run-wau 'kret)
(run-wau 'el)
(run-wau 'kor)
(run-wau 'ark)
(run-wau 'boiot)
(run-wau 'lesb)

(define (run-s dest)
  (print "  " (symbol->string dest))
  (print (run-list 'urgr1 dest (list
    "ploksmós"
    "ekstrós"
    "ai̯ksmā́"
    "persnā́"
    "orsmā́"
    "parstádes"
    "tʰúrstʰen"
    "prosorā́"
    "akóu̯sō"

    "éu̯sō"
    "éserpon"
    "iserós"

    "īsáomai̯"
    "isáni̯ō"
    "esús"
    "īsā́tro-"
    "witswos"
    )))
  (newline))
(run-s 'urgr)
(run-s 'att)
(run-s 'lesb)

(define (run-1ed dest)
  (print "  " (symbol->string dest))
  (print (run-list 'urgr1 dest (list
    "korsā́"
    "akou̯sā́"
    "au̯sṓs"
    "ou̯satós"
    "kórsā"
    "akóu̯sō"
    "éu̯sō"
    "au̯si̯ō"
    "kʰésras"
    " selásnā"
    "kʰánsas"
    "énemsa"
    "kʷélson"
    "nasu̯ós"
    "gʷolnā́"
    "mēnsós"
    "mḗns"
    "u̯ésma"
    )))
  (newline))
(run-1ed 'urgr)
(run-1ed 'att)
(run-1ed 'inseldor)
(run-1ed 'kret)
(run-1ed 'lesb)

(define (run-pala dest)
  (print "  " (symbol->string dest))
  (print (run-list 'uridg dest (list
    "gʷm̥i̯oh₂"
    "komi̯os"
    "ali̯os"
    )))
  (print (run-list 'urgr1 dest (list
    "katʰari̯ō"
    "pʰtʰeri̯ō"
    "oi̯ktiri̯ō"
    "kteni̯ō"
    "krini̯ō"
    "pluni̯ō"
    )))
  (newline))
(run-pala 'ion-att)
(run-pala 'kypr)
(run-pala 'thess)
(run-pala 'boiot)

(define (run-myk)
  (let ((dest 'ion-att))
    (print (run-list 'uridg dest (list
      "Hi̯os"
      "trei̯es"
      "dh₃tis"
      "di̯ḗu̯s"
      "di̯ḗm"
      "toti̯os"
      "medʰi̯os"
      "podsi"
      "mr̥tós"
      "h₂n̥rós"
      )))
    (run-list 'urgr dest (list
      "pantsa"
      ))))
(run-myk)

(define (run-gr-kw dest)
  (print "  " (symbol->string dest))
  (print (run-list 'uridg dest (list
    "ai̯gkʷólos"
    "h₁ék̑u̯os"
    )))
  (newline))
(run-gr-kw 'urgr)
(run-gr-kw 'att)

(define (run-gr-s)
  (print (run-list 'uridg 'att (list
    "au̯si̯ō"
    "psde-"
    "pn̥dʰskoh₂"
    "dikskos"
    "legʰskeh₂"
    "migskoh₂"
    "u̯oi̯dtʰh₂e"
    "u̯iddʰi"
    "parstádes"
    "tʰúrstʰen"
    "kenttós"
    "tu̯e-"
    "etu̯e-"
    ))))
(run-gr-s)

\end{verbatim}


\end{document}
